\documentclass{article}
\textwidth=6in
\hoffset=0in
\voffset=0in

\usepackage[a4paper, total={6in, 8in}]{geometry}
\usepackage{amsmath}
\usepackage{amssymb}
\usepackage{stmaryrd}
\usepackage{graphicx}
\usepackage{tikz}
\usetikzlibrary{automata, arrows}
\usepackage{pifont}
\usepackage{amssymb}
\usepackage{gensymb}
\usepackage{ngerman}
\usepackage[ampersand]{easylist}

% needs to be updated
\author{Max Springenberg, 177792}
\title{\
    EA Uebungsblatt 2
    }
\setcounter{section}{2}
\date{}

% custom commands
% \Theta \Omega \omega
\newcommand{\tab}{\null\ \qquad}
\newcommand{\gap}{\null\ \\ \ \\}
\newcommand{\la}{\leftarrow}
\newcommand{\ra}{\rightarrow}
\newcommand{\lra}{\leftrightarrow}
\newcommand{\ue}{\infty}
\newcommand{\eps}{\epsilon}
\newcommand{\task}[1]{\textbf{#1} \\ \gap}
\newcommand{\cmark}{\ding{51}}
\newcommand{\xmark}{\ding{55}}
\newcommand{\degr}{\degree}

% content
\begin{document}
% title page
\maketitle
\newpage
% actual paper
\subsection{
    Tiefensuche, Breitensuche, Zusammenhang, starker Zusammenhang: Wiederholung 
        und Verständnisfragen
    }
\subsubsection{
    Wie ist der Ablauf von Tiefensuche und Breitensuche in ungerichteten 
        Graphen?
    }
In der Tiefensuche wird erst der neuexn Knoten abgearbeitet. In der Breitensuche
    werden immer erst die neuen Knoten abgearbeitet.\\
\subsubsection{
    Auf welcher Datenstruktur werden die Algorithmen durchgeführt?
    }
Knoten werden in eine Queue zwischengespeichert und in einem Array/ einer Map
    mit random access markiert.\\
Verbindungen/ Kanten werden mittels einer Adjazenzliste /-matrix realisiert.\\
Nummern/ IDs f"ur die Ordnung hinsichtlich der Entdeckung der Knoten in 
    DFS-Liste.\\
\subsubsection{
    Unter welchen Voraussetzungen ist die Höhe eines Breitensuche-Baumes 
        eindeutig? Gilt bei diesen Voraussetzungen auch, dass der 
        Breitensuche-Baum eindeutig ist?
    }
Die H"ohe des Baums ist immer eindeutig, dessen Zusammensetzung jedoch nicht.\\
\subsubsection{
    Ist für einen gegebenen Graphen jeder Breitensuche-Baum stets höchstens so 
        tief wie jeder Tiefensuche-Baum?
    }
Ja, da der BFS-Baum generell fr"uher die Knoten abhandelt und dadurch eine 
    gleichm"a"sigere Aufteilung der Kinder entsteht.\\
\subsubsection{
    Welche zusätzlichen Kantenklassen gibt es bei Tiefensuche auf gerichteten 
        Graphen?
    }
Baum (T-), Forward (F-), Backward (B-) und Kreuz (C-) Kanten. Sonst gleich.\\
\subsubsection{
    Wie lassen sich die Kantenklassen im Tiefensuche-Algorithmus unterscheiden?
    }
\begin{tabular}{l|l}
    Baum (T-)       & neu/ direkt gefunden\\
    Forward (F-)    & Niedriegere auf h"ohere DFS-Nummer\\
    Backward (B-)   & H"ohere auf niedrigere DFS-Nummer\\
    Kreuz (C-)      & alle anderen Kanten\\
\end{tabular}
\subsubsection{
    Wie sind Zusammenhangskomponenten (ZHK) und starke Zusammenhangskomponenten 
        (SZHK) definiert? Worin unterscheiden diese sich?
    }
Existiert eine gerichteter weg von Knoten v nach w mit $v, w \in V$, so 
    schreiben wir $v \ra w$.\\
Starkter Zusammenhang ist definiert als 
    $v \lra w \equiv v \ra w \land w \ra v$\\
\\
Zusammenhangskomponenten sind f"ur ungerichtete Graphen definiert und besagen,
    dass man zwischen zwei Knoten hin und her gehen kann.\\
\subsubsection{
    Welchen Algorithmus haben Sie in der Vorlesung kennen gelernt, um die SZHKs 
        eines Graphen zu berechnen? Wie lautet die asymptotische Laufzeit?
    }
In der Vorlesung wurde der Algorithmus von Ksaraju vorgestellt.\\
Bei diesem Algorithmus word zwei mal eine DFS durchgef"uhrt. Dabei werden 
    f-Nummern in absteigender Reihenfolge verteilt. Anschlie"send werden alle
    Kanten invers gerichtet. Abschlie"send wird noch eine DFS-Traversierung
    durchgef"uhrt, ausgehend von f[v] = 1. Die Dabei entstehenden T-Kanten
    sind starke Zusammenhangskomponenten.\\
Die Laufzeit bel"auft sich auf die der DFS mit $O(|V| + |E|)$\\
\newpage
\subsection{Starker Zusammenhang – Algorithmus durchführen}
\task{DFS-, f-, ZHK-Nummern}
G\\
\begin{tabular}{lll}
    Knoten      &DFS-Nummer     &f-Nummer\\
    A           &1              &7\\
    B           &2              &6\\
    C           &4              &4\\
    D           &3              &5\\
    E           &5              &3\\
    F           &6              &2\\
    G           &7              &1\\
\end{tabular}
$G^*$\\
\begin{tabular}{ll}
    Knoten      &DFS-Nummer
\end{tabular}
ZHK\\
\begin{enumerate}
    \item G,E
    \item F
    \item C,A,B,D
\end{enumerate}
\gap
\task{Kantenklassen}
$G$:\\$
    T = \{(A,B), (B,D), (D,C), (E,F), (E,G)\}\\
    F = \{(A,D)\}\\
    B = \{(G,E)\}\\
    C = \{G,F\}\\
    $\\
$G^*$:\\$
    T = \{(G,E), (C,D), (D,A), (D,B)\}\\
    F = \emptyset\\
    B = \{(E,G), (A,C)\}\\
    C = \{(F,E), (F,G), (B,A), (B,F)\}\\
    $\\
\subsection{(Starker) Zusammenhang - Eigenschaften}
\subsubsection{
    Sei G ein beliebiger gerichteter Graph. Sind die SZHKs auf G eindeutig?
    }
Die SZHK sind eindeutig, nicht jedoch die Reihenfolge, in der sie gefunden
    werden.\\
\subsubsection{
    Sei G ein beliebiger gerichteter Graph. Ergeben sich auf G∗ immer die 
        gleichen SZHKs wie auf G?
    }
Ja, die da sich Zusammenhangskomponenten bei invertierung des Graphens bestehen
    bleiben.\\
Beweis:\\$
    A \ra B \not \equiv B \ra A = A \ra^{-1} B\\
    \\
    A \lra B \equiv A \ra B \land B \ra A\\
    A \lra^{-1} B \equiv B \ra A \land A \ra B \equiv A \lra B\\
    $
Damit verliert nach Invertierung der Richtungen eine Starke 
    Zusammenhangskomponenten nicht ihre Eigenschaft und ein nicht in der 
    "Aquivalenzrelation inbegriffenes Tupel erh"alt auch nicht die 
    Eigenschaft.\\
\subsubsection{
    Betrachten Sie den Korrektheitsbeweis des Algorithmus von Kosaraju. Bei dem
        DFS-Durchlauf auf $G^∗$ kann es vorkommen, dass eine Kante von 
        $T_x$ nach $T_{x+1}$ verläuft. Welcher Kantenklasse werden diese Kanten
        zugeordnet?
    }
\subsection{
    Beschreiben Sie einen Algorithmus, der das bipartite Matchingproblem mit 
        Hilfe eines Flussalgorithmus löst. Welche Laufzeit hat Ihr Algorithmus?
    }
In der Vorlesung wurde ein Algorithmus vorgestallt, bei dem "ahnlich, wie bei
    Hopfield und Karp eine Quelle und Senke angef"ugt wurde. Zudem wurden alle
    Knoten der bipartiten Mengen je nach Mengenzugeh"origkeit mit Quelle oder 
    Senke Verbunden und Kanten zur Senke gerichtet.\\
Jeder Knoten hat die Flusskapazit"at 1. Es wird sein maximaler Fluss 
    mittels dem Algorithmus von Ford und Fulkerson berechnet.\\
Die Laufzeit betr"agt dabei $O(B  (|V | + |E|))$ und ist echt 
    pseudopolynomiell.\\
\end{document}
