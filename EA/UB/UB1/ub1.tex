\documentclass{article}
\textwidth=6in
\hoffset=0in
\voffset=0in

\usepackage[a4paper, total={6in, 8in}]{geometry}
\usepackage{amsmath}
\usepackage{amssymb}
\usepackage{stmaryrd}
\usepackage{graphicx}
\usepackage{tikz}
\usetikzlibrary{automata, arrows}
\usepackage{pifont}
\usepackage{amssymb}
\usepackage{gensymb}
\usepackage{ngerman}
\usepackage[ampersand]{easylist}

% needs to be updated
\author{Max Springenberg, 177792}
\title{\
    EA Uebungsblatt 1
    }
\setcounter{section}{1}
\date{}

% custom commands
% \Theta \Omega \omega
\newcommand{\tab}{\null\ \qquad}
\newcommand{\gap}{\null\ \\ \ \\}
\newcommand{\lA}{\leftarrow}
\newcommand{\rA}{\rightarrow}
\newcommand{\ue}{\infty}
\newcommand{\eps}{\epsilon}
\newcommand{\task}[1]{\textbf{#1} \\ \gap}
\newcommand{\cmark}{\ding{51}}
\newcommand{\xmark}{\ding{55}}
\newcommand{\degr}{\degree}

% content
\begin{document}
% title page
\maketitle
\newpage
% actual paper
\subsection{Wiederholung}
\task{Was ist ein matching und wie ist es definiert?}
Ein Matching is die Teilmenge $E_M$ von einerm Graphen $G(E,V)$, die nur Kanten
    enth"alt, sodass jeder Knoten nur jeweils eiener Kante zugeordnet ist.\[
        E_M \subseteq E \land \forall e, e' \in E_M : e \cap e' = \emptyset
    \]
\gap
\task{\
    Was ist ein M-Verbessernder Pfad? Wie viele M-Verbessernde Pf"ade koennen
        fuer ein optimales Matching $M_{opt}$ gefunden werden?
    }
Ein M-Verbsessernder Pfad ist ein Pfad im Graphen, der zwischen Matching- und 
    nicht Matching-Kanten altaniert und dabei mit einer nicht Matching-Knoten
    anf"angt und aufh"ort.\\
Werden bei einem M-Verbesserndem Pfad Matching und ncht Matching-Kanten
    hinsichtlich ihrer zugeh"origkeit zu M geswapt steigt $|M|$ um 1.\\
Ist ein Matching M optimal, so kann es nichtmehr verbessert werden.
    Dementsprechend ist die Anzahl von M-Verbessernden Pfaden in $M_{opt}$
    gleich Null.\\
\gap
\task{\
    Warum ist Matching auf bipartiten Graphen einfacher? Welche Teile des
        Algorithmus von Hopcroft und Karp sind nicht direkt auf allgemeine
        Graphen "ubertragbar?
    }
Bipartite Graphen sind dadurch definiert, dass ein bipartiter Graph $G(V,E)$
    in Teilmengen $V_1, V_2$ unterteilt werden kann, sodass alle Kanten nur
    zwischen Knoten aus $V_1, V_2$ verlaufen.\\
\end{document}
