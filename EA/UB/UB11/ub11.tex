\documentclass{article}
\textwidth=6in
\hoffset=0in
\voffset=0in

\usepackage[a4paper, total={6in, 8in}]{geometry}
\usepackage{amsmath}
\usepackage{amssymb}
\usepackage{stmaryrd}
\usepackage{graphicx}
\usepackage{tikz}
\usetikzlibrary{automata, arrows}
\usepackage{pifont}
\usepackage{amssymb}
\usepackage{gensymb}
\usepackage{ngerman}
\usepackage[ampersand]{easylist}

% needs to be updated
\author{Max Springenberg, 177792}
\title{\
    EA Uebungsblatt 11
    }
\setcounter{section}{11}
\date{}

% custom commands
% \Theta \Omega \omega
\newcommand{\tab}{\null\ \qquad}
\newcommand{\gap}{\null\ \\ \ \\}
\newcommand{\la}{\leftarrow}
\newcommand{\ra}{\rightarrow}
\newcommand{\lra}{\leftrightarrow}
\newcommand{\ue}{\infty}
\newcommand{\eps}{\epsilon}
\newcommand{\task}[1]{\textbf{#1} \\ \gap}
\newcommand{\cmark}{\ding{51}}
\newcommand{\xmark}{\ding{55}}
\newcommand{\degr}{\degree}

% content
\begin{document}
% title page
\maketitle
\newpage
% actual paper

% Fragen
\subsection\

% A2
\subsection\
Wir brauchen einen Algorithmus, der mit Wahrscheinlichkeit von\[
    P = \frac{w(e_i)}{W}, 
    \text{ mit $W = \sum_{e \in E} w(e)$ und $e_i \in E$}
    \]
ein einzelnes Zufallsexperiment in $O(|V|)$ durchf"uhrt.\\
\gap
Da wir pro Knoten nur eine Operation machen d"urfen m"ussen wir "uber die
    knoten iterieren. Dabei muss allerdings die Wahrscheinlichkeit aufrecht
    erhalten werden.\\
Dazu ben"otigen wir ein Preprocessing, dass in $O(V^2)$ vor dem eigentlichem 
    Algorithmus ausgef"uhrt wird.\\
Beaobachtung:\[
    \frac{w(e_i)}{W} = \frac{w(e_i)}{w'} \frac{w'}{W}
    \]
, mit $w' \in \mathbb{R}$\\
Aus dieser Beaobachtung folgt das Preprocessing, dass vor der schleife des
    Hauptalgorithmus in $O(|E|) = O(|V|^2)$ festlegt, welche Kanten pro
    Knoten betrachtet werden sollen, ohne dass die Bedingung verletzt wird.\\
\gap
$pre\_pro(G = (V,E)):$\\
\begin{enumerate}
    \item berechne $W =  \sum_{e \in E} w(e)$
    \item unmarked = $Matrix(|V|,|V|, init=\top)$
    \item prob = $\emptyset$
    \item num = 1
    \item $\forall v \in V:$
    \item \tab $w'$ = $\sum_{e \in adj(v)} w(e)$
    \item \tab $\forall e = \{x,y\} \in adj(v)$:
    \item \tab \tab IF $unmarked[x][y]$ THEN:
    \item \tab \tab \tab mit W'keit von $ num * w(e) / w'$ DO:
    \item \tab \tab \tab \tab prob = prob $\cup \{(e, w'/W)\}$
    \item \tab \tab \tab \tab $unmarked[u][v] = \bot$
    \item \tab \tab \tab \tab BREAK inner for-loop
    \item \tab \tab num += w(e) / W
    \item return prob
\end{enumerate}
Der eigentliche Algorithmus wird dann wie folgt angepasst:\\
\begin{enumerate}
    \item prob = pre\_prob(G)
    \item Repeat
    \item \tab num = 1
    \item \tab $\forall (e=\{u,v\}, w) \in$ prob:
    \item \tab \tab mit W'keit von $num * w$:
    \item \tab \tab \tab l"osche $(e=\{u,v\}, w)$ aus prob
    \item \tab \tab \tab G := Kontraktion(G,u,v)
    \item \tab \tab \tab BREAK inner for-loop
    \item \tab \tab num += w(e) / W
    \item Until $|V| = 2$
    \item Gib den durch die beiden Knoten definierten Schnitt aus.\\
\end{enumerate}
\end{document}
