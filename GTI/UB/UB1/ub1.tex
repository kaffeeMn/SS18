\documentclass{article}
\usepackage{amsmath}
\usepackage{amssymb}
\usepackage{stmaryrd}
\usepackage{graphicx}
\usepackage{tikz}
\usetikzlibrary{automata, arrows}
\usepackage{pifont}
\usepackage{amssymb}
\usepackage{gensymb}
\usepackage[ampersand]{easylist}

% needs to be updated
\author{Max Springenberg, 177792}
\title{\
    GTI Uebungsblatt 1
    }
\setcounter{section}{1}
\date{}

% custom commands
% \Theta \Omega \omega
\newcommand{\tab}{\null\ \qquad}
\newcommand{\gap}{\null\ \\ \\}
\newcommand{\lA}{\leftarrow}
\newcommand{\rA}{\rightarrow}
\newcommand{\ue}{\infty}
\newcommand{\eps}{\epsilon}
\newcommand{\task}[1]{\textbf{#1} \\ \gap}
\newcommand{\cmark}{\ding{51}}
\newcommand{\xmark}{\ding{55}}
\newcommand {\degr}{\degree}

% content
\begin{document}
% title page
\maketitle
\newpage
% actual paper

% begin A1
\subsection\
\subsubsection{\
    Seien $\beta = (ab)^* $ und $ \alpha_1, ... , \alpha_8 $ die folgenden 
        erweiterten regulären Ausdrücke. Beurteilen Sie für alle
        $i \in\{3, ... , 8\}$, ob $L(\alpha_i) \subseteq L(\beta)$ gilt. 
        Vervollständigen Sie dazu die folgende Tabelle analog zu den 
        Beispiel-Ausdrücken $\alpha_1, \alpha_2$: 
        Falls $L(\alpha_i \not \subseteq L(\beta)$ gilt, geben Sie ein Wort 
        $w_i \in L(\alpha_i) - L(\beta)$ an.
    }
\begin{tabular}{l|l|l}
    RE              &$L(\alpha_i) \subseteq L(\beta)$   &Gegenbeispiel\\
    \hline \\
    $\alpha_1 = (ab)^*(ab)^*$                   & \cmark    &\\
    $\alpha_2 = (ba)^*$                         & \xmark    & ba\\
    $\alpha_3 = (a*b*)$                         & \xmark    & aa\\
    $\alpha_4 = (b?a?)^*$                       & \xmark    & aa\\
    $\alpha_5 = (a + \eps)(b + \eps)(a*b*)^*$   & \xmark    & aa\\
    $\alpha_6 = (ab*ab*)^*$                     & \cmark    &\\
    $\alpha_7 = (abab)^*$                       & \cmark    &\\
    $\alpha_8 = a(ba)^+ b$                      & \cmark    &\\
\end{tabular}
\subsubsection{\
    Das Verhalten eines Netzwerk-Controllers soll anhand protokollierter 
        Ausgaben analysiert wer- den. Der Controller schreibt, abhängig von der 
        Eingabe, beliebig lange Bitfolgen. Eine Bitfolge wird als Wort über dem 
        Alphabet $\{0, 1\}$ repräsentiert. Die Menge der gültigen Ausgaben wird
        im  Handbuch des Controllers formal als Sprache 
        $L = \{0,1\}^* - L(\gamma)$ für den erweiterten regulären Ausdruck\[
            \gamma = (0+1)^* (000 + 111)(0+1)^* + (10)*1? + (01)^*0
            \]
        spezifiziert. Beschreiben Sie L natürlichsprachlich kurz in einem Satz.
    }
In $\gamma$ enthaltene Woerter haben mindestens eine der folgenden
Eigenschaften:\\
\begin{itemize}
    \item Das Wort enthaelt drei aufeinanderfolgenede Nullen oder Einsen
    \item Das Wort beginnt mit 10 gefolgt von einer 1 oder $\eps$ und endet
    \item Das Wort beginnt mit 01 gefolgt von einer 0 oder $\eps$ und endet
\end{itemize}
, L enthaelt nur Woerter ueber $\{0, 1\}$, fuer die das nicht der Fall ist,
    also Woerter die alle der folgenden Eigenschaften gelten:\\
\begin{itemize}
    \item Das Wort enthaelt keine drei aufeinanderfolgenede Nullen oder Einsen
    \item Das Wort beginnt nicht mit 10 gefolgt von einer 1 oder $\eps$ und 
            endet
    \item Das Wort beginnt nicht mit 01 gefolgt von einer 0 oder $\eps$ und 
            endet
\end{itemize}

\gap
% begin A2
\subsection{
    Geben Sie im Folgenden reguläre Ausdrücke bzw. erweiterte reguläre Ausdrücke
        an. Beschreiben Sie für jede Konstruktion kurz, warum Ihr Ausdruck die 
        Sprache beschreibt (warum er alle Wörter der Sprache erzeugt und warum
        er kein Wort außerhalb der Sprache erzeugt).
    }
\subsubsection{\
    Sei $\Sigma = \{0,\ldots,9,\oplus,\ominus,.,\circ,C\}$ .\\
    Konstruieren Sie einen erweiterten regulaeren Ausdruck $\alpha$ ueber 
        $\Sigma$, der genau die gueltigen Temparaturangaben mit zwei 
        nachkommastellen in Grad Celsius beschreibt.\\
    }
\task{
    Eine Solche Temperaturangabe ist gueltig, wenn sie :\\
    \begin{itemize}
        \item genau zwei Nachkommastellen besitzt
        \item keine ueberfluessigen fuehrenden Nullen im ganzzahligen Anteil
                ausfweisst
        \item den Minimalwert von $-273.15 \degr C$ nicht unterschreitet
        \item mit dem Zusatz $\degr C$ endet
    \end{itemize}\ \\
    }
Positive Temperaturangaben beginnen mit $\oplus$ oder keinem Vorzeichen,
    negative Temperaturangaben mit $\ominus$.\\
Der ganzzahlige Bereich darf keine fuehrenden Nullen aufweisen, wenn dieser
    nicht genau 0 ist. Ausserdem darf der Ganzzahlige Bereich nicht leer sein.\\
Der regulaere Ausdruck\[
        u = 0 + (1-9)^+(0-9)^*
    \]
    erfuellt genau diese Kriterien.\\
Nachkommastellen koennen beliebige Ziffern sein, aber es muessen genau zwei
    vorkommen. Nachkommstellen erfolgen unmittelbar nach dem Zeichen `.'.\\
Der regulare Ausdruck\[
        v = .(0-9)^2
    \]
erfuellt genau diese Kriterien.\\
Waehrend Temperaturangaben unendlich gross werden duerfen wird eine untere
    Grenze $-273.15 \degr C$ angegeben. Das bedeutet, dass im negativen Bereich
    sicher gestellt werden muss, dass keine Zahl kleiner ist.\\
    Der regulaere Ausdruck\[
        n = \ominus(
            0v
            + (1-9)((1-9)?)v    % up to  99.99
            + 1(0-9)^2v         % up to 199.99
            + 2(1-6)(0-9)v      % up to 269.99
            + 27(1-2)v          % up to 272.99
            + 273.(0+1)(0-5)    % up to 273.15
            )
        \] 
    erfuellt die Kriterien fuer den numerischen Anteil von Temperaturangaben im 
    negativen Bereich.\\
Der Suffix der Temperaturangabe muss genau $\degr C$ sein.\\
Der regulaere Ausdruck\[
        c = \degr C
        \]
    erfuellt genau dieses Kriterium.\\
\\
Aus dieser Regel und den oben angegebenen Teilausdruecken ergibt sich 
    $\alpha$ mit\[
        \alpha = (\oplus ? uv + \ominus n)c
        \]
\subsubsection{\
    Seien $\Sigma = \{a,b\} 
        , L_1 = \{\sigma\sigma u | \sigma \in \Sigma u \in \Sigma^*\}
        , L_2 = \{\tau w \tau | \tau \in \Sigma w \in \Sigma^*\}$\\
    Geben Sie regulaere Ausdruecke $\alpha_1,\alpha_2,\alpha$ an, mit
        $L_1 = L(\alpha_1), L_2 = L(\alpha_2), L = L_1 \cap L_2$ an.
    }
\task{$L_1$}
In $L_1$ enthaltene Woerter besizten mindest zwei beliebigen Zeichen aus 
    $\Sigma$ beginnen und koennen mit darauf folgend mit beliebig vielen
    Zeichen aus $\Sigma$ konkatiniert werden.\\
Der entsprechende regulaere Ausdruck lautet:\[
    \alpha_1 = (a+b)^2 (a+b)^*
    \]
\task{$L_2$}
In $L_2$ enthaltene Woerter fangen mit einem beliebigem Zeichen aus $\Sigma$ an
    und enden mit einem beliebigem Zeichen aus $\Sigma$. Zwischen diesen Zeichen
    koennen beliebige Zeichen aus Sigma vorkommen.\\
Der entsprechende regulaere Ausdruck lautet:\[
    \alpha_2 = (a+b)(a+b)^*(a+b)
    \]
\task{$L$}
Im folgendem wird die Aeuquivalenz zwischen $\alpha_1$ und $\alpha_2$ und ferner
    die Aequivalenz von $L_1$ und $L_2$, sowie dann auch $L$ bewiesen.\\
\gap
\[
    \alpha_1 = (a+b)^2(a+b)^*
        \equiv (a+b)(a+b)(a+b)^*
        \equiv^{(\beta \beta^* \equiv \beta^* \beta)} 
            (a+b)(a+b)^*(a+b)
        \equiv \alpha_2
    \]
Fuer den regulaeren Ausdruck $\alpha$ muss gelten, dass er nur Woerter
    beschreibt, die auch $\alpha_1,\alpha_2$ beschreiben. Durch die Aequivalenz
    von $\alpha_1,\alpha_2$ muss damit dann auch gelten\[
    \alpha \equiv \alpha_1 \equiv \alpha_2
    \]
Dies ist etwa mit $\alpha = \alpha_1$ gegeben.\\
Damit gilt \[
    L(\alpha) = L(\alpha_1);
    L(\alpha_1) \equiv L(\alpha_2);
    L(\alpha )= L(\alpha) \cap L(\alpha)
        = L(\alpha_1) \cap L(\alpha_1)
        \equiv L(\alpha_1) \cap L(\alpha_2) 
    \]
\subsubsection{\
    Nach dem Standard ISO 8601 wird ein Datum in der Form JJJJ-MM-TT notiert.
        Beispielsweise wird der Geburtstag Alan Turings, der 23. Juni 1912, 
        durch 1912-06-23 repräsentiert. Konstruieren Sie einen erweiterten 
        regulären Ausdruck über dem Alphabet $\Sigma = \{0,1,...,9, \ominus\}$,
        der alle gültigen Daten des Jahres 2018 beschreibt.
    }
Es sind folgende regeln zu beachten:\\
(i) Das Jahr ist 2018 und wird von einem $\ominus$ gefolgt\\
\\
Der regulaere Ausdruck\[
    \beta = 2018 \ominus
    \]
erfuellt genau diese Regel.\\
\gap
(ii) Ein Monat besteht aus genau zwei Ziffern und darf nicht kleiner als 01 oder
    groesser als 12 sein. Der Februar hat nur 28 Tage
    , die Monate $\{01,04,06,08,10,12\}$ koennen bis zu 31 
    und die Monate $\{03,05,07,09,11\}$ koennen bis zu 30 Tage haben.
    Monate und Tage sind mit einem $\ominus$ konkatiniert.
    Tage bestehen aus zwei zwei Ziffern und koennen nicht kleiner als 01 sein.\\
\\
$m_{31} = (01+04+06+08+10+12), m_{30} = (03+05+07+09+11), m_{28} = 02$ seien die
    regulaeren Ausdruecke fuer Monate mit 31, 30 oder 28 Tagen.\\
Darueber ergibt sich fuer die Tage und Monate der regulaere Ausdruck\\
\\
$\gamma = m_{31} \ominus (0(1-9)+(1+2)(0-9)+3(0+1))\\
    \tab + m_{30} \ominus (0(1-9)+(1+2)(0-9)+30))\\
    \tab + m_{28} \ominus (0(1-9)+1(0-9)+2(0-8))
    $\\
\gap
Der aus den Regeln resultierende regulaere Ausdruck ist\[
    \alpha = \beta (\gamma)
    \]
\end{document}
