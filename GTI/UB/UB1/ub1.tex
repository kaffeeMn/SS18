\documentclass{article}
\textwidth=6in
\hoffset=0in
\voffset=0in

\usepackage[a4paper, total={6in, 8in}]{geometry}
\usepackage{amsmath}
\usepackage{amssymb}
\usepackage{stmaryrd}
\usepackage{graphicx}
\usepackage{tikz}
\usetikzlibrary{automata, arrows}
\usepackage{pifont}
\usepackage{amssymb}
\usepackage{gensymb}
\usepackage{ngerman}
\usepackage[ampersand]{easylist}

% needs to be updated
\author{Max Springenberg, 177792}
\title{\
    GTI "Ubungsblatt 1\\
    Tutor: Marko Schmellenkamp\\
    ID: MS1\\
    "Ubung: Mi 16-18
    }
\setcounter{section}{1}
\date{}

% custom commands
% \Theta \Omega \omega
\newcommand{\tab}{\null\ \qquad}
\newcommand{\gap}{\null\ \\ \\}
\newcommand{\lA}{\leftarrow}
\newcommand{\rA}{\rightarrow}
\newcommand{\ue}{\infty}
\newcommand{\eps}{\epsilon}
\newcommand{\task}[1]{\textbf{#1} \\ \gap}
\newcommand{\cmark}{\ding{51}}
\newcommand{\xmark}{\ding{55}}
\newcommand{\degr}{\null \degree}
\newcommand{\error}{\task{FEHLER:}}
\newcommand{\correction}{\task{KORREKTUR:}}
\newcommand{\mdef}{\overset{\text{def}}{=}}

\tiksset[initial text={}]

% content
\begin{document}
% title page
\maketitle
\newpage
% actual paper

% begin A1
\subsection\
\subsubsection{\
    Seien $\beta = (ab)^* $ und $ \alpha_1, ... , \alpha_8 $ die folgenden 
        erweiterten regul"aren Ausdr"ucke. Beurteilen Sie f"ur alle
        $i \in\{3, ... , 8\}$, ob $L(\alpha_i) \subseteq L(\beta)$ gilt. 
        Vervollst"andigen Sie dazu die folgende Tabelle analog zu den 
        Beispiel-Ausdr"ucken $\alpha_1, \alpha_2$: 
        Falls $L(\alpha_i \not \subseteq L(\beta)$ gilt, geben Sie ein Wort 
        $w_i \in L(\alpha_i) - L(\beta)$ an.
    }
\begin{tabular}{l|l|l}
    RE              &$L(\alpha_i) \subseteq L(\beta)$   &Gegenbeispiel\\
    \hline \\
    $\alpha_1 = (ab)^*(ab)^*$                   & \cmark    &\\
    $\alpha_2 = (ba)^*$                         & \xmark    & ba\\
    $\alpha_3 = (a^*b^*)$                       & \xmark    & aa\\
    $\alpha_4 = (b?a?)^*$                       & \xmark    & aa\\
    $\alpha_5 = (a + \eps)(b + \eps)(ab)^*$     & \xmark    & aab\\
    $\alpha_6 = (ab^*ab^*)^*$                   & \xmark    & aab\\
    $\alpha_7 = (abab)^*$                       & \cmark    &\\
    $\alpha_8 = a(ba)^+ b$                      & \cmark    &\\
\end{tabular}
\subsubsection{\
    Das Verhalten eines Netzwerk-Controllers soll anhand protokollierter 
        Ausgaben analysiert werden. Der Controller schreibt, abh"angig von der 
        Eingabe, beliebig lange Bitfolgen. Eine Bitfolge wird als Wort "uber dem 
        Alphabet $\{0, 1\}$ repr"asentiert. Die Menge der g"ultigen Ausgaben wird
        im  Handbuch des Controllers formal als Sprache 
        $L = \{0,1\}^* - L(\gamma)$ f"ur den erweiterten regul"aren Ausdruck\[
            \gamma = (0+1)^* (000 + 111)(0+1)^* + (10)*1? + (01)^*0?
            \]
        spezifiziert. Beschreiben Sie L nat"urlichsprachlich kurz in einem Satz.
    }

%%%% Fehler
\error
In $\gamma$ enthaltene W"orter haben mindestens eine der folgenden
Eigenschaften:\\
\begin{itemize}
    \item Das Wort enth"alt drei aufeinanderfolgenede Nullen oder Einsen
    \item Das Wort beginnt mit 10 gefolgt von einer 1 oder $\eps$ und endet
    \item Das Wort beginnt mit 01 gefolgt von einer 0 oder $\eps$ und endet
\end{itemize}
, L enthaelt nur W"orter "uber $\{0, 1\}$, fuer die das nicht der Fall ist,
    also W"orter, f"ur die alle der folgenden Eigenschaften gelten:\\
\begin{itemize}
    \item Das Wort enth"alt keine drei aufeinanderfolgenede Nullen oder Einsen
    \item Das Wort beginnt nicht mit 10 gefolgt von einer 1 oder $\eps$ und 
            endet
    \item Das Wort beginnt nicht mit 01 gefolgt von einer 0 oder $\eps$ und 
            endet
\end{itemize}

\gap
%%%% Korrigiert
\correction
In $\gamma$ enthaltene W"orter haben mindestens eine der folgenden
Eigenschaften:\\
\begin{itemize}
    \item Das Wort enth"alt drei aufeinanderfolgenede Nullen oder Einsen
    \item Das Wort besteht aus eine beliebigen Folge von 10 endend auf 1 
            oder $\eps$
    \item Das Wort besteht aus eine beliebigen Folge von 01 endend auf 0 
            oder $\eps$
\end{itemize}
, L enthaelt nur W"orter "uber $\{0, 1\}$, fuer die das nicht der Fall ist,
    also W"orter, f"ur die alle der folgenden Eigenschaften gelten:\\
\begin{itemize}
    \item Das Wort enth"alt keine drei aufeinanderfolgenede Nullen oder Einsen
    \item Das Wort besteht nicht aus eine beliebigen Folge von 10 endend auf 1 
            oder $\eps$
    \item Das Wort besteht nicht aus eine beliebigen Folge von 01 endend auf 0 
            oder $\eps$
\end{itemize}

% begin A2
\subsection{
    Geben Sie im Folgenden regul"are Ausdr"ucke bzw. erweiterte regul"are 
        Ausdr"ucke an. Beschreiben Sie f"ur jede Konstruktion kurz, warum Ihr 
        Ausdruck die Sprache beschreibt (warum er alle W"orter der Sprache 
        erzeugt und warum er kein Wort außerhalb der Sprache erzeugt).
    }
\subsubsection{\
    Sei $\Sigma = \{0,\ldots,9,\oplus,\ominus,.,\circ,C\}$ .\\
    Konstruieren Sie einen erweiterten regul"aren Ausdruck $\alpha$ "uber 
        $\Sigma$, der genau die g"ultigen Temparaturangaben mit zwei 
        nachkommastellen in Grad Celsius beschreibt.\\
    }
\task{
    Eine Solche Temperaturangabe ist g"ultig, wenn sie :\\
    \begin{itemize}
        \item genau zwei Nachkommastellen besitzt
        \item keine "uberfluessigen f"uhrenden Nullen im ganzzahligen Anteil
                ausfweisst
        \item den Minimalwert von $-273.15 \degr C$ nicht unterschreitet
        \item mit dem Zusatz $\degr C$ endet
    \end{itemize}\ \\
    }
%%%% Fehler
\error
Positive Temperaturangaben beginnen mit $\oplus$ oder keinem Vorzeichen,
    negative Temperaturangaben mit $\ominus$.\\
\\
Der ganzzahlige Bereich darf keine f"uhrenden Nullen aufweisen, wenn dieser
    nicht genau 0 ist. Au"serdem darf der Ganzzahlige Bereich nicht leer sein.\\
Der regul"are Ausdruck\[
        u = 0 + (1-9)(0-9)^*
    \]
    erf"ullt genau diese Kriterien.\\
\\
Nachkommastellen k"onnen beliebige Ziffern sein, aber es muessen genau zwei
    vorkommen. Nachkommstellen erfolgen unmittelbar nach dem Zeichen `.'.\\
Der regul"are Ausdruck\[
        v = .(0-9)^2
    \]
erf"ullt genau diese Kriterien.\\
\\
W"ahrend Temperaturangaben unendlich gross werden d"urfen, wird eine untere
    Grenze $-273.15 \degr C$ angegeben. Das bedeutet, dass im negativen Bereich
    sicher gestellt werden muss, dass keine Zahl kleiner ist.\\
    Der regul"are Ausdruck\[
        n = \ominus(
            0v
            + (1-9)((1-9)?)v    % up to  99.99
            + 1(0-9)^2v         % up to 199.99
            + 2(1-6)(0-9)v      % up to 269.99
            + 27(1-2)v          % up to 272.99
            + 273.(0+1)(0-5)    % up to 273.15
            )
        \] 
    erf"ullt die Kriterien f"ur den numerischen Anteil von Temperaturangaben im 
    negativen Bereich.\\
\\
Der Suffix der Temperaturangabe muss genau $\degr C$ sein.\\
Der regul"are Ausdruck\[
        c = \degr C
        \]
    erf"ullt genau dieses Kriterium.\\
\\
Aus diesen Regel und den oben angegebenen Teilausdr"ucken ergibt sich 
    $\alpha$ mit\[
        \alpha = (\oplus ? uv + n)c
        \]

\gap
%%%% Korrigiert
\correction
Positive Temperaturangaben beginnen mit $\oplus$ oder keinem Vorzeichen,
    negative Temperaturangaben mit $\ominus$.\\
\\
Der ganzzahlige Bereich darf keine f"uhrenden Nullen aufweisen, wenn dieser
    nicht genau 0 ist. Au"serdem darf der Ganzzahlige Bereich nicht leer sein.\\
Der regul"are Ausdruck\[
    u \mdef 0 + [1-9][0-9]^*
    \]
    erf"ullt genau diese Kriterien.\\
\\
Nachkommastellen k"onnen beliebige Ziffern sein, aber es muessen genau zwei
    vorkommen. Nachkommstellen erfolgen unmittelbar nach dem Zeichen `.'.\\
Der regul"are Ausdruck\[
        v \mdef .[0-9]^2
    \]
erf"ullt genau diese Kriterien.\\
\\
W"ahrend Temperaturangaben unendlich gross werden d"urfen, wird eine untere
    Grenze $-273.15 \degr C$ angegeben. Das bedeutet, dass im negativen Bereich
    sicher gestellt werden muss, dass keine Zahl kleiner ist.\\
    Der regul"are Ausdruck\[
        n \mdef \ominus(
            0v
            + [1-9]([0-9]?)v    % up to  99.99
            + 1[0-9]^2v         % up to 199.99
            + 2[1-6][0-9]v      % up to 269.99
            + 27[1-2]v          % up to 272.99
            + 273.(0[0-9]+1[0-5])% up to 273.15
            )
        \] 
    erf"ullt die Kriterien f"ur den numerischen Anteil von Temperaturangaben im 
    negativen Bereich, da die erste Veroderung die Zahlen bis 99.99, die
    Zweite die Zahlen bis 199.99, die Dritte die Zahlen bis 269.99, die Vierte
    die Zahlen bis 272.99 und die F"unfte die Zahlen bis 273.15 abdeckt.\\
\\
Der Suffix der Temperaturangabe muss genau $\degr C$ sein.\\
Der regul"are Ausdruck\[
        c \mdef \degr C
        \]
    erf"ullt genau dieses Kriterium.\\
\\
Aus diesen Regel und den oben angegebenen Teilausdr"ucken ergibt sich 
    $\alpha$ mit\[
        \alpha \mdef (\oplus ? uv + n)c
        \]

\newpage
% begin A2
\subsubsection{\
    Seien $\Sigma = \{a,b\} 
        , L_1 = \{\sigma\sigma u | \sigma \in \Sigma, u \in \Sigma^*\}
        , L_2 = \{\tau w \tau | \tau \in \Sigma, w \in \Sigma^*\}$\\
    Geben Sie regul"are Ausdr"ucke $\alpha_1,\alpha_2,\alpha$ an, mit
        $L_1 = L(\alpha_1), L_2 = L(\alpha_2), L = L_1 \cap L_2$ an.
    }

%%%% Fehler
\error
\task{$L_1$}
In $L_1$ enthaltene W"orter m"ussen mit mindest zwei beliebigen Zeichen aus 
    $\Sigma$ beginnen und k"onnen darauf folgend mit beliebig vielen
    Zeichen aus $\Sigma$ konkatiniert werden.\\
Der entsprechende regul"are Ausdruck lautet:\[
    \alpha_1 \mdef (a+b)^2 (a+b)^*
    \]
\task{$L_2$}
In $L_2$ enthaltene W"orter fangen mit einem beliebigem Zeichen aus $\Sigma$ an
    und enden mit einem gleichem Zeichen aus $\Sigma$. Zwischen diesen Zeichen
    k"onnen beliebige Zeichen aus Sigma vorkommen.\\
Der entsprechende erweiterte regul"are Ausdruck lautet:\[
    \alpha_2 \mdef a(a+b)^*a + b(a+b)^*b
    \]
\task{TODO: Bewei"s ohne "Aquivalenz-Kette}
\task{$L$}
Im Folgendem wird die "Auquivalenz zwischen $\alpha_1$ und $\alpha_2$ und ferner
    die Gleichheit von $L_1$ und $L_2$, sowie dann auch $L$ gezeigt.\\
\gap
\[
    \alpha_1 = (a+b)^2(a+b)^*
        \equiv (a+b)(a+b)(a+b)^*
        \overset{{(\beta \beta^* \equiv \beta^* \beta)}}{\equiv}
            (a+b)(a+b)^*(a+b)
        \equiv \alpha_2
    \]
\gap
Da $\alpha_1,\alpha_2$ "aquivalent und damit der Schnitt der durch sie erzeugten
    Sprachen gleich den Sprachen selbst ist, muss
    fuer den regul"aren Ausdruck $\alpha$ gelten, dass er nur W"orter
    beschreibt, die auch $\alpha_1,\alpha_2$ beschreiben.\[
    \alpha \equiv \alpha_1 \equiv \alpha_2
    \]
Dies ist etwa mit $\alpha = \alpha_1$ gegeben.\\
Damit gilt \[
    L(\alpha) = L(\alpha_1) = L(\alpha_2);
    L(\alpha )= L(\alpha) \cap L(\alpha)
        = L(\alpha_1) \cap L(\alpha_1)
        = L(\alpha_1) \cap L(\alpha_2) 
    \]

%%%% Korrigiert
\correction
\task{$L_1$}
In $L_1$ enthaltene W"orter m"ussen mit mindest zwei gleiche Zeichen aus 
    $\Sigma$ beginnen und k"onnen darauf folgend mit beliebig vielen
    Zeichen aus $\Sigma$ konkatiniert werden.\\
Der entsprechende regul"are Ausdruck lautet:\[
    \alpha_1 \mdef a^2 (a+b)^* + b^2 (a+b)^*
    \]
\task{$L_2$}
In $L_2$ enthaltene W"orter fangen mit einem beliebigem Zeichen aus $\Sigma$ an
    und enden mit einem beliebigem Zeichen aus $\Sigma$. Zwischen diesen Zeichen
    k"onnen beliebige Zeichen aus Sigma vorkommen.\\
Der entsprechende erweiterte regul"are Ausdruck lautet:\[
    \alpha_2 \mdef (a+b)(a+b)^*(a+b)
    \]
\task{$L$}
Im folgendem wird die "Auquivalenz zwischen $\alpha_1$ und $\alpha_2$ und ferner
    die Gleichheit von $L_1$ und $L_2$, sowie dann auch $L$ gezeigt.\\
\gap
\[
    \alpha_1 = (a+b)^2(a+b)^*
        \equiv (a+b)(a+b)(a+b)^*
        \equiv^{(\beta \beta^* \equiv \beta^* \beta)} 
            (a+b)(a+b)^*(a+b)
        \equiv \alpha_2
    \]
\gap
Da $\alpha_1,\alpha_2$ "aquivalent und damit der Schnitt der durch sie erzeugten
    Sprachen gleich den Sprachen selbst ist, muss
    fuer den regul"aren Ausdruck $\alpha$ gelten, dass er nur W"orter
    beschreibt, die auch $\alpha_1,\alpha_2$ beschreiben.\[
    \alpha \equiv \alpha_1 \equiv \alpha_2
    \]
Dies ist etwa mit $\alpha \mdef \alpha_1$ gegeben.\\
Damit gilt \[
    L(\alpha) = L(\alpha_1) = L(\alpha_2);
    L(\alpha )= L(\alpha) \cap L(\alpha)
        = L(\alpha_1) \cap L(\alpha_1)
        = L(\alpha_1) \cap L(\alpha_2) 
    \]



\subsubsection{\
    Nach dem Standard ISO 8601 wird ein Datum in der Form JJJJ-MM-TT notiert.
        Beispielsweise wird der Geburtstag Alan Turings, der 23. Juni 1912, 
        durch 1912-06-23 repr"asentiert. Konstruieren Sie einen erweiterten 
        regul"aren Ausdruck "uber dem Alphabet $\Sigma = \{0,1,...,9, \ominus\}$,
        der alle g"ultigen Daten des Jahres 2018 beschreibt.
    }
Es sind folgende Regeln zu beachten:\\
(i) Das Jahr ist 2018 und wird von einem $\ominus$ gefolgt\\
\\
Der regul"are Ausdruck\[
    \beta \mdef 2018 \ominus
    \]
erfuellt genau diese Regel.\\
\gap
(ii) Ein Monat besteht aus genau zwei Ziffern und darf nicht kleiner als 01 oder
    groe"ser als 12 sein. Der Februar hat nur 28 Tage
    , die Monate $\{01,04,06,08,10,12\}$ koennen bis zu 31 
    und die Monate $\{03,05,07,09,11\}$ koennen bis zu 30 Tage haben.
    Monate und Tage sind mit einem $\ominus$ konkatiniert.
    Tage bestehen aus zwei Ziffern und k"onnen nicht kleiner als 01 sein.\\
\\
$m_{31} \mdef (01+04+06+08+10+12)
    , m_{30} \mdef (03+05+07+09+11)
    , m_{28} \mdef 02$ 
    seien die regul"aren Ausdruecke f"ur Monate mit 31, 30 oder 28 Tagen.\\
Daraus ergibt sich fuer die Tage und Monate der regul"are Ausdruck\\
\\
% Fehler
\error
$\gamma = m_{31} \ominus (0(1-9)+(1+2)(0-9)+3(0+1))\\
    \tab + m_{30} \ominus (0(1-9)+(1+2)(0-9)+30))\\
    \tab + m_{28} \ominus (0(1-9)+1(0-9)+2(0-8))
    $\\
\gap
Der aus den Regeln resultierende regul"are Ausdruck ist\[
    \alpha = \beta (\gamma)
    \]
\gap
%Korrigiert
\correction
$\gamma \mdef m_{31} \ominus (0[1-9]+(1+2)[0-9]+3(0+1))\\
    \tab + m_{30} \ominus (0[1-9]+(1+2)[0-9]+30))\\
    \tab + m_{28} \ominus (0[1-9]+1[0-9]+2[0-8])
    $\\
\gap
Der aus den Regeln resultierende regul"are Ausdruck ist\[
    \alpha \mdef \beta (\gamma)
    \]
\end{document}
