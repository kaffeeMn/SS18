\documentclass{article}
\textwidth=6in
\hoffset=0in
\voffset=0in


\usepackage[a4paper, total={6in, 8in}]{geometry}
\usepackage{amsmath}
\usepackage{amssymb}
\usepackage{stmaryrd}
\usepackage{graphicx}

\usepackage{tikz}
\usetikzlibrary{automata, arrows}
\tikzset{initial text={}}

\usepackage{pifont}
\usepackage{amssymb}
\usepackage{gensymb}
\usepackage{ngerman}
\usepackage[ampersand]{easylist}

% needs to be updated
\author{Max Springenberg, 177792}
\title{\
    GTI "Ubungsblatt 5\\
    Tutor: Marko Schmellenkamp\\
    ID: MS1\\
    "Ubung: Mi 16-18
    }
\setcounter{section}{5}
\date{}

% custom commands
% \Theta \Omega \omega
\newcommand{\tab}{\null\ \qquad}
\newcommand{\gap}{\null\ \\ \\}
\newcommand{\lA}{\leftarrow}
\newcommand{\rA}{\rightarrow}
\newcommand{\LA}{\Leftarrow}
\newcommand{\RA}{\Rightarrow}
\newcommand{\ue}{\infty}
\newcommand{\eps}{\epsilon}
\newcommand{\task}[1]{\textbf{#1} \\ \gap}
\newcommand{\cmark}{\ding{51}}
\newcommand{\xmark}{\ding{55}}
\newcommand{\degr}{\null \degree}
\newcommand{\error}{\task{FEHLER:}}
\newcommand{\correction}{\task{KORREKTUR:}}
\newcommand{\mdef}{\overset{\text{def}}{=}}
\newcommand{\rao}[1]{\overset{#1}{\rightarrow}}
\newcommand{\automaton}[1]{
    \begin{tikzpicture}
    #1
    \end{tikzpicture}
    }
\newcommand{\nd}[4]{
    \node[#1](#2)at(#3){#4};
    }


% content
\begin{document}
% title page
\maketitle
\newpage
% actual paper

% A1
\subsection\
% A1 a)
\subsubsection\
Eine m"ogliche L"osung die die Grammatik $G$ mit:\\
$
\begin{array}{llll}
    S   &\rA    &\eps   &| S'\\
    S'  &\rA    &D      &| H\\
    D   &\rA    &aDbb   &| abb\\
    H   &\rA    &aaHb   &| aab\\
\end{array}\\
$\\
Ein Beweis war nicht erforderlich, die Motivation hinter der Struktur der 
    Grammatik ist:\\
\\
Ein Wort kann entweder halb oder doppelt soviele $b$'s, wie $a$'s haben.\\
$L(G)$ enth"alt das leere wort, da $|\eps| = 0$ gilt und $0$ unter
    Multiplikation idempotent ist ($2 * \#_a(\eps) = 2 * 0 = \#_b(\eps)$)\\
Der Fall f"ur doppelt soviele $b$'s wie $a$'s ist durch die Variable $D$ 
    abgedeckt, der Fall f"ur halb soviel durch die Variable $H$.\\
Von dem Startsymbol aus kann nur genau eine, oder keine der beiden Variablen 
    erreicht werden, die Ableitungsb"aume der Variablen haben keine Gemeinsamen
    Variablen.\\

% A1 b)
\subsubsection\
Eine m"ogliche L"osung ist die Grammatik $G$ mit:\\
$
\begin{array}{lllllllll}
    V   &\rA    &str    &| num       &| true &| false &| null &| A &| O\\
    O   &\rA    &\{\}   &| \{V_o\}\\
    V_o &\rA    &str:V  &| V_o, V_o\\
    A   &\rA    &[]     &| [V_a]\\
    V_a &\rA    &V      &| V_a,V_a\\
\end{array}\\
$, mit dem Startsymbol $V$.\\
Ein Beweis wurde nicht gefordert, die Motivation hinter der Konstruktion der 
    Grammatik ist:\\
\\
Es sollten genau die g"ultigen JSON $values$ beschrieben werden, damit ist das
    leere Wort $\eps$ nicht in der Sprache $L(G)$, da anstelle dessen in JSON
    $null$ verwendet wird.\\
Ein $value$ kann aus einem der Terminalsymbolen $num, str, true false, null$ 
    oder einem Array oder Objekt bestehen.\\
Neben den trivialen Regeln f"ur die Terminalsymbole ergeben sich die Variablen
    O, f"ur Objekte, und A, f"ur Arrays, durch die vorgeschriebene Syntax.\\
Werte in einem Array sind durch je ein Komma getrennt und k"onnen s"amtliche
    $values$ enthalten.\\
Wertepaare in einem Objekt haben $str$ als Schl"ussel, ein $value$ als Wert
    und sind wie auch im Array durch je einem Komma getrennt.\\

% A2
\subsection\
% A2 a)
\subsubsection\
Die Menge $V_e$ ergibt sich zu:\\
$V_e = \{C, D, S, A,, B\}$\\
Nicht erzeugende Variablen sind: $E$\\
\\
Die Menge wurde gebildet, indem zuerst alle Variablen mit regeln, die eine 
    Regel mit nur Terminalsymbolen enthalten hinzugef"ugt. Darauf folgend
    solange Variablen, die Regeln mit nur Variablen aus der Menge enthalten,
    hinzugef"ugt, bis keine mehr gefunden werden.\\
\\
$E$ und Alle Regeln, die $E$ enthalten werden aus der Grammatik $G_0$ entfernt.\\
\\
Der Erreichbarkeitsgraph, der erzeugenden Variablen ergibt sich zu:\\
\automaton{
    \nd{state}{S}{0,0}{S};
    \nd{state}{A}{2,-2}{A};
    \nd{state}{B}{2,0}{B};
    \nd{state}{C}{3,-1}{C};
    \nd{state}{D}{0,-2}{D};
    \path
        (S)
            edge [->] node {} (B)
            edge [->] node {} (D)
        (B)
            edge [->, loop above] node {} (B)
            edge [->] node {} (D)
            edge [->] node {} (C)
        (A)
            edge [->, loop left] node {} (A)
            edge [->] node {} (B)
            edge [->] node {} (C)
        ;
    }\\
nicht erreichbare Variablen sind: $A$\\
\\
Dies kann z.B. durch BFS oder DFS auf der Startsymbol S im Graphen ermittelt
    werden.\\
\\
A und alle Regeln, die A enthalten werden entfernt.\\
\gap
Daraus ergibt sich die Grammatik $G_0$ mit:$\\
\begin{array}{lllll}
    S &\rA  &Bb     &| Da\\
    B &\rA  &bBD    &| Bb   &| C\\
    C &\rA  &c      &| D    &| B\\
    D &\rA  &a\\
\end{array}
$
% A2 b)
\subsubsection\
CNF 2:\\
Die Grammatik $G'$ ergibt sich nach CNF2 zu:\\
$
\begin{array}{lllll}
    S   &\rA    &BBW_b      &| W_bW_c\\
    A   &\rA    &B          &| AW_a &| W_c\\
    B   &\rA    &BABAW_a    &| BW_b &| C\\ 
    C   &\rA    &W_c        &| A    &| B\\
\end{array}\\
$\\
Dabei wird f"ur jedes vorkommende Terminalsymbol $\sigma$ eine Variable 
    $W_\sigma$ eingef"uhrt, die als einzige Regel das jeweilige Terminalsymbol 
    enth"alt. Danach werden alle vorherigen Vorkommnisse des Terminalsymbols
    durch die jeweilige Variable ersetzt.\\
\gap
CNF 3:\\
die Grammatik $G_1$ ergibt sich nach CNF3 zu:\\
$
\begin{array}{lllll}
    S   &\rA &BS_1  &| W_bW_c\\
    S_1 &\rA &BW_b\\
    A   &\rA &B     &| AW_a &| W_c\\
    B   &\rA &BB_1  &| BW_b &| C\\
    B_1 &\rA &AB_2\\
    B_2 &\rA &BB_3\\
    B_3 &\rA &AW_a\\
    C   &\rA &W_c   &| A    &| B\\
    W_a &\rA &a\\
    W_b &\rA &b\\
    W_c &\rA &c\\
\end{array}\\
$\\
Dabei werden die rechten Seiten so verk"urzt, dass die Bedingung gilt, dass
    nur je zwei Variablen konkateniert vorkommen.
    Dies geschieht durch einf"ugen von Variablen, die Teilregeln ersetzen unter
    dieser Bedingung.\\

% A2 c)
\subsubsection\
CNF4:
Die Grammatik $G'$ ergibt sich nach CNF4 zu:\\
$
V' = \{C,B,A\}\\
\gap
\begin{array}{lllllll}
    S   &\rA &W_bA  &| W_b  &| BW_c &| W_c\\
    A   &\rA &B     &| AW_a &| W_a  &| W_c\\
    B   &\rA &BW_a  &| W_a  &| BW_b &| W_b  &| C\\
    C   &\rA &W_c   &| A    &| B\\
    W_a &\rA &a\\
    W_b &\rA &b\\
    W_c &\rA &c\\
\end{array}
$\\
Die Menge $V'$ ergibt sich durch das Hinzuf"ugen von Variablen mit Regeln
    die das Terminalsymbol $\eps$ enthalten und anschlie"send Variablen mit 
    Regeln die zu Variablen f"uhren, die in der Menge $V'$ enthalten sind, bis
    keine mehr gefunden werden.\\
Anschlie"send werden alle regeln mit $\eps$ entfernt und f"ur jede Regel mit
    Variablen aus $V'$ neue Regeln ohne diese Variablen angef"ugt.\\
\\
CNF5:\\
Die Grammatik $G_3$ ergibt sich nach CNF5 zu:\\
$
U = \{
    (A,B), (A,W_a), (A,W_c), (A,W_b), (A,C)\\
    , (B,W_a), (B,W_b), (B,C), (B,W_c), (B,A)\\
    , (C,W_c), (C,A), (C,W_a), (C,B), (C,W_b)
    \}\\
\gap
\begin{array}{lllllllll}
    S   &\rA &W_bA  &| b &| BW_c &| c\\
    A   &\rA &BW_a  &| a &| BW_b &| b &| c &| A     &| AW_a\\
    B   &\rA &BW_a  &| a &| BW_b &| b &| c &| AW_a  &| B\\
    W_a &\rA &a\\
    W_b &\rA &b\\
    W_c &\rA &c\\
\end{array}
$\\
Die Menge U ergibt sich durch Variablen, die in einer Regel auf nur eine
    Variable verweisen, die nicht gleich ihnen selbst ist.\\
Abschlie"send werden die Einheitsregeln durch die von ihnen erreichten 
    nicht-Einheitsregeln ersetzt.\\
Zus"atzlich werden alle nicht erreichbaren Regeln entfernt.\\

% A3
\subsection\
% A3 a)
\subsubsection\
gegeben:\\
Grammatik $G$ mit:$\\
    S \rA aSbb | abb\\
    $\\
Sprache $L$ mit:$\\
    L = \{a^nb^m | n,m \in \mathbb{N} \land m \geq 2n\}\\
    $\gap
z.z.: $\forall w \in L(G): w = a^nb^m, n,m \in \mathbb{N}, m \geq 2n)$\\
\\
Wir f"uhren eine Induktion "uber die Wortl"ange $n = |w|$,
    mit der Menge der Wortl"angen von $L(G)$:\[
        N_L = \{3n | n \in \mathbb{N}\}
    \]
Ich nehme an, dass ich f"ur $N_L$ keinen Induktionsbeweis f"uhren muss, da
    die Wortl"ange aller W"orter aus der Sprache der Grammatik mit den einzigen 
    Regeln $S \rA abb | aSbb$ gleich 3 beim kleinsten, 3+3 beim n"achst 
    kleinsten Element ist und S induktiv definiert ist.\\
Es geht mehr darum $k$ aus einer sinnvolleren Menge als $\mathbb{N}$ zu 
    w"ahlen.\\
\gap
Aussage: \[
    \forall k \in N_L:
        w = a^nb^m, w \in L(G), |w| = n:, (n,m \in \mathbb{N}, m \geq 2n)\\
    \]\
\\
I.A.$\\
\\
k = 3 \text{, da $abb$ kleinstes Element der Sprache L(G) ist}\\
\\
\nexists w \in L(G): w \not \equiv abb \land |w| = 3
    \text{
        , da $S \RA^G abb$ 
        die einzige Ableitung f"ur W"orter der L"ange 3 ist.
        }\\
\\
\text{F"ur $w=abb$:}\\
\#_a(w) = 1, \#_b(w) = 2\\
2 = 2*1\\
\text{damit gilt auch } 2 \geq 2*1\\
\\
\text{Dadurch wurde gezeigt, dass die Aussage f"ur k=3 gilt.}\\
$\\
\\
I.V.\\
Die Aussage gelte f"ur $k' \in N_L$ beliebig, aber fest.\\
\\
I.S.$\\
k = k'+3\\
\\
\text{definiert seien:}\\
w_k \text{, mit: } |w| = k, w_k \in L(G)\\
w_{k'} \text{, mit: } |w| = k', w_{k'} \in L(G)\\ 
\\
\text{nach der Ableitungsregel von $S$ gilt:}\\
w_k = aw_{k'}bb\\
\\
\text{nach der I.V. gilt:}\\
\#_b(w_{k'}) \geq 2 * \#_a(w_{k'})\\
\\
\text{Daraus folgt:}\\
2 * \#_a(w_{k})
    = 2 * (1+\#_a(w_{k'}))
    = 2 + 2 * \#_a(w_{k'})
    \overset{I.V.}{\leq} 2 + \#_b(w_{k'})
    = \#_b(w_k)\\
\\
\text{Damit wurde die Aussage f"ur beliebige $k \in N_L$ gezeigt.}\\
$\\

% A3 b)
\subsubsection\
Annahme $L \subseteq L(G)$:\\
\\
Daraus w"urde folgen:\\
$
\forall w \in L: w \in L(G)\\
\\
w \mdef abbb\\
w \in L, w \not\in L(G) \lightning\\
$\\
w ist nicht in $L(G)$, da $L(G)$ unter anderem keine W"orter der L"ange 4 
    enth"alt.\\
\\
Damit gilt die Aussage nicht.
\end{document}
