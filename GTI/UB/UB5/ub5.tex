\documentclass{article}
\textwidth=6in
\hoffset=0in
\voffset=0in


\usepackage[a4paper, total={6in, 8in}]{geometry}
\usepackage{amsmath}
\usepackage{amssymb}
\usepackage{stmaryrd}
\usepackage{graphicx}

\usepackage{tikz}
\usetikzlibrary{automata, arrows}
\tikzset{initial text={}}

\usepackage{pifont}
\usepackage{amssymb}
\usepackage{gensymb}
\usepackage{ngerman}
\usepackage[ampersand]{easylist}

% needs to be updated
\author{Max Springenberg, 177792}
\title{\
    GTI "Ubungsblatt 5\\
    Tutor: Marko Schmellenkamp\\
    ID: MS1\\
    "Ubung: Mi 16-18
    }
\setcounter{section}{5}
\date{}

% custom commands
% \Theta \Omega \omega
\newcommand{\tab}{\null\ \qquad}
\newcommand{\gap}{\null\ \\ \\}
\newcommand{\lA}{\leftarrow}
\newcommand{\rA}{\rightarrow}
\newcommand{\LA}{\Leftarrow}
\newcommand{\RA}{\Rightarrow}
\newcommand{\ue}{\infty}
\newcommand{\eps}{\epsilon}
\newcommand{\task}[1]{\textbf{#1} \\ \gap}
\newcommand{\cmark}{\ding{51}}
\newcommand{\xmark}{\ding{55}}
\newcommand{\degr}{\null \degree}
\newcommand{\error}{\task{FEHLER:}}
\newcommand{\correction}{\task{KORREKTUR:}}
\newcommand{\mdef}{\overset{\text{def}}{=}}
\newcommand{\rao}[1]{\overset{#1}{\rightarrow}}
\newcommand{\automaton}[1]{
    \begin{tikzpicture}
    #1
    \end{tikzpicture}
    }
\newcommand{\nd}[4]{
    \node[#1](#2)at(#3){#4};
    }


% content
\begin{document}
% title page
\maketitle
\newpage
% actual paper

% A1
\subsection\
% A2 a)
\subsubsection\
$
S   \rA \eps | S'\\
S'  \rA D | H\\
D   \rA AD'BB\\
D'  \rA AD'BB | \eps\\
H   \rA AAH'B\\
H'  \rA AAH'B | \eps\\
A   \rA a\\
B   \rA b\\
$
% A2 b)
\subsubsection\
$
S   \rA \eps | S'\\
S'  \rA L | D | S'S'\\
L   \rA [E_l] | []\\
E_l \rA V | D | L | E_l,E_l\\
D   \rA {E_d} | {}\\
E_d \rA V:V | V:D | V:L | E_d,E_d\\
V   \rA str | num | true | false | null\\ 
$

% A2
\subsection\
% A2 a)
\subsubsection\
nicht erzeugende variablen sind: $E$\\
nicht erreichbare variablen sind: $A$\\
\gap
Daraus ergibt sich die Grammatik $G_0'$ mit:$\\
    S \rA Bb | Da\\
    A \rA aAB | aA | C\\
    B \rA bBD | Bb | C\\
    C \rA c | D | B\\
    D \rA a\\
    $
% A2 b)
\subsubsection\
$
S   \rA BS_1 | W_b W_c\\
S_1 \rA BW_b\\
A   \rA B | AW_a | W_c\\
B   \rA BB_1 | BW_b | C\\
B_1 \rA AB_2\\
B_2 \rA BB_3\\
B_3 \rA AW_a\\
C   \rA W_c | A | B\\
W_a \rA a\\
W_b \rA b\\
W_c \rA c\\
$
% A2 c)
\subsubsection\
CNF4:
$
V' = \{C,B,A\}\\
\gap
S   \rA W_bA | W_b | BW_c | W_c\\
A   \rA B | AW_a | W_a | W_c\\
B   \rA BW_a| W_a | BW_b | W_b | C\\
C   \rA W_c | A | B\\
W_a \rA a\\
W_b \rA b\\
W_c \rA c\\
$
\\
CNF5:\\
$
U = \{
    (A,B), (A,W_a), (A,W_c), (A,W_b), (A,C)\\
    , (B,W_a), (B,W_b), (B,C), (B,W_c), (B,A)\\
    , (C,W_c), (C,A), (C,W_a), (C,B), (C,W_b)
    \}\\
\gap
S   \rA W_bA | b | BW_c | c\\
A   \rA BW_a| a | BW_b | b | c | A | AW_a\\
B   \rA BW_a| a | BW_b | b | c | AW_a\\
W_a \rA a\\
W_b \rA b\\
W_c \rA c\\
$

% A3
\subsection\
% A3 a)
\subsubsection\
gegeben:\\
Grammatik $G$ mit:$\\
    S \rA aSbb | abb\\
    $\\
Sprache $L$ mit:$\\
    L = \{a^nb^m | n,m \in \mathbb{N} \land m \geq 2n\}\\
    $\gap
z.z.: $\forall w \in L(G): w = a^nb^m, n,m \in \mathbb{N}, m \geq 2n)$\\
\\
Wir f"uhren eine Induktion "uber die Wortl"ange $n = |w|$\\
mit der Menge der Wortl"angen von $L(G)$:\[
        N_L = \{3n | n \in \mathbb{N}\}
    \]
Aussage: $
    \forall w \in L(G), |w| \in N_L:
        w = a^nb^m, n,m \in \mathbb{N}, m \geq 2n)\\
    $\\
\\
I.A.$\\
\\
n = 3 \text{, da $abb$ kleinstes Element der Sprache L(G) ist}\\
\nexists w: w \not \equiv abb \land abb \in L(G)
    \text{, da $S \RA abb$ die Einzige Ableitung f"ur W"orter der L"ange 3 ist.}\\
\\
\text{F"ur $w=abb$:}\\
\#_a(w) = 1, \#_b(w) = 2\\
2 = 2*1\\
\text{damit gilt auch } 2 \geq 2*1\\
\text{dadurch wurde Gezeigt, dass die Aussage f"ur n=3 gilt.}\\
$\\
\\
I.V.\\
Die Aussage gelte f"ur $n' \in N_L$ beliebig, aber fest.\\
\\
I.S.$\\
n = n'+3\\
w_n \text{, mit: } |w| = n, w_n \in L(G)\\
w_{n'} \text{, mit: } |w| = n', w_{n'} \in L(G)\\ 
\text{nach der Ableitungsregel von $S$ gilt:}\\
w_n = aw_{n'}bb\\
\\
\text{nach der I.V. gilt:}\\
\#_b(w_{n'}) \geq 2 * \#_a(w_{n'})\\
\\
\text{Aus:}\\
\#_a(w_{n}) = 1 + \#_a(w_{n'})
    \leq 2 * (1+\#_a(w_{n'}))
    = 2 + 2 * \#_a(w_{n'})
    \overset{I.V.}{\leq} 2 + \#_b(w_{n'})
    = \#_b(w_n)\\
\\
\text{Damit wurde die Aussage f"ur beliebige $n \in N_L$ gezeigt.}\\
$\\

% A3 b)
\subsubsection\
Annahme $L \subseteq L(G)$:\\
$
\forall w \in L: w \in L(G)\\
w \mdef abbb\\
w \in L, w \not\in L(G) \lightning\\
$
w ist nicht in $L(G)$, da $L(G)$ keine W"orter der L"ange 4 enth"alt.\\
\\
Damit gilt die Aussage nicht.
\end{document}
