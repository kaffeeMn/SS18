\documentclass{article}
\textwidth=6in
\hoffset=0in
\voffset=0in


\usepackage[a4paper, total={6in, 8in}]{geometry}
\usepackage{amsmath}
\usepackage{amssymb}
\usepackage{stmaryrd}
\usepackage{graphicx}

\usepackage{tikz}
\usetikzlibrary{automata, arrows}
\tikzset{initial text={}}

\usepackage{pifont}
\usepackage{amssymb}
\usepackage{gensymb}
\usepackage{ngerman}
\usepackage[ampersand]{easylist}
\usepackage{xcolor}

% needs to be updated
\author{Max Springenberg, 177792}
\title{\
    GTI "Ubungsblatt 12\\
    Tutor: Marko Schmellenkamp\\
    ID: MS1\\
    "Ubung: Mi 16-18
    }
\setcounter{section}{12}
\date{}

% custom commands
% \Theta \Omega \omega
\newcommand{\tab}{\null\ \qquad}
\newcommand{\gap}{\null\ \\ \\}
\newcommand{\da}{\downarrow}
\newcommand{\la}{\leftarrow}
\newcommand{\lA}{\leftarrow}
\newcommand{\ra}{\rightarrow}
\newcommand{\rA}{\rightarrow}
\newcommand{\LA}{\Leftarrow}
\newcommand{\RA}{\Rightarrow}
\newcommand{\Ra}{\Rightarrow}
\newcommand{\Lra}{\Leftrightarrow}
\newcommand{\ue}{\infty}
\newcommand{\eps}{\epsilon}
\newcommand{\task}[1]{\textbf{#1} \gap}
\newcommand{\cmark}{\ding{51}}
\newcommand{\xmark}{\ding{55}}
\newcommand{\degr}{\null \degree}
\newcommand{\error}[1]{\colorbox{red}{\task{FEHLER:}\\#1}}
\newcommand{\correction}[1]{\colorbox{green}{\task{KORREKTUR:}\\#1}}
\newcommand{\mdef}{\overset{\text{def}}{=}}
\newcommand{\rao}[1]{\overset{#1}{\rightarrow}}
\newcommand{\automaton}[1]{
    \begin{tikzpicture}
    #1
    \end{tikzpicture}
    }
\newcommand{\nd}[4]{
    \node[#1](#2)at(#3){#4};
    }
\newcommand{\dm}{\mathbin{\scriptstyle\dot{\smash{\textstyle-}}}}
\newcommand{\s}{\rhd}
\renewcommand{\u}{\underline}
\renewcommand{\phi}{\varphi}


% content
\begin{document}
% title page
\maketitle
\newpage
% actual paper

% A1
\subsection\

% a)
\subsubsection\
z.z.: $G_{RAPH}I_{SOMORPHIE} \in NP$\\
\\
Nach Vorlesung hatten wir die Zugeh"origkeit von Problemen wie folgt definiert:\\
\begin{enumerate}
    \item es existiert ein Suchraum von L"osungen
    \item jede L"osung ist polynomiell gro"s in ihrer Eingabe
    \item jede L"osung kann in polynomieller Zeit "uberpr"uft werden
\end{enumerate}
\gap
Der Suchraum umfasst Bijektionen von Graph $G$ nach $G'$, oder TUringmaschinen
    die eine Bijektion umsetzten.\\
\\
Diese Funktionen, oder auch Turingmaschinen sind polynomiell in ihrer 
    Eingabegr"o"se, da f"ur jede Kante und jeden Knoten je ein Fall, oder ein
    Zustand ben"otigt wird, der die Jeweiligen variablen umbenennt.\\
Insbesondere ergibt sich daraus, dass Die Funktion oder auch TM durch den Betrag
    an Kanten und Knoten mit $O(|V|+E|)$ eingeschr"ankt werden kann.\\
\\
Der Test eines L"osungskandidaten kann wie folgt getestet werden.\\
$G = (V,E), G' = (V',E')$\\
\begin{enumerate}
    \item $marked = \emptyset$
    \item FOR $v \in V$ DO
    \item \tab $marked = marked \cup \{v\}$
    \item \tab FOR $u \in adj(v) \cap marked$ DO
    \item \tab \tab IF $\neg ( f(u) \in adj(f(v)) )$ THEN
    \item \tab \tab \tab return false
    \item \tab \tab $marked = marked \cup \{u\}$
    \item return tr"u
\end{enumerate}
Der algorithmus testet f"ur jeden Knoten, allen adjazenten Knoten in $adj$ 
    auch nach anwenden der Bijektion noch adjazent sind. Und umgekehrt.\\
Wenn dies der Fall ist wurde die Kantenrelation eingehalten sonst nicht.\\
Die umkehrung der codierung einer bijektiven Funktion, oder einer TM 
    kann polynomiell berechnet werden.\\
Insbesondere ist der Test durch die Anzahl an Kanten begrenzt.\\
Der Algorithmus hat also eine Laufzeitschranke von $O(|E|)$.\\

% b)
\subsubsection\
Eine aussagenlogische Formel $\phi$ bestehe aus $k$ Klauseln und $n$ Literalen,
    mit $k,n \in \mathbb{N}$\\
Da eine Klausel aus Verorderungen besteht ist diese Bereits erf"ullt, sobald
    eine Belegung eine in der Klausel enthaltenem Variable so belegt, dass das
    Literal positiv ist, ist die KLausel erf"ullt.\\
Bereits erf"ullte Klauseln sind nicht mehr ausschlaggeben f"ur die 
    Erf"ullbarkeit von $\phi$.\\
\\
$f$ Sei gegeben durch \[
    f(\phi, \alpha) = \phi'
    \]
Wobei $\phi'$ (i) nur Klauseln aus $\phi$ enth"alt, 
    die unter Belegung $\alpha$ nicht
    schon erf"ullt sind 
    und (ii) deren Klauseln nur noch Variablen 
    enthalten, die nicht im Definitionsbereich von $\alpha$ liegen.\\



\newpage
% A2
\subsection\

% a)
\subsubsection\
Betrachte $I=\{2,3,4\}$ mit:\\
$
\bigcup_{i \in I} S_i = \{a,b,c,d,h\}\\
S_2 \cap S_3  = \emptyset\\
S_2 \cap S_4  = \emptyset\\
S_3 \cap S_4  = \emptyset\\
$\\
$I$ ist eine entsprechende Indexmenge.\\

% b)
\subsubsection\
$
f(M,(S_1,\ldots,S_4)) = 
\gap
(i) (c_{1,e}x_1 + \ldots + c_{k,e}x_k \geq 1):\\
e=a:\\
x_1 + x_4 \geq 1\\
e=b:\\
x_2 + x_2 \geq 1\\
e=c:\\
x_2 \geq 1\\
e=d:\\
x_3 \geq 1\\
e=h:\\
x_3 \geq 1\\
\\
(ii) (x_i + x_j \leq 1, \forall i \neq j : S_i \cap S_j \neq \emptyset):\\
i=1,j=2:\\
x_1 + x_2 \leq 1\\
i=2,j=1:\\
x_2 + x_1 \leq 1\\
i=1,j=4:\\
x_1 + x_4 \leq 1\\
i=4,j=1:\\
x_4 + x_1 \leq 1\\
$

% c)
\subsubsection\
(i) $(c_{1,e}x_1 + \ldots + c_{k,e}x_k \geq 1)$:\\
Es muss f"ur jede Variable eine Menge gew"ahlt werden, in der diese Variable 
    enthalten ist.\\
(ii) $(x_i + x_j \leq 1, \forall i \neq j : S_i \cap S_j \neq \emptyset)$:\\
Es darf von Menge, die zueinander einen nicht leeren Schnitt haben nur eine der
    jeweiligen Mengen genommen werden.\\

% d)
\subsubsection\

% e)
\subsubsection\
(i) $(c_{1,e}x_1 + \ldots + c_{k,e}x_k \geq 1)$:\\
Es gibt $|M|$ Ungleichungen, deren L"ange je durch $O(k)$ eingegrenzt werden
    kann.\\
Damit also in $O(|M| * k)$.\\
(ii) $(x_i + x_j \leq 1, \forall i \neq j : S_i \cap S_j \neq \emptyset)$:\\
Es kann maximal jede Menge zu jeder anderen Menge nicht disjunkt sein.\\
Daraus ergeben sich dann maximal $2*k^2$ Ungleichungen deren L"ange jeweils
    durch $O(1)$ beschr"ankt ist.\\
Damit also in $O(k^2)$\\
\\
Insgesamt ist die Laufzeit der Funktion damit in $O(k^2 + |M| * k)$ und damit
insbesondere polynomiell zur Gr"o"se der Eingaben.\\
\end{document}
