\documentclass{article}
\textwidth=6in
\hoffset=0in
\voffset=0in


\usepackage[a4paper, total={6in, 8in}]{geometry}
\usepackage{amsmath}
\usepackage{amssymb}
\usepackage{stmaryrd}
\usepackage{graphicx}

\usepackage{tikz}
\usetikzlibrary{automata, arrows}
\tikzset{initial text={}}

\usepackage{pifont}
\usepackage{amssymb}
\usepackage{gensymb}
\usepackage{ngerman}
\usepackage[ampersand]{easylist}
\usepackage{xcolor}

% needs to be updated
\author{Max Springenberg, 177792}
\title{\
    GTI "Ubungsblatt 12\\
    Tutor: Marko Schmellenkamp\\
    ID: MS1\\
    "Ubung: Mi 16-18
    }
\setcounter{section}{12}
\date{}

% custom commands
% \Theta \Omega \omega
\newcommand{\tab}{\null\ \qquad}
\newcommand{\gap}{\null\ \\ \\}
\newcommand{\da}{\downarrow}
\newcommand{\la}{\leftarrow}
\newcommand{\lA}{\leftarrow}
\newcommand{\ra}{\rightarrow}
\newcommand{\rA}{\rightarrow}
\newcommand{\LA}{\Leftarrow}
\newcommand{\RA}{\Rightarrow}
\newcommand{\Ra}{\Rightarrow}
\newcommand{\Lra}{\Leftrightarrow}
\newcommand{\ue}{\infty}
\newcommand{\eps}{\epsilon}
\newcommand{\task}[1]{\textbf{#1} \gap}
\newcommand{\cmark}{\ding{51}}
\newcommand{\xmark}{\ding{55}}
\newcommand{\degr}{\null \degree}
\newcommand{\error}[1]{\colorbox{red}{\task{FEHLER:}\\#1}}
\newcommand{\correction}[1]{\colorbox{green}{\task{KORREKTUR:}\\#1}}
\newcommand{\mdef}{\overset{\text{def}}{=}}
\newcommand{\rao}[1]{\overset{#1}{\rightarrow}}
\newcommand{\automaton}[1]{
    \begin{tikzpicture}
    #1
    \end{tikzpicture}
    }
\newcommand{\nd}[4]{
    \node[#1](#2)at(#3){#4};
    }
\newcommand{\dm}{\mathbin{\scriptstyle\dot{\smash{\textstyle-}}}}
\newcommand{\s}{\rhd}
\renewcommand{\u}{\underline}
\renewcommand{\phi}{\varphi}


% content
\begin{document}
% title page
\maketitle
\newpage
% actual paper

% A1
\subsection\

% a)
\subsubsection\
z.z.: $G_{RAPH}I_{SOMORPHIE} \in NP$\\
\\
Nach Vorlesung hatten wir die Zugeh"origkeit von Problemen wie folgt definiert:\\
\begin{enumerate}
    \item es existiert ein Suchraum von L"osungen
    \item jede L"osung ist polynomiell gro"s in ihrer Eingabe
    \item jede L"osung kann in polynomieller Zeit "uberpr"uft werden
\end{enumerate}
\gap
1.\\
Der Suchraum umfasst Bijektionen von Graph $G$ nach $G'$, oder Turingmaschinen
    die eine Bijektion umsetzten.\\
\\
2.\\
Diese Funktionen, oder auch Turingmaschinen sind polynomiell in ihrer 
    Eingabegr"o"se, da f"ur jede Kante und jeden Knoten je ein Fall, oder ein
    Zustand ben"otigt wird, der die Jeweiligen variablen umbenennt.\\
Insbesondere ergibt sich daraus, dass Die Funktion oder auch TM durch den Betrag
    an Kanten und Knoten mit $O(|V|+E|)$ eingeschr"ankt werden kann.\\
\\
3.\\
Der Test eines L"osungskandidaten kann wie folgt getestet werden.\\
$G = (V,E), G' = (V',E')$\\
\begin{enumerate}
    \item $marked = \emptyset$
    \item FOR $v \in V$ DO
    \item \tab $marked = marked \cup \{v\}$
    \item \tab FOR $u \in adj(v) \cap marked$ DO
    \item \tab \tab IF $\neg ( f(u) \in adj(f(v)) )$ THEN
    \item \tab \tab \tab return false
    \item \tab \tab $marked = marked \cup \{u\}$
    \item return true
\end{enumerate}
Der algorithmus testet f"ur jeden Knoten, allen adjazenten Knoten in $adj$ 
    auch nach anwenden der Bijektion noch adjazent sind. Sowie implizit umgekehrt.\\
Wenn dies der Fall ist wurde die Kantenrelation eingehalten, sonst nicht.\\
Insbesondere ist der Test durch die Anzahl an Kanten begrenzt da nur unmarkierte
    Kanten durchlaufen werden.\\
Der Algorithmus hat also eine Laufzeitschranke von $O(|E|)$.\\

% b)
\subsubsection\
Eine aussagenlogische Formel $\phi$ bestehe aus $k$ Klauseln und $n$ Literalen,
    mit $k,n \in \mathbb{N}$\\
Da eine Klausel aus Veroderungen besteht ist diese Bereits erf"ullt, sobald
    eine Belegung eine in der Klausel enthaltenem Variable so belegt, dass das
    Literal positiv ist, ist die KLausel erf"ullt.\\
Bereits erf"ullte Klauseln sind nicht mehr ausschlaggeben f"ur die 
    Erf"ullbarkeit von $\phi$.\\
\\
$f$ Sei gegeben durch \[
    f(\phi, \alpha) = \phi'
    \]
Wobei $\phi'$ (i) nur Klauseln aus $\phi$ enth"alt, 
    die unter Belegung $\alpha$ nicht
    schon erf"ullt sind 
    und (ii) deren Klauseln nur noch Variablen 
    enthalten, die nicht im Definitionsbereich von $\alpha$ liegen.
    Und (iii) $\phi'$ abschlie"send mit einer 1 verundet wird.\\
\\
Zu zeigen bleibt:\\
(1) f ist total\\
(2) f ist polynomiell berechenbar\\
(3) f erf"ullt $F_{INISH}S_{AT} \leq S_{AT}$\\
\gap
(1) Es wurden keine Definitionsl"ucken definiert also, ist f total.\\
\\
(2) f ist polynomiell berechenbar, da:\\
(i)\\
F"ur jede Klausel in polynomiell getested werden kann ob eines der belegten
    Literale wahr ist.\\
(ii)\\
Literale, die Variablen aus einer Menge enthalten in polynomieller zeit 
    rausgefiltert werden k"onnen.\\
(iii)\\
Eine 1 in polynomieller Zeit verundet werden kann.\\
\\
(3)\\
$(\phi, \alpha) \in F_{INISH}S_{AT} \Ra \phi' \in S_{AT}$\\
Wenn es eine erweiterte Belgeung $\beta$ zu $\alpha$ gibt, die $\phi$ erf"ullt,
    dann:\\
\\
1. War $\phi$ schon durch $\alpha$ erf"ullt:\\
    So wurden alle Klauseln aus $\phi$ f"ur $\phi'$ getstirchen und nur noch
    1 bleibt "ubrig, $\phi'$ hat keine Variablen und ist immer erf"ullt.\\
\\
2. Sonst:\\
    Dann bleibt noch mindestens eine Klausel und eine Variable "ubrig. Da alle 
    bereits belegten Variablen entfernt wurden sind nur noch die Variablen zu
    belegen, die unter der Erweiterung von $\beta$ von $\alpha$ dazu kommen.
    sei nun $\gamma$ die Belegung, die wie $\beta$ auf allen Variablen die 
    in $\beta$ aber nicht in $\alpha$ definiert sind, so ist $\gamma$ auf
    $\phi'$ total und ferner eine $\phi'$ erf"ullende Belegung.\\
\\
$\phi' \in S_{AT} \Ra (\phi, \alpha) \in F_{INISH}S_{AT}$\\
Wenn f"ur $\phi'$ eine erf"ullende Belgung existiert, so ist:\\
\\
1. $\phi'$ frei von variablen und enth"alt nur 1:\\
    Dann hatte bereits $\alpha$ f"ur jede Klausel mindestens ein Literal 
    erf"ullt und alle erweiterungen von $\alpha$ auf eine totale Belegung
    w"urden $\phi$ erf"ullen.\\
\\
2. Sonst:\\
    Da nur noch Variablen, die nicht in $\alpha$ vorkommen betrachtet werden
    und eine erf"ullende Belegung $\gamma$ existiert, 
    muss $\alpha$ unter der Erweiterung
    der Belegungen aus $\gamma$ auch $\phi$ erf"ullen.\\
\\
Damit ist $f$ eine polynomielle Reduktion von $F_{INISH}S_{AT}$ auf $S_{AT}$.\\


\newpage
% A2
\subsection\

% a)
\subsubsection\
Betrachte $I=\{2,3,4\}$ mit:\\
$
\bigcup_{i \in I} S_i = \{a,b,c,d,h\} = M\\
S_2 \cap S_3  = \emptyset\\
S_2 \cap S_4  = \emptyset\\
S_3 \cap S_4  = \emptyset\\
$\\
$I$ ist eine entsprechende Indexmenge.\\

% b)
\subsubsection\
$
f(M,(S_1,\ldots,S_4)) = 
\gap
(i) (c_{1,e}x_1 + \ldots + c_{k,e}x_k \geq 1):\\
e=a:\\
x_1 + x_4 \geq 1\\
e=b:\\
x_2 + x_2 \geq 1\\
e=c:\\
x_2 \geq 1\\
e=d:\\
x_3 \geq 1\\
e=h:\\
x_3 \geq 1\\
\\
(ii) (x_i + x_j \leq 1, \forall i \neq j : S_i \cap S_j \neq \emptyset):\\
i=1,j=2:\\
x_1 + x_2 \leq 1\\
i=2,j=1:\\
x_2 + x_1 \leq 1\\
i=1,j=4:\\
x_1 + x_4 \leq 1\\
i=4,j=1:\\
x_4 + x_1 \leq 1\\
$

% c)
\subsubsection\
(i) $(c_{1,e}x_1 + \ldots + c_{k,e}x_k \geq 1)$:\\
Es muss f"ur jede Variable eine Menge gew"ahlt werden, in der diese Variable 
    enthalten ist.\\
(ii) $(x_i + x_j \leq 1, \forall i \neq j : S_i \cap S_j \neq \emptyset)$:\\
Es darf von Menge, die zueinander einen nicht leeren Schnitt haben nur eine der
    jeweiligen Mengen genommen werden.\\

% d)
\subsubsection\
z.z.:\\
$(M,S_1,\ldots,S_k) \in E_{XACT}C_{OVER} \Lra f(M,S_1,\ldots,S_k) \in 0-1-ILP$\\
\\
$(M,S_1,\ldots,S_k) \in E_{XACT}C_{OVER} \Ra f(M,S_1,\ldots,S_k) \in 0-1-ILP$\\
Ist $(M,S_1,\ldots,S_k) \in E_{XACT}C_{OVER}$, so existiert eine Indexmenge $I$
mit:\\ 
$\bigcup_{i \in I} S_i = M \land \forall i \neq j: S_i \cap S_j = \emptyset$\\
\\
F"ur die durch $f$ berechneten Ungleichungen bedeutet das, dass:\\
\\
(i) alle Ungleichungen der Form $(c_{1,e}x_1 + \ldots + c_{k,e}x_k \geq 1)$ 
    erf"ullt werden k"onnen. Da $\bigcup_{i \in I} S_i = M$ gilt. Also jede Variable
    in mindestens einer Menge vorkommt.\\
\\
(ii) alle Ungleichungen der Form 
    $(x_i + x_j \leq 1, \forall i \neq j : S_i \cap S_j \neq \emptyset)$ 
    ebenfalls erf"ullt werden, da 
    $\forall i \neq j: S_i \cap S_j = \emptyset$ 
    gilt. Also von den Variablen aus den Mengen, die gew"ahlt wurden keine in
    mehr als eine der Mengen vorkommt.\\
Damit sind dann auch alle Ungleichungen erf"ullt.\\
\\
$f(M,S_1,\ldots,S_k) \in 0-1-ILP \Ra (M,S_1,\ldots,S_k) \in E_{XACT}C_{OVER}$\\
Ist $f(M,S_1,\ldots,S_k) \in 0-1-ILP$, so gilt, dass alle der Unglichungen:\\
(i) $(c_{1,e}x_1 + \ldots + c_{k,e}x_k \geq 1)$\\
(ii) $(x_i + x_j \leq 1, \forall i \neq j : S_i \cap S_j \neq \emptyset)$\\
erf"ullbar sind.\\
\\
Das bedeutet im R"uckschluss f"ur $(M,S_1,\ldots,S_k)$:\\
(1) Jedes Element aus $M$ ist in einer der Teilmengen $S_i, i \in I$ enthalten,
    da (i) gilt.\\
(2) Jede der Teilmengen $S_i \in I$ zu jeder anderen Teilmenge Disjunkt ist,
    da (ii) gilt.\\
Damit ist 
    $\bigcup_{i \in I} S_i = M \land \forall i \neq j: S_i \cap S_j = \emptyset$
    erf"ullt.\\
\\
Ferner ist damit die Reduktionseigenschaft erf"ullt.\\


% e)
\subsubsection\
(i) $(c_{1,e}x_1 + \ldots + c_{k,e}x_k \geq 1)$:\\
Es gibt $|M|$ Ungleichungen, deren L"ange je durch $O(k)$ eingegrenzt werden
    kann.\\
Damit also in $O(|M| * k)$.\\
(ii) $(x_i + x_j \leq 1, \forall i \neq j : S_i \cap S_j \neq \emptyset)$:\\
Es kann maximal jede Menge zu jeder anderen Menge nicht disjunkt sein.\\
Daraus ergeben sich dann maximal $2*k^2$ Ungleichungen deren L"ange jeweils
    durch $O(1)$ beschr"ankt ist.\\
Damit also in $O(k^2)$\\
\\
Insgesamt ist die Laufzeit der Funktion damit in $O(k^2 + |M| * k)$ und damit
insbesondere polynomiell zur Gr"o"se der Eingaben.\\
\end{document}
