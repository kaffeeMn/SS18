\documentclass{article}
\textwidth=6in
\hoffset=0in
\voffset=0in


\usepackage[a4paper, total={6in, 8in}]{geometry}
\usepackage{amsmath}
\usepackage{amssymb}
\usepackage{stmaryrd}
\usepackage{graphicx}

\usepackage{tikz}
\usetikzlibrary{automata, arrows}
\tikzset{initial text={}}

\usepackage{pifont}
\usepackage{amssymb}
\usepackage{gensymb}
\usepackage{ngerman}
\usepackage[ampersand]{easylist}

% needs to be updated
\author{Max Springenberg, 177792}
\title{\
    GTI "Ubungsblatt 10\\
    Tutor: Marko Schmellenkamp\\
    ID: MS1\\
    "Ubung: Mi 16-18
    }
\setcounter{section}{10}
\date{}

% custom commands
% \Theta \Omega \omega
\newcommand{\tab}{\null\ \qquad}
\newcommand{\gap}{\null\ \\ \\}
\newcommand{\da}{\downarrow}
\newcommand{\la}{\leftarrow}
\newcommand{\lA}{\leftarrow}
\newcommand{\ra}{\rightarrow}
\newcommand{\rA}{\rightarrow}
\newcommand{\LA}{\Leftarrow}
\newcommand{\RA}{\Rightarrow}
\newcommand{\Ra}{\Rightarrow}
\newcommand{\Lra}{\Leftrightarrow}
\newcommand{\ue}{\infty}
\newcommand{\eps}{\epsilon}
\newcommand{\task}[1]{\textbf{#1} \gap}
\newcommand{\cmark}{\ding{51}}
\newcommand{\xmark}{\ding{55}}
\newcommand{\degr}{\null \degree}
\newcommand{\error}{\task{FEHLER:}}
\newcommand{\correction}{\task{KORREKTUR:}}
\newcommand{\mdef}{\overset{\text{def}}{=}}
\newcommand{\rao}[1]{\overset{#1}{\rightarrow}}
\newcommand{\automaton}[1]{
    \begin{tikzpicture}
    #1
    \end{tikzpicture}
    }
\newcommand{\nd}[4]{
    \node[#1](#2)at(#3){#4};
    }
\newcommand{\dm}{\mathbin{\scriptstyle\dot{\smash{\textstyle-}}}}
\newcommand{\s}{\rhd}
\renewcommand{\u}{\underline}


% content
\begin{document}
% title page
\maketitle
\newpage
% actual paper

% A1
\subsection\
\subsubsection\
Siehe ausgedrucktes Blatt vom Aufgabenzettel.\\

\subsubsection\

% A2
\subsection\
% a)
\subsubsection\
$L_a = L_1 \cup L_2$\\
\\
Wir wissen, dass $L_n$ von einer TM $M_n, n \in \{1,2\}$ entschieden wird.\\
Ferner muss unsere TM nach Aufgabenstellung nicht f"ur alle Eingaben terminieren,
    wenn diese nicht in der Sprache sind.\\
Damit erg"abe sich $M_a$ aus den Turingmaschinen $M_1, M_2$ wie folgt:\\
\begin{enumerate}
    \item Zust"ande aus $M_1, M_2$ werden so umgennant, dass sie nicht 
            konkurierent sind
    \item Der Starzustand aus $M_a$ sei der Startzustand aus $M_1$
    \item $M_a$ simuliert zun"achst $M_1$, bis entweder akzeptiert wird, oder
            nicht. wenn nicht wird der String wieder auf die initiale eingabe
            gesetzt und in den Startzustand von $M_2$ gewechselt.\\
\end{enumerate}

% b)
\subsubsection\
$L_b = L_1 \odot L_2$\\
\\
Wir wissen, dass $L_n$ von einer TM $M_n, n \in \{1,2\}$ entschieden wird.\\
Wir k"onnen die TM $M_b$ wie folgt konstruieren.\\
\begin{enumerate}
    \item Zust"ande aus $M_1, M_2$ werden so umgennant, dass sie nicht 
            konkurierent sind
    \item Es wird mit der TM $M_1$ begeonnen.
            Zun"achst wird getestet, ob die ganze Eingabe in $L_1$ ist.
            Wenn nicht wird das letzte Symbol $\sigma \in \Sigma$ der Eingabe
            markiert (durch z.B.$\u{\sigma}$) und erneut getest ob der 
            Teilstring aller nicht markierten Zeichen in $L_1$ ist, solange
            bis der erste Teilstring aus $L_1$ gefunden wurde.
    \item Nun wird das letzte Zeichen des Teilstring aus $L_1$ markiert und 
            alle zovor markierten zeichen demarkiert. 
            $M_2$ wird ab dem markiertem Zeichen simuliert.
    \item Wenn der Verbliebene String nicht in $L_2$ ist werden alle Zeichen 
            rechts vom markierten letzten Symbol, dass noch zu dem Teilstring
            aus $L_1$ markiert und auf allen unmarkierten Zeichen wieder, wie
            zuvor durch sequentielles suchen und markieren des letzten Zeichens
            vestgestellt, welches der n"achst l"angste Teilstring aus $L_1$
            ist.
    \item weiter bei 3
\end{enumerate}

% A3
\subsection\
% a)
\task{\begin{tabular}{ll}
        Problem:    &A\\
        Gegeben:    &TM $M$, $k \in \mathbb{N}_0$\\
        Frage:      &Erzeugt $M$ bei Eingabe $0^k$ die Ausgabe 1?\\
    \end{tabular}}
\\
Nach der Vorlesung existiert keine TM $M_{hw}$, die entscheidet, ob eine TM bei
    einer Eingabe $I$ $ `hello world' $ ausgibt.\\
Der Beweis, dass es keine solche Turingmaschine $M_1$ f"ur das Problem A gibt
    kann analog gezeigt werden.\\
\\
Da es keine M"oglichkeit gibt mit einer Turingmaschine zu testen ob eine andere
    TM $M$ bei Eingabe $0^k$ 1 ausgibt, kann es auch keine Turingmaschine $M'$
    mit $L(M') = \{k | \text{$M$ gibt bei Eingabe $0^k$ 1 aus}\}$ geben.\\
Ferner ist A damit auch nicht semientscheidbar.\\

% b)
\subsubsection\
\task{\begin{tabular}{ll}
        Problem:    &B\\
        Gegeben:    &TM $M$\\
        Frage:      &Erzeugt $M$ bei keiner Eingabe die Ausgabe 1?\\
    \end{tabular}}
\\
F"ur das komplement"arproblem, ob eine eingabe existiert, f"ur die $M$ 1 ausgibt
    ist semientscheidbar.\\
Ferner ist damit B unentscheidbar.\\

% A4
\subsection\
Konstante funktionen sind primitiv rekursiv.\\
$g(x) = 1, h(x) = 0$ sind solche konstanten Funktionen.\\
\\
$even(x)$ l"asst sich nun auch als:\\
$
even(x) =   \begin{cases}
                g(x)& \text{, x ist gerade}\\
                h(x)& \text{, x ist ungerade}\\
            \end{cases}\\
$\\
primitive Rekursionen von primitiv rekursiven Funktionen sind auch primitiv
    rekursiv.\\
Wir stellen fest $even$ kann auch wie folgt definiert werden:\\
$
even(0) = g(x)\\
even(1) = h(x)\\
even(x+2) = even(x)\\
$
da $g,h$ primitiv rekursiv ist $even$ nach der Definition von primitv rekursiven
    Funktionen aus der Vorlesung auch primitiv rekursiv.\\
\end{document}
