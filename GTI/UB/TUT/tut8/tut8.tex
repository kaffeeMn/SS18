\documentclass{article}
\textwidth=6in
\hoffset=0in
\voffset=0in


\usepackage[a4paper, total={6in, 8in}]{geometry}
\usepackage{amsmath}
\usepackage{amssymb}
\usepackage{stmaryrd}
\usepackage{graphicx}

\usepackage{tikz}
\usetikzlibrary{automata, arrows}
\tikzset{initial text={}}

\usepackage{pifont}
\usepackage{amssymb}
\usepackage{gensymb}
\usepackage{ngerman}
\usepackage[ampersand]{easylist}

% needs to be updated
\author{Max Springenberg, 177792}
\title{\
    Tutorium 8\\
    }
\setcounter{section}{8}
\date{}

% custom commands
% \Theta \Omega \omega
\newcommand{\tab}{\null\ \qquad}
\newcommand{\gap}{\null\ \\ \\}
\newcommand{\lA}{\leftarrow}
\newcommand{\rA}{\rightarrow}
\newcommand{\LA}{\Leftarrow}
\newcommand{\RA}{\Rightarrow}
\newcommand{\ue}{\infty}
\newcommand{\eps}{\epsilon}
\newcommand{\task}[1]{\textbf{#1} \gap}
\newcommand{\cmark}{\ding{51}}
\newcommand{\xmark}{\ding{55}}
\newcommand{\degr}{\null \degree}
\newcommand{\error}{\task{FEHLER:}}
\newcommand{\correction}{\task{KORREKTUR:}}
\newcommand{\mdef}{\overset{\text{def}}{=}}
\newcommand{\rao}[1]{\overset{#1}{\rightarrow}}
\newcommand{\automaton}[1]{
    \begin{tikzpicture}
    #1
    \end{tikzpicture}
    }
\newcommand{\nd}[4]{
    \node[#1](#2)at(#3){#4};
    }
\makeatletter
\newcommand{\SF}[1]{
    #1&\rA&\checkNextRule
    }
\newcommand{\checkNextRule}[1]{&#1 \@ifnextchar\bgroup{\nextRule}{\\}}
\newcommand{\nextRule}[1]{&| #1 \@ifnextchar\bgroup{\nextRule}{\\}}
\makeatother


% content
\begin{document}
% title page
\maketitle
\newpage
% actual paper

% A1
\subsection\
$G$ mit:$\\
\begin{array}{llllllllllll}
    \SF{S}{aBb}{bDbb}{\eps}
    \SF{A}{AEF}{\eps}
    \SF{B}{AA}{a}
    \SF{C}{aFb}
    \SF{D}{B}
    \SF{E}{B}{D}{ab}{\eps}
    \SF{F}{Fa}{FE}
\end{array}\\
$
\gap
\task{CNF1}
(i) Entfernen der Nicht erzeugenden Variablen\\
$V_e = \{S,A,B,D,E\}$\\
Daraus folgt $G'$:\\
$
\begin{array}{llllllllllll}
    \SF{S}{aBb}{bDbb}{\eps}
    \SF{A}{\eps}
    \SF{B}{AA}{a}
    \SF{D}{B}
    \SF{E}{B}{D}{ab}{\eps}
\end{array}\\
$
(ii) Erreichbarkeitsgraph\\
\\
\automaton{
    \nd{state}{S}{0,0}{S}
    \nd{state}{A}{0,2}{A}
    \nd{state}{B}{2,0}{B}
    \nd{state}{D}{2,-2}{D}
    \nd{state}{E}{0,-2}{E}
    \path
        (S)
            edge [->] node {}(A)
            edge [->] node {}(B)
            edge [->] node {}(D)
        (B)
            edge [->] node {}(A)
        (D)
            edge [->] node {}(B)
        (E)
            edge [->] node {}(B)
            edge [->] node {}(D)
    ;
    }
erreichbar nur S,A,D,B\\
Daraus folgt $G_1$:$\\
\begin{array}{llllllllllll}
    \SF{S}{aBb}{bDbb}{\eps}
    \SF{A}{\eps}
    \SF{B}{AA}{a}
    \SF{D}{B}
\end{array}\\
$
\gap
\task{CNF2}
$
\begin{array}{llllllllllll}
    \SF{S}{W_aBb}{W_bDW_bW_b}{\eps}
    \SF{A}{\eps}
    \SF{B}{AA}{W_a}
    \SF{D}{B}
\end{array}\\
$
\gap
\task{CNF3}
$
\begin{array}{llllllllllll}
    \SF{S}{W_aBb}{W_bS_1}{\eps}
    \SF{S_1}{DS_2}
    \SF{S_2}{W_bW_b}
    \SF{A}{\eps}
    \SF{B}{AA}{W_a}
    \SF{D}{B}
\end{array}\\
$
\gap
\task{CNF4}
$
V' = \{S,A,B,D\}\\
\begin{array}{llllllllllll}
    \SF{S}{W_aBb}{W_bS_1}
    \SF{S_1}{DS_2}
    \SF{S_2}{W_bW_b}
    \SF{B}{W_a}
    \SF{D}{B}
\end{array}\\
$
\gap
\task{CNF5}
$
U = \{(B,W_a),(D,B)\}\\
\begin{array}{llllllllllll}
    \SF{S}{W_aBb}{W_bS_1}
    \SF{S_1}{DS_2}
    \SF{S_2}{W_bW_b}
    \SF{D}{W_a}
\end{array}\\
$
\gap
\task{CNF6}
$
\begin{array}{llllllllllll}
    \SF{S'}{\eps}{S}
    \SF{S}{W_aBb}{W_bS_1}
    \SF{S_1}{DS_2}
    \SF{S_2}{W_bW_b}
    \SF{D}{W_a}
\end{array}\\
$
% A2
\subsection\
CNF analog zu 1.

% A3
\subsection\
gegeben:\\
$G$ mit:$
\begin{array}{lllllll}
    \SF{S}{aSa}{bSa}{bSb}{\eps}
\end{array}\\
\\
L = \{w \in \{a,b\}^* | |w| \text{gerade}\}\\
$\\
\subsubsection{$L(G) \subseteq L$}
Aussage:\\
$k, w$, mit $S \RA^k w : w \in L$\\
\\
Induktion "uber Abgleitungsl"ange $k$.\\
\\
I.A.\\
$k = 1$:\\
Es gibt genau eine Ableitung der L"ange $k=1$, mit\[
    S \RA^1 \eps
    \]
Es gilt $|\eps| = 0$, ferner ist 0 gerade und damit die Aussage f"ur $k=1$
    erf"ullt.\\
\\
I.V.\\
Die Aussage gelte f"ur $k \in \mathbb{N}$ beliebig, aber fest\\
\\
I.S.\\
$k = k'+1$\\
F"ur Ableitungen der L"ange $k \geq 1$ m"ussen folgende Regeln betrachtet werden:\\
$
    (i) S \rA aSa\\
    (ii) S \rA aSb\\
    (iii) S \rA bSa\\
    (iii) S \rA bSb\\
$\gap
(i)\\
$S \rA aSa $, ergibt sich zu $w \mdef ava$, mit $ S \RA^{k'} v$, ferner gilt 
    aus I.V., dass $|v|$ gerade ist und damit dann $|w| = |ava| = |v| + 2$ 
    auch.\\
\\
(ii) - (iv) analog.\\

\subsubsection{$L \subseteq L(G)$}
Aussage:\\
$i$, mit $|w| = i, w \in L : w \in L(G)$\\
\\
Induktion "uber Wortl"ange i.\\
Dabei muss die Wortl"ange nach definition aus 
    $N_{|w|=2n} = \{2n | n \in \mathbb{N}_0\}$ sein.\\
Ferner inkrementiert die Wortl"ange damit in der Induktion um 2.\\
\\
I.A.\\
$i = 0$\\
Nur das leere Wort $\eps$ hat die Wortl"ange 0.\\
$S \RA \eps$ ist aus $G$ ableitbar, damit ist $\eps \in G$ und die Aussage f"ur
    $i=0$ erf"ullt.\\
\\
I.V.\\
Die Aussage gelte f"ur $k \in N_{|w|=2n}$ beliebig, aber fest.\\
\\
I.S.\\
$i = i'+2$\\
$\Sigma_L$ sei das Alphabet der Sprache $L$.\\
Wir betrachten das Wort $w \mdef \sigma v \sigma'$, mit 
    $|w| = i, |v| = i', (i,i' \in N_{|w|=2n}), (\sigma, \sigma' \in \Sigma_L)$.\\
\\
Nach I.V. gilt, dass $v$ aus $G$ ableitbar ist.\\
\\
Nun bleibt zu zeigen, dass:$\\
\\
(i) w_{(i)} \mdef ava\\
(ii) w_{(ii)} \mdef avb\\
(iii) w_{(iii)} \mdef bva\\
(iv) w_{(iv)} \mdef bvb\\
$\\
aus $G$ abgeleitet werden k"onnen.\\
\gap
(i)\\
nach I.V. gilt $S \RA^* v$, dann gilt auch $S \RA^* ava$, aufgrund der Regel
    $S \rA aSa$\\
\\
(ii) - (iv) analog.\\

% A4
\subsection\
$G$ (mindestens ein $b$) mit:\\
$
\begin{array}{llllllllllllll}
    \SF{S}{TS}{b}
    \SF{T}{ST}{a}
\end{array}\\
$\\
$G_{Greibach}$ mit:\\
$
\begin{array}{llllllllllllll}
    \SF{S}{aS}{bS'}{b}
    \SF{S'}{aS'}{bS'}{a}{b}
\end{array}\\
$
\end{document}
