\documentclass{article}
\textwidth=6in
\hoffset=0in
\voffset=0in


\usepackage[a4paper, total={6in, 8in}]{geometry}
\usepackage{amsmath}
\usepackage{amssymb}
\usepackage{stmaryrd}
\usepackage{graphicx}

\usepackage{tikz}
\usetikzlibrary{automata, arrows}
\tikzset{initial text={}}

\usepackage{pifont}
\usepackage{amssymb}
\usepackage{gensymb}
\usepackage{ngerman}
\usepackage[ampersand]{easylist}

% needs to be updated
\author{Max Springenberg, 177792}
\title{\
    GTI "Ubungsblatt 6\\
    Tutor: Marko Schmellenkamp\\
    ID: MS1\\
    "Ubung: Mi 16-18
    }
\setcounter{section}{6}
\date{}

% custom commands
% \Theta \Omega \omega
\newcommand{\tab}{\null\ \qquad}
\newcommand{\gap}{\null\ \\ \\}
\newcommand{\lA}{\leftarrow}
\newcommand{\ra}{\rightarrow}
\newcommand{\rA}{\rightarrow}
\newcommand{\LA}{\Leftarrow}
\newcommand{\RA}{\Rightarrow}
\newcommand{\ue}{\infty}
\newcommand{\eps}{\epsilon}
\newcommand{\task}[1]{\textbf{#1} \\ \gap}
\newcommand{\cmark}{\ding{51}}
\newcommand{\xmark}{\ding{55}}
\newcommand{\degr}{\null \degree}
\newcommand{\error}{\task{FEHLER:}}
\newcommand{\correction}{\task{KORREKTUR:}}
\newcommand{\mdef}{\overset{\text{def}}{=}}
\newcommand{\rao}[1]{\overset{#1}{\rightarrow}}
\newcommand{\automaton}[1]{
    \begin{tikzpicture}
    #1
    \end{tikzpicture}
    }
\newcommand{\nd}[4]{
    \node[#1](#2)at(#3){#4};
    }


% content
\begin{document}
% title page
\maketitle
\newpage
% actual paper

% A1
\subsection\
Mit $p, q \in Q, \sigma \in \Sigma, \tau \in \Gamma, w\in \Gamma^*$:\\
In deterministischen Kellerautomaten darf es 
    für jede Kombination von $p, \sigma, \tau$ 
    nur eine Transition $(p, \sigma, \tau, q, w)$ in $\delta$ 
    geben.\\
Bei einem deterministischen Kellerautomaten, der mit leerem Keller akzeptiert
    gibt es keine akzeptierenden Zust"ande und wenn f"ur $p, \sigma, \tau$ kein
    $(p, \sigma, \tau, q, w)$ in $\delta$ existiert wird abgelehnt.\\
\gap
Eine Idee f"ur einen Kellerautomaten, der die Sprache entscheidet ist:\\
    Jedes $\sigma \in \{a,b\}$, das eingelesen wird eben
    dieses $\sigma$ auf den Keller zu legen. 
    Wenn dann ein $c$ eingelesen wird,
    wird abgeglichen, ob genau so viele $c$ hintereinander eingelesen werden, 
    wie zuvor $a$ auf den Keller gelegt wurden. 
    Wenn statt dem $c$ ein $d$ eingelesen wird, 
    wird abgeglichen, ob genau so viele $d$ hintereinander eingelesen werden, 
    wie zuvor $b$ auf den Keller gelegt wurden.\\
\\
Ein Kellerautomat, der Diese Idee umsetzt, deterministisch ist und mit leerem
    Keller akzeptiert ist:\\
\automaton{
    \nd{initial, state}{s}{0,0}{s}
    \nd{state}{ca}{4,2}{$q_{c,a}$}
    \nd{state}{cb}{0,2}{$q_{c,b}$}
    \nd{state}{da}{0,-2}{$q_{d,a}$}
    \nd{state}{db}{4,-2}{$q_{d,b}$}

    \path
        (s)
        edge [->, loop right] node {$\begin{array}{l}
                                        a,*:a*\\
                                        b,*:b*\\
                                    \end{array}\\$} (s)
        edge [->,left] node {$c,a:\eps$} (ca)
        edge [->,left] node {$c,b:b$} (cb)
        edge [->,left] node {$d,a:a$} (da)
        edge [->,left] node {$d,b:\eps$} (db)

        (ca)
        edge [->, loop right] node {$\begin{array}{l}
                                        c,a:\eps\\
                                        \eps,\#:\eps\\
                                        \eps,b:\eps\\
                                    \end{array}$} (ca)

        (cb)
        edge [->, loop above] node {$\eps,b:\eps$}(cb)
        edge [->, above] node {$\eps,a:\eps$} (ca)

        (db)
        edge [->, loop right] node {$\begin{array}{l}
                                        d,b:\eps\\
                                        \eps,\#:\eps\\
                                        \eps,a:\eps\\
                                    \end{array}$} (db)

        (da)
        edge [->, loop below] node {$\eps,a:\eps$}(da)
        edge [->, below] node {$\eps,b:\eps$} (db)
    ;}\\
\\
Im Startzustand werden $a,b$ auf den Keller gelegt, nachdem $d$ oder $c$ gelesen
    werden, wird in Zust"ande gewechselt, die den Keller leeren k"onnen.\\
Dabei muss vor dem Erreichen der Zust"ande mit $\eps$-Transitionen die 
    Kellersymbole $a,b$ l"oschen sichergestellt werden, dass zuvor auch ein $c$
    f"ur das l"oschen aller $b$'s und ein $d$ f"ur das l"oschen aller $a$'s
    eingelesen wurde, dies wird durch die Transitionen von $s$ zu den 
    Zust"anden $q_{c,\sigma}, q_{d,\sigma}, \sigma \in \{a,b\}$ umgesetzt.\\
Da gegebenen falls beim einlesen des ersten $c$ oder $d$ ein unerw"unschtes
    Kellersymbol, das nicht durch einlesen von $c$ oder $d$ gel"oscht werden 
    kann, muss je ein 
    Zwischenzustand gegeben sein, der die unerw"unschten und das erste
    gew"unschte Kellersymbol l"oscht. Dies wird durch die Zust"ande 
    $q_{c,a}, q_{c,b}$ und deren Transitionen f"ur $c$,
    sowie $q_{d,a}, q_{d,b}$ und deren Transition f"ur $d$ umgesetzt.\\

% A2
\subsection\

% A2 a)
\subsubsection\
Gegeben ist eine ausgewertetes Tableau des CYK-Algorithmus f"ur das 
    Wort $baabc$.\\
Die W"orter 

    (i) $w_1 = baabc$,\\
    
    (ii) $w_2 = baab$,\\
    
    (iii) $w_3 = aabc$\\

enth"alt $baabc$ als Teilw"orter an den Indices 1 bis 5 f"ur (i), 1 bis 4
    f"ur (ii) und 2 bis 5 f"ur (iii).\\
Dies sind die einzigen Vorkommnisse der Teilw"orter in $baabc$, da ihr ihre
    Position unter anderem durch die zwei aufeinander folgenden $a$ festgelegt
    ist.\\
\\
Im Tableau f"ur ein Wort $w$ enthalten die Felder (i,j), mit 
    $i, j \in \mathbb{N}_0, i \leq |w|, j \leq |w|$ die Variablen, die
    das Teilwort von w, das am Index i beginnt und j aufh"ort abgeleiten 
    k"onnen.\\ 
\\
Aus der Tabelle l"asst sich f"ur die jeweiligen Indices der Teilw"orter ablesen,
    dass:\\

    (i) die Variable $S$ nicht im Feld (1,5) des Tableau eingetragen ist und 
        ferner das  Wort nicht in der Sprache ist, da es nicht aus der 
        Startvariablen abgeleitet werden kann.\\

    (ii) die Variable $S$ im Feld (1,4) des Tableau eingetragen ist und ferner
        das Wort in der Sprache ist, da es aus der Startvariablen abgeleitet
        werden kann.\\

    (iii) die Variable $S$ im Feld (2,5) des Tableau eingetragen ist und ferner
        das Wort in der Sprache ist, da es aus der Startvariablen abgeleitet
        werden kann.\\
% A3
\subsection\

\subsubsection\

F"ur LL(1) Grammtiken muss gelten:\\
$
\forall X \in V, X \ra \alpha, X \ra \beta, \alpha \neq \beta:\\
(i) FIRST(\alpha) \cap FIRST(\beta) = \emptyset\\
(ii) \alpha \RA^* \eps \text{, dann} FOLLOW(X) \cap FIRST(\beta) = \emptyset\\
$\\
\gap

\task{$G_1$}
Die $FIRST$- Mengen ergeben sich wie folgt:\\
\gap
$
FIRST(S) = \{a,e,f\}\\
FIRST(A) = \{e,d,b\}\\
FIRST(B) = \{\eps, b\}\\
FIRST(C) = \{\eps, d\}\\
$
Da kein element mehrfach entdeckt wurde gilt (i)\\
\\
Die FOLLOW Mengane ist nur f"ur $\alpha \RA^* \eps$ interessant, ferner ist das
    in $G_1$ $A,B,C$\\
\\
betrachte:\\
$
A \RA ea | CB\\
CB \RA^* \eps\\
\\
e \in FIRST(ea), e \in FIRST(S)\\
S \RA eAS \text{, damit} e \in FOLLOW(A)\\
\\
FIRST(ea) \cap FOLLOW(A) \neq \emptyset \lightning\\
$
Damit wurde die Bedingung (ii) verletzt.\\

\gap
\task{$G_2$}
Die $FIRST$- und $FOLLOW$- Mengen ergeben sich wie folgt:\\
$
FIRST(S) = \{a,c,b\}\\
FIRST(B) = \{\eps, b\}\\
FIRST(C) = \{a\}\\
$
Da kein element mehrfach entdeckt wurde gilt (i)\\
\\
$
FOLLOW(S) = \emptyset
FOLLOW(B) = \emptyset
FOLLOW(C) = \emptyset
$
Da jede rechte Regelseite, die nicht $\eps$ ist,
    mit genau einer Variablen endet, nur eine Variable enth"alt, 
    und einem Terminalsymbol anf"angt sind alle
    $FOLLOW$-Mengen leer.\\
Ferner gilt damit dann auch (ii)\\
Damit ist $G_2$ eine LL(1) Grammtik.\\
    
\gap
\task{$G_3$}
Die $FIRST$- Mengen ergeben sich wie folgt:\\
\gap
$
FIRST(S) = \{a,b,c,d\}\\
FIRST(A) = \{a,b\}\\
FIRST(B) = \{\eps, c\}\\
FIRST(C) = \{\eps, d\}\\
$
Da kein element mehrfach entdeckt wurde gilt (i)\\
\\
Die FOLLOW Mengen ist nur f"ur $\alpha \RA^* \eps$ interessant, ferner ist das
    in $G_3$ $B,C$\\
\\
betrachte:\\
$
B \ra \eps | cS\\
FIRST(cS) = \{c\}\\
\\
FOLLOW(B) = (FIRST(C) \cup FIRST(A)) - \{\eps\}\\
c \not \in (FIRST(C) \cup FIRST(A)) - \{\eps\}\\
\text{damit ist (ii) f"ur $B$ erf"ullt}\\
\gap
C \ra \eps | dC\\
FIRST(dC) = \{d\}\\
FOLLOW(C) = FIRST(A)\\
d \not \in FIRST(A)\\
\text{damit ist (ii) f"ur $C$ erf"ullt}\\
$\\
Da alle relevanten F"alle abgedeckt wurden ist (ii) erf"ullt und $G_3$ eine 
    LL(1) Grammtik.\\

\subsubsection\

Es muss f"ur alle Satzformen $\alpha, \beta, \gamma$ und Variablen $X \in V$ der
    gegebenen Grammtik $G_4$ mit $\sigma \in \Sigma$ gelten:\\

    (i) ($S \RA^*_r \alpha X \sigma x \RA_r \alpha \beta \sigma x
         \land \gamma \RA \alpha \beta \sigma y$)
         $\RA$ ($\gamma$ ist nicht aus $S$ ableitbar), 
         mit $x,y \in \Sigma^*, \gamma \neq \alpha X \sigma x $.\\

    (ii) ($S \RA^*_r \alpha X \RA_r \alpha \beta
         \land \gamma \RA \alpha \beta$)
         $\RA$ ($\gamma$ ist nicht aus $S$ ableitbar), 
         mit $\gamma \neq \alpha X$.\\
\\
Nun hat $G_4$ die folgenden Rechtsableitungen:\\
(1)\\
$S \RA_r aA \RA_r abCa \RA_r abcca$\\
\\
(2)\\
$S \RA_r abBE \RA_r abBab \RA_r abCab \RA_r abccab$\\
\\
\task{Annahme $G$ sei LR(1) Grammtik}\\
w"ahle $\alpha = ab, X = C, \sigma = a, x = \eps, \beta = cc, 
    \gamma = abCab, y=b$\\
\\
$
\alpha X \sigma x = abCa \RA_r abcca = \alpha \beta \sigma x\\
\gamma = abCab \RA_r abccab = \alpha \beta \sigma y\\
$\\
aus (2) geht hervor, dass $\gamma$ aus $S$ ableitbar ist $\lightning$.\\
Damit ist $G_4$ nach Widerspruch keine LR(1) Grammtik.\\
\end{document}
