\documentclass{article}
\textwidth=6in
\hoffset=0in
\voffset=0in


\usepackage[a4paper, total={6in, 8in}]{geometry}
\usepackage{amsmath}
\usepackage{amssymb}
\usepackage{stmaryrd}
\usepackage{graphicx}

\usepackage{tikz}
\usetikzlibrary{automata, arrows}
\tikzset{initial text={}}

\usepackage{pifont}
\usepackage{amssymb}
\usepackage{gensymb}
\usepackage{ngerman}
\usepackage[ampersand]{easylist}

% needs to be updated
\author{Max Springenberg, 177792}
\title{\
    GTI "Ubungsblatt 9\\
    Tutor: Marko Schmellenkamp\\
    ID: MS1\\
    "Ubung: Mi 16-18
    }
\setcounter{section}{9}
\date{}

% custom commands
% \Theta \Omega \omega
\newcommand{\tab}{\null\ \qquad}
\newcommand{\gap}{\null\ \\ \\}
\newcommand{\da}{\downarrow}
\newcommand{\la}{\leftarrow}
\newcommand{\lA}{\leftarrow}
\newcommand{\ra}{\rightarrow}
\newcommand{\rA}{\rightarrow}
\newcommand{\LA}{\Leftarrow}
\newcommand{\RA}{\Rightarrow}
\newcommand{\Ra}{\Rightarrow}
\newcommand{\Lra}{\Leftrightarrow}
\newcommand{\ue}{\infty}
\newcommand{\eps}{\epsilon}
\newcommand{\task}[1]{\textbf{#1} \gap}
\newcommand{\cmark}{\ding{51}}
\newcommand{\xmark}{\ding{55}}
\newcommand{\degr}{\null \degree}
\newcommand{\error}{\task{FEHLER:}}
\newcommand{\correction}{\task{KORREKTUR:}}
\newcommand{\mdef}{\overset{\text{def}}{=}}
\newcommand{\rao}[1]{\overset{#1}{\rightarrow}}
\newcommand{\automaton}[1]{
    \begin{tikzpicture}
    #1
    \end{tikzpicture}
    }
\newcommand{\nd}[4]{
    \node[#1](#2)at(#3){#4};
    }
\newcommand{\dm}{\mathbin{\scriptstyle\dot{\smash{\textstyle-}}}}
\newcommand{\s}{\rhd}
\renewcommand{\u}{\underline}


% content
\begin{document}
% title page
\maketitle
\newpage
% actual paper

% A1
\subsection\
% a)
\subsubsection{}
Gegeben sei ein 2-Kellerautomat $A$ mit \[
    A=(Q, \Sigma, \Gamma, \delta, s, \tau_0, F)
    \]
Dabei hat $\delta$ die Form\[
    \delta \subseteq 
        (Q \times (\Sigma \cup \{\eps\}) \times \Gamma \times \Gamma)
        \times (Q \times \Gamma^* \times \Gamma^*)
    \]
\gap
Folglich besteht eine Konfiguration von $A$ aus:\\
\begin{enumerate}
    \item einem Zustand $q \in Q$
    \item einem Kellerinhalt des ersten Kellers $u \in \Gamma$
    \item einem Kellerinhalt des zweiten Kellers $v \in \Gamma$
    \item der noch zu lesenden Eingabe $w \in \Sigma^*$
\end{enumerate}
Dementsprechend kann eine Solche Konfiguration auch als Tupel notiert werden,
    mit:\[
        (q,w,u,v), q \in Q, w \in \Sigma^*, u, v \in \Gamma^*
    \]
Die untersten Kellsymbole der jeweiligen Keller k"onnen, aber m"ussen nicht 
    gleich sein, wichtig ist lediglich, dass sie als solche definiert sind.\\
$\tau_n, n \in \{1,2\}$, sei das unterste Kellersymbol des n-ten Kellers.\\
Des weiteren sei ein Stratzustand $s$ nach Notation des Automaten aus der
    Aufgabenstellung definiert.\\
\gap
Daraus folgt die Startkonfiguration 
    $K_0 = (s,w,\tau_1,\tau_2), w \in \Sigma^*$, mit
    dem Eingabewort w f"ur $A$.\\
\gap
Die Folgekonfigurationsrelation $\vdash_A$ sei wie folgt definiert:\\
$
\forall 
    p,q \in Q,
    \tab \sigma \in \Sigma, y \in \Sigma^*,
    \tab \tau',\tau'' \in \Gamma, u,v,z',z'' \in \Gamma^*
    :\\
\\
(p, \sigma y, \tau'u, \tau''v) 
    \vdash_A (q, y, z'u,z''v)
    , \text{falls: } ((p, \sigma, \tau', \tau''), (q, z',z'')) \in \delta\\
\\
(p, y, \tau'u, \tau''v) 
    \vdash_A (q, y, z'u,z''v)
    , \text{falls: } ((p, \eps, \tau', \tau''), (q, z',z'')) \in \delta\\
$\\
Eine Konfiguration $K'$ ist also genau dann eine Nachfolgekonfiguration von $K$,
    wenn gilt $K \vdash_A K'$.\\
\gap
Nach Aufgabenstellung wird mit akzeptierenden Zust"anden akzeptiert.\\
Eine Eingabe $w \in \Sigma^*$ wird genau dann akzeptiert, wenn gilt:\\
\\
$
(F \neq \emptyset)
    \land (K_0 \vdash_A^* (q, \eps, w', w''), w', w'' \in \Gamma^*, q \in F)
$\\
\gap
F"ur die Semantik von $A$ bedeuted dies ferner:\[
    L(A) = \{w \in \Sigma^* | A \text{ akzeptiert } w\}
    \]
\gap
Damit $A$ deterministisch ist muss zus"atzlich mit den Konfigurationen
    $K, K_1, K_2$ gelten:\\
$
\forall K \nexists K_1, K_2: K \vdash_A K_1 \land K \vdash_A K_2 \land K_1 \neq K_2\\
$

% b)
\subsubsection{}
Das unterste Kellersymbol beider Keller sei $\s$, der erste Keller enthalte
    den Teil des Strings vom linken Rand bis zur aktuellen Position, der
    zweite Keller enthalte den Teil rechts von der aktuellen Position.\\
\\
Zun"achst soll die Eingabe komplett eingelesen werden.\\
Dies geschieht "uber Transitionsregeln:\[
    (
        (p, \sigma y, u, \s), 
        (q, \sigma u, \s)
    ) \in \delta
    \]
, die zu Nachfolgekonfigurationen der Form:\[
    (p, \sigma y, u, \s) 
        \vdash_A (q, y, \sigma u, \s),
    \]
, mit $
    \sigma \in \Sigma, y \in \Sigma^*, u \in \Gamma^*, 
    p,q \in Q - F$ f"uhren.\\
\\
Nachdem die Eingabe eingelesen wurde befindet sich $A$ also noch nicht
    in einem Akzeptierenden Zustand.\\
Fortan wird mit $\eps$-Transitionen die Touringmaschine 
    $M = (Q_M, \Gamma, \delta_M, s_M)$ simuliert.\\
\\
Dabei wird zun"achst wieder zur"uck zum Start des Strings gelaufen:\[
    (p,\eps,\tau u \s, \s v) \vdash_A (q,\eps,u\s,\s \tau v),
    \tau \in \Gamma, u,v \in \Gamma^*, p,q \in Q-F\\
    \]
, bis die Konfiguration 
    $K_{M0} = (q_{M0}, \eps, \s , \s w), q_{M0} \in Q-F, w \in \Gamma^*$
    erreicht wird.\\
\\
%Akzeptiert $M$ $\eps$ so existiert eine Transition 
%    $(
%        (q_{M0}, \eps, \s, \s w), 
%        (q_{M\eps}, \s, \s w)
%    ) \in \delta$, mit $q_{M\eps} \in F$\\
\\
Jede Transition
    $\delta_M(p_M,\tau) = (q_M, \tau', a),
    p_M,q_M \in Q_M, \tau, \tau' \in \Gamma, a \in \{\ra,\la,\da\}$
    von $M$ wird durch:\\
$\\
((p, \eps, \tau u, \tau''v), (q, u, \tau' \tau'' v))
    , \text{ wenn } a = \la\\
((p, \eps, \tau u, v), (q, u \tau' \tau'', v))
    , \text{ wenn } a = \ra\\
((p, \eps, \tau u, v), (q, u \tau', \tau'' v))
    , \text{ wenn } a = \da\\
,p,q \in Q, \tau'' \in \Gamma\\
$\\
in $A$ , mit $p,q$ denau dann akzeptierend, wenn $p_M$, bzw. $q_M \in \{ja\}$,
    simuliert.\\

% A2
\subsection\
Zu zeigen ist:\[
    Reach \leq DG-Cycle
    \]
Dazu m"ussen wir:

    (i) eine Reduktionsfunktion $f$ angeben

    (ii) beweisen, dass $f$ eine g"ultige Reduktion ist
\gap
\task{(i)}
Ein gerichteter Graph $G = (V,E)$ enth"alt genau dann einen Kreis durch 
    $v \in V$, wenn ein Weg von $v$ nach $v$ "uber mindestens einen zu $v$ 
    verschiedenen Knoten gibt.\\
Wenn man nun eine Kante von einem neuem Knoten $s$ zu $v$ und Kanten die nach
    $v$ gehen auch an einen neuen Knoten $t$ anf"ugt, so existiert mit einem 
    Solchen Kreis "uber $v$ auch zwingend ein Weg von $s$ nach $t$.\\
\gap
Aus dieser "Uberlegung folgt $f$ mit:\\
$
f(G) = f((V,E)) = G' = (V',E')\\
(1) V' = V \cup \{s,t\}\\
(2) E' = E \cup {(s,v)} \cup \{(w,t) | \forall w \in V: (w,v) \in E\}\\
$
\gap
\task{(ii)}
Nun bleibt zu zeigen, dass $f$ total und berechenbar ist, sowie dass $f$ die 
    Vor"uberlegung erf"ullt und diese korrekt ist.\\
\\
Das $f$ berechenbar ist geht aus der Definition mittles Mengenvereinigung
    hervor.\\
Das $f$ total ist, daraus, dass $f$ auf jeden Graphen $G = (V,E)$ angewandt
    werden kann, da keine Bedingungen an $V, E$ gekn"upft sind und lediglich
    Kanten und Knoten hinzugef"ugt werden.\\
\\
$f$ erf"ullt die Vor"uberlegung, da das hinzuf"ugen von den Knoten durch (1)
    und das der Kanten durch (2) umgesetzt wird.\\
\\
Nun zur Korrektheit der "Uberlegung:\\
\gap
Annahme: Es existiert ein Graph $G$ mit einem Kreis durch $v$, f"ur den
    $f(G)$ mit einem Weg von $s$ nach $t$ diesen nicht ermittelt.\\
\\
Nach Annahme existiert ein Kreis durch $v$, das bedeutet insbesondere, das
    mindestens eine Kante, die zu $v$ f"uhrt von $v$ und damit auch von
    $s$, mit $(s,v) \in E'$ erreichbar ist. Inzidente Kanten zu $t$ wurden durch
    (2) definiert. Existiert also von einem durch $s$ und auch $v$
    erreichbaren Knoten keine Kante zu $t$, so existiert auch keine Kante zu 
    $v$.\\
Nach Annahme darf keine solche Kante existieren, das w"urde bedeuten, dass auch
    kein Weg von $v$ nach $v$ existiert und damit dann auch kein Kreis "uber $v$
    .$\lightning$\\
\\
Nach Widerspruch ist damit $f$ korrekt.\\
\\
Dadurch, dass $f$ korrekt, berechenbar und total ist, ist $f$ eine Reduktion
    f"ur $Reach \leq DG-Cycle$\\


% A3
\subsection\
% a)
\subsubsection{}
Aus den Aussagen der Aufgabenstellung folgt, dass eine Menge dann abz"ahlbar
    unendlich ist, wenn sie gleichm"achtig zur Mende $\mathbb{N}$ ist.\\
Ferner sind die Mengen $A, B$ genau dann zueinander gleichm"achtig, wenn eine
    Bijektion $f: A \ra B$ existiert.\\
Zu zeigen bleibt nun, dass eine solche Bijektion f"ur $\Sigma^*, \mathbb{N}$
    existiert.\\
Das Alphabet besteht aus den Elementen 0,1.\\
\\
Beobachtung:\\
Betrachte ein m"ogliche zuordnung von W"ortern und Zahlen:\[
    \begin{array}{ll}
        \eps    &1\\
        0       &2\\
        1       &3\\
        00      &4\\
        01      &5\\
        10      &6\\
        11      &7\\
        000     &8\\
        001     &9\\
        010     &10\\
        011     &11\\
        100     &12\\
        101     &13\\
        110     &14\\
        111     &15\\
        \ldots  &\ldots\\
    \end{array}
    \]
Wir sehen, das einerseits die umwandlung von Bin"ar- in Dezimalwerte, sowie
    auch die Wortl"ange selbst einen Faktor darstellen.\\
$b2d(w): \{0,1\}^* \ra \mathbb{N}$ sei die Funktion, die jeder Bin"arzahl die
    "aquuivalente Dezimalzahl zuordnet.\\
Wir m"ussen f"ur jede Wortl"ange $n \in \mathbb{N}$ neue Zahlen aus $\mathbb{N}$
    vergeben k"onnen. Da wir ein zweielementiges Alphabet haben bedeutet das,
    dass vor vergabe der Zahlen f"ur die n"achst gr"o"sere 
    Worl"ange $n+1$ bereits $2^n$ Zahlen vergeben wurden.
    Daraus ergibt sich ein Bias von $2^n$ f"ur alle B"inarzahlen.\\
Ferner muss das leere Wort $\eps$ auf die 1 abgebildet werden, da wir auf
    $\mathbb{N}$ abbilden und die 0 nicht enthalten ist.\\
\\
Betrachte die Funktion:\\
$f: \Sigma^* \ra \mathbb{N}$, mit\[
    f(w) \mdef
    \begin{cases}
        1, w = \eps\\
        2^{|w|} + b2d(w), \text{sonst}\\
    \end{cases}
    \]
\\
Zu zeigen bleibt, dass $f$ eine Bijektion ist.\\
\\
(1) $f$ ist injektiv:\\
$
w, w' \in \Sigma^*\\
f(w) = f(w') \overset{!}{\Leftrightarrow} w = w'\\
\\
(i) w = \eps \lor w' = \eps\\
f(\eps) = 1\\
\nexists x \in \Sigma^* - \{\eps\}: 2^{|x|} + b2d(x) = 1\\
\Ra w = \eps = w'\\
\\
(ii) w \neq \eps \land w' \neq \eps\\
f(w) = 2^{|w|} + b2d(w)\\
f(w') = 2^{|w'|} + b2d(w')\\
f(w) = f(w)\\
\Lra 2^{|w|} + b2d(w) = 2^{|w'|} + b2d(w')\\
$\\
Wenn $|w| > |w'|$, dann $2^{|w|} > 2^{|w'|} + b2d(w')$, da $ b2d(w') < 2^{|w'|}$\\
Wenn $|w| < |w'|$, dann $2^{|w'|} > 2^{|w|} + b2d(w)$, da $ b2d(w) < 2^{|w|}$\\
Wenn $|w| = |w'|$, dann $\Lra 2^{|w|} + b2d(w) = 2^{|w'|} + b2d(w')$ genau dann,
    wenn $b2d(w) = b2d(w')$, dies ist genau dann der Fall, wenn $w = w'$.\\
Damit ist $f$ injektiv.\\
\gap
(2) $f$ ist surjektiv:\\
Aussage:\\
$\\
n \in \mathbb{N} \land \exists w \in \Sigma^*: f(w) = n\\
\\
I.A.:\\
n = 1\\
f(\eps) = 1\\
\\
I.V.\\
$Die Aussage gelte f"ur $n' \in \mathbb{N}$ beliebig, aber fest.\\$
\\
I.S.\\
n = n'+1\\
n = f(w) = 2^{|w|} + b2d(w)\\
\Lra n-1 = 2^{|w|} + b2d(w) - 1\\
\Lra n' = 2^{|w|} + b2d(w) - 1\\
$Es existiert $ w'$, mit $ f(w') = 2^{|w|} + b2d(w) - 1 $, mit$:\\
|w| = |w'|, b2d(w) = b2d(w')+1,
    2^{|w|} + b2d(w') = 2^{|w|} + b2d(w) -1\\
\\
$oder falls $ w = 0^{|w|}:\\
|w'| = |w| - 1, b2d(w') = 2^{|w'|} - 1, b2d(w) = 0,\\
2^{|w'|} + 2^{|w'|} - 1 = 2 * 2^{|w| - 1} - 1 = 2^{|w|} - 1 = 2^{|w|} + 0 - 1
    = 2^{|w|} + b2d(w) - 1\\
\\
$damit gilt also$:\\
n' = 2^{|w|} + b2d(w)\\
\Lra n' = f(w')\\
$Dies gilt nach $ I.V.\\
$\\
Damit ist $f$ surjektiv.\\
\\
Da $f$ in- und surjektiv ist, ist es ferner auch bijektiv und damit sind dann 
    auch $\Sigma^*$ und $\mathbb{N}$ gleichm"achtig.\\

% b)
\subsubsection{}
Im Rahmen der Veranstaltung haben wir keien Ordnung auf Mengen definiert, wir
    k"onnen aber annehmen, dass es Tupel $T$ gibt, deren Elemente an i-ter 
    stelle, $0 \leq i < |T|, i \in \mathbb{N}_0$, fest definiert sind.\\ 
\\
Wir definieren die Funktionen $s2t: S \ra T, t2s: T \ra S$, mit:\\
$\\
s2t(S) = (s_0, \ldots, s_n), s_0
    , \ldots, s_n \in S, s_0 < \ldots < s_n, n \in \mathbb{N}\\
t2s((t_0, \ldots, t_n)) = \{t_0, \ldots, t_n\}
    ,  n \in \mathbb{N}\\
$, die eine Menge mit ausschlie"slich unterschiedlichen Elementen auf ein 
    aufsteigend sortiertes Tupel und ein Tupel auf eine Menge abbilden.\\
\\
Wir definieren die Funktionen 
    $map_t: f \times T \ra T, map_s: f \times S \ra S$, mit:$\\
map_t(f, (a_0, \ldots, a_n)) = (f(a_0), \ldots, f(a_n)), n \in \mathbb{N}\\
map_s(f, \{a_0, \ldots, a_n\}) = \{f(a_0), \ldots, f(a_n)\}, n \in \mathbb{N}\\
$
, die auf jedes Element von einem Tupel T/ einer Menge S, die Funktion f 
    anwenden.\\
\\
Die Ordnungsstruktur erleichtert uns sp"ater den Beweis.\\
\gap
Wir wissen, dass $\Sigma^*$ abz"ahlbar unendlich ist und damit Bijektionen
    $b : \Sigma^* \ra \mathbb{N}$ und $b^{-1} : \mathbb{N} \ra \Sigma^*$ 
    existieren.\\
Ferner bedeutet dies, dass wenn gezeigt werden kann, dass $P(\mathbb{N})$
    "uberabz"ahlbar unendlich ist, dann auch 
    $P(map_s(b^{-1},\mathbb{N})) = P(\Sigma)$\\
\\
Wir betrachten alle Unendlichen Mengen $M \in P(\mathbb{N})$, die zueinander
    verschieden sind und jedes Element nur einzeln enthalten.\\
M lie"se sich in etwa wie folgt rekursiv definieren:\\
$M = \{S \in P(\mathbb) |
    \text{S ist unendlich und enth"lt jedes element nur einmal}, S \not \in M\}$\\
Nun bilden wir die Menge $M_t$, die alle Mengen als aufsteigend sortierte Tupel
    enth"alt.\\
$M_t = map_s(s2t, M)$.\\
Alle Tupel aus $M_t$ enthalten also noch
    die selben Elemente wie zuvor die jeweiligen Mengen in $M$.\\
Wir nehmen an, dass jeder Zahl $n \in \mathbb{N}$ ein Element aus $M$
    zugeordnet wurde, M also abz"ahlbar unendlich sei.\\
\\
Wir definieren $T_t = (T_0, \ldots, T_i)$, als alle 
    Tupel aus $M_t$, f"ur die gilt $t_{i,j} < t_{i+1,j}, i,j \in \mathbb{N}$, 
    wobei $t_{i,j}$ das $j-te$ Element aus dem $i-ten$ Tupel aus $T_t$ ist.
    Dann kann zun"achst das Tupel $
    T = (t_{0,0}, \ldots, t{i,j}), i,j \in \mathbb{N}_0, i=j
    $ gebildet werden. $T$ enth"alt also das $j-te$ Element, des $i-ten$ Tupel 
    aus $T_t$, mit $i=j$ und ist damit auch aufsteigend sortiert.\\
\\
Wenden wir nun auf $T$, die Funktion $add1(x) = x+1$ an, so erhalten wir
    mit:\[
    T' = map_t(add1, T)
    \]
ein Tupel, dass zu jedem Tupel aus $T_t$ ein
    Element $t_j \in T'$, mit $t_j \neq t_{i,j}, i,j \in \mathbb{N}_0$ enth"alt.\\
\\
Ferner ist das Tupel $T'$ damit nicht in $T_t$ enthalten, obwohl es die Kriterien 
    f"ur $T_t$ erf"ullt und auch $S=t2s(T')$ ist nicht in $M$, da sonst
    $T'$ in $T_t$ w"are.\\
Ferner gilt, das $M$ damit nach Diagonalisierung "uberabz"ahlbar unendlich
    ist. $\lightning$\\
\\
Da $M \subset P(\mathbb{N})$ und $M$ "uberabz"ahlbar unendlich ist, ist dann auch
    die echt m"achtigere Menge $P(\mathbb{N})$ auch "uberabz"ahlbar.\\
\gap
Nun wissen wir, dass $P(\mathbb{N})$ "uberabz"ahlbar unendlich ist.\\
Wir K"onnen analog zu dem Beweis von $P(\mathbb{N})$ agieren, indem wir 
    zun"auchst $P(\mathbb{N}) = P(map_s(b, \Sigma^*))$ bilden, dann wie zuvor 
    vorgehen und argumentieren, dass 
    $map_s(b^{-1}, S) \not \in \{map_s(b^{-1}, m) | m \in M\}$ gilt
    und dadurch den Widerspruch erlangen.\\
\\
Einfacher ist es aber zu sagen, dass $\Sigma^*$ und $\mathbb{N}$ gleichm"achtig
    und damit dann auch $P(\Sigma^*), P(\mathbb{N})$ gleichm"achtig sind. Da
    $P(\mathbb{N})$"uberabz"ahlbar unendlich ist, ist jede andere gleichm"achtige
    Menge dann auch "uberabz"ahlbar unendlich.\\
Damit ist auch $P(\Sigma^*)$ nach Diagonalisierung "uberabz"ahlbar unendlich.\\
\end{document}
