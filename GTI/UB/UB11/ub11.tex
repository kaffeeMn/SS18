\documentclass{article}
\textwidth=6in
\hoffset=0in
\voffset=0in


\usepackage[a4paper, total={6in, 8in}]{geometry}
\usepackage{amsmath}
\usepackage{amssymb}
\usepackage{stmaryrd}
\usepackage{graphicx}

\usepackage{tikz}
\usetikzlibrary{automata, arrows}
\tikzset{initial text={}}

\usepackage{pifont}
\usepackage{amssymb}
\usepackage{gensymb}
\usepackage{ngerman}
\usepackage[ampersand]{easylist}
\usepackage{xcolor}

% needs to be updated
\author{Max Springenberg, 177792}
\title{\
    GTI "Ubungsblatt 11\\
    Tutor: Marko Schmellenkamp\\
    ID: MS1\\
    "Ubung: Mi 16-18
    }
\setcounter{section}{11}
\date{}

% custom commands
% \Theta \Omega \omega
\newcommand{\tab}{\null\ \qquad}
\newcommand{\gap}{\null\ \\ \\}
\newcommand{\da}{\downarrow}
\newcommand{\la}{\leftarrow}
\newcommand{\lA}{\leftarrow}
\newcommand{\ra}{\rightarrow}
\newcommand{\rA}{\rightarrow}
\newcommand{\LA}{\Leftarrow}
\newcommand{\RA}{\Rightarrow}
\newcommand{\Ra}{\Rightarrow}
\newcommand{\Lra}{\Leftrightarrow}
\newcommand{\ue}{\infty}
\newcommand{\eps}{\epsilon}
\newcommand{\task}[1]{\textbf{#1} \gap}
\newcommand{\cmark}{\ding{51}}
\newcommand{\xmark}{\ding{55}}
\newcommand{\degr}{\null \degree}
\newcommand{\error}[1]{\colorbox{red}{\task{FEHLER:}\\#1}}
\newcommand{\correction}[1]{\colorbox{green}{\task{KORREKTUR:}\\#1}}
\newcommand{\mdef}{\overset{\text{def}}{=}}
\newcommand{\rao}[1]{\overset{#1}{\rightarrow}}
\newcommand{\automaton}[1]{
    \begin{tikzpicture}
    #1
    \end{tikzpicture}
    }
\newcommand{\nd}[4]{
    \node[#1](#2)at(#3){#4};
    }
\newcommand{\dm}{\mathbin{\scriptstyle\dot{\smash{\textstyle-}}}}
\newcommand{\s}{\rhd}
\renewcommand{\u}{\underline}


% content
\begin{document}
% title page
\maketitle
\newpage
% actual paper

% A1
\subsection\
% a)
\subsubsection\
Wir betrachten das Hilfs-Programm Prog in Pseudo-Code:
\\
Eingaben: 
    Knoten $v$, 
    Array mit boolschen Werten von markierten Knoten $mk$,
    $k \in \mathbb{N}$\\
\begin{enumerate}
    \item IF $k>0$ THEN
    \item \tab IF $\neg \bigvee_{u \in adj(v)} mk[u]$ THEN
    \item \tab \tab return false
    \item \tab $V' = \{u \in adj(v) | \neg mk[u]\}$
    \item \tab return $\bigvee_{u \in V'} Prog(u,mk',k-1)$
                    , mit $mk'$ als $mk$ mit $mk[u] = true$
    \item return true
\end{enumerate}
Dieses Programm testet, ob von einem Knoten aus alle m"oglichen Wege bis zur
    L"ange k.\\
Dabei wird "uber die Adjazenzlisten in $O(n^2)$ iteriert und insegesamt $O(n^2)$
    rekursive Aufrufe, je Rekursionsschritt get"atigt, die jedoch nur maximal
    k mal ausgef"uhrt werden, mit $n = |V|$.\\
$O(\prod_{i=1}^{k} n^2) = O(n^{2*k})$\\
\\
Da $k$ eine konstante ist ,ist die Laufzeit f"ur k-WEG-FORTSETZUNG noch 
    polynomiell.\\
Ist k nun aber eine codierte Eingabe, die eine Zahl repr"asentiert, so gilt
    f"ur den Algorithmus, dass er $O(n^{2 * 2^|k|})$ ben"otigt, da exponentiell
    gro"se Zahle in logarithmischer L"ange codiert werden. Ferner ist die 
    Laufzeit dann nichtmehr polynomiell.\\

% b)
\newpage
\subsubsection\
$ALG_\text{WF}$ sei ein Algorithmus, der WEG-FORTSETZUNG l"ose.\\

(i) $ALG_\text{WFW}$ sei wie folgt definiert:\\
\begin{enumerate}
    \item FOR $k:=|V|$ DOWNTO 1 DO
    \item IF $ALG_\text{WF}(G,w,k)$ THEN
    \item \tab return k
\end{enumerate}
Die maximale L"ange eines Weges ist $|V|$, da jeder Knoten H"ochstens einmal
    besucht werden kann.\\
Wenn getestet werden kann, ob ein Weg der L"ange $k$ m"oglich ist, so kann von
    dem gr"o"st m"oglichem bis zu den je n"achst gr"os"sten in polynomieller 
    Zeit getestet werden, welches $k$ die L"ange der maximalen Fortsetzung w"are.\\
Damit Kann WEG-FORTSETZUNG-O in polynomieller Zeit berechnet werden, wenn
    WEG-FORTSETZUNG in polynomieller Zeit berechnet werden kann.\\

(ii) $ALG_\text{WFO}$ sei wie folgt definiert:\\
\begin{enumerate}
    \item $k_\text{max}$ = $ALG_\text{WFW}$(G,w)
    \item $\forall v \in V$ DO
    \item $G' = (V - \{v\}, E - \{\{v,u\}| u \in V\})$
    \item IF $ALG_\text{WFW}(G,w) = k_\text{max}$ THEN
    \item \tab \tab G = G'
    \item return V
\end{enumerate}
Wenn die L"ange einer maximalen Fortsetzung ermittelt werden kann, so kann f"ur
    jeden Knoten, sowie dann auch seinen inzidenten Kanten in polynomieller Zeit
    getestet werden, ob diese Auf dem Weg liegen, bzw. in dem Weg enthalten sind.\\
Wenn man alle Knoten und Kanten entfernt, die nicht auf dem Weg liegen, bzw.
    in diesem enthalten sind, so bleibt nur noch der Weg "ubrig.\\
Damit Kann WEG-FORTSETZUNG-O in polynomieller Zeit berechnet werden, wenn
    WEG-FORTSETZUNG-W in polynomieller Zeit berechnet werden kann.\\

% A2
\subsection\

% a)
\subsubsection\
Eine m"ogliche L"osung w"are das Wort $w \mdef c_1 c_3 c_2$\\

% b)
\subsubsection\
Damit ein Problem in NP liegt m"ussen 3 Eigenschaften gelten:\\
\begin{enumerate}
    \item Es gibt einen Suchraum von L"osungskandidaten
    \item Die L"osungskandidaten sind polynomiell gro"s in der Eingabe
    \item Jeder L"osungskandidat kann in polynomieller Zeit getestet werden
\end{enumerate}
1.\\
Der Suchraum von L"osungen ist \[
        L_\text{Such} = \{
            w \in \Sigma^* 
            |\forall \sigma \in \Sigma: \#_\sigma(w) > 0
            \}
    \]
\\
3.\\
Die Zusatzeingabe ist ein Wort $w$ aus $L(\beta)$.\\
Nach Der Vorlesung kann zu jedem RE ein $\eps-NFA$ und zu jedem $\eps-NFA$ ein
    $NFA$.\\
Der zugeh"orige $NFA$ $A_\beta$ testet dann in polynomieller Zeit ob $w$ in der
    Sprache $L(\beta)$ ist.\\
Damit kann ein L"osungskandidat in polynomieller Zeit getestet werden.\\
\\
2.\\
Der Ausdruck, das Alphabet und die Zusatzeingabe habe allesamt polynomielle 
    Gr"o"se in ihrer Eingabe.\\
\\
Da alle Eigenschaften erf"ullt werden k"onnen gilt: $ALLEZEICHEN \in NP$\\

% c)
\subsubsection\
$
\phi = (\neg x_2 \lor \neg x_1) 
    \land (\neg x_1 \lor \neg x_3 \lor x_1)
    \land (x_2 \lor x_3 \lor \neg x_3)\\
\\
f(\phi) = (\beta_\phi, \Sigma_\phi) \text{, mit:}\\
\Sigma_\phi = \{c_1, c_2, c_3\}\\
\beta_\phi = (p(x_1) + n(x_1))
    (p(x_2) + n(x_2))
    (p(x_3) + n(x_3))\\
= (\eps\eps\eps + c_1 c_2 \eps)
    (c_1\eps\eps + \eps c_2 c_3)
    (\eps c2 c_3 + \eps\eps c_3)\\
= (\eps + c_1 c_2) (c_1 + c_2 c_3) (c_2 c_3 + c_3)\\
$
% d)
\subsubsection\
Es existieren $m$ Variablen und $k$ Klauseln in $\phi$.\\
\\
F"ur $\beta_\phi$:\\
Es wird f"ur jede Variable eine der String von Klauseln-repr"asentierenden 
    Zeichen erstellt die sie erf"ullt, wenn sie positiv - 
    und der String von Klauseln-repr"asentierenden
    Zeichen erstellt, die sie erf"ullt wenn sie neativ belegt wird.\\
Insgesamt werden also $2*m*k$ Teilausdr"ucke erstellt, verodert und 
    konkateniert.\\
Dies geht in $O(m*k)$\\
\\
F"ur $\Sigma_\phi$:\\
$k$ Klauseln-repr"asentierende Zeichen k"onnen in $O(k)$ erstellt werden.\\
Damit ist der Aufwand der Funktion in $O(m*k)$ und ferner polynomiell.\\

% e)
\subsubsection\
$w_i$ sei der $i-te$ regul"are Teilausdruck von $\beta_\phi$ mit 
    $w_i = (p(x_i) + n(x_i)), 1 \leq i \leq m$, $\beta_\phi$ kann also auch als
    konkatenation aller $w_i$ geschrieben werden.\\
\\
1. $\phi \in SAT \Ra f(\phi) \in ALLEZEICHEN$\\
\\
Wir wissen:\\
(i) Wenn die Variable $x_i, 1 \leq i \leq m$ die Klausel $\psi_j, 1 \leq j \leq k$
    unter einer Belegung $\alpha$ erf"ullt, so ist das 
    Klausel-repr"asentierende Zeichen $c_j$ in $w_i$ durch $p(x_i)$ w"ahlbar.\\
(ii) Wenn die Variable $\neg x_i, 1 \leq i \leq m$ die Klausel $\psi_j, 1 \leq j \leq k$
    unter einer Belegung $\alpha$ erf"ullt, so ist das 
    Klausel-repr"asentierende Zeichen $c_j$ in $w_i$ durch $n(x_i)$ w"ahlbar.\\
\\
Wenn es eine Belegung $\alpha$ gibt, die $\phi$ erf"ullt, so gilt, dass die 
    Variablen unter der Belegung auch alle Klauseln der KNF erf"ullen, demnach
    sind alle Klauseln-repr"asentierenden Zeichen und ferner alle Zeichen des
    Alphabets durch (i) oder (ii) in $\beta$ w"ahlbar und ein Wort mit allen
    Zeichen des Alphabets existiert.\\
Damit gilt dann auch $f(\phi) \in ALLEZEICHEN$\\
\\
2. $f(\phi) \in ALLEZEICHEN \Ra \phi \in SAT$\\
$f(\phi) \in ALLEZEICHEN \Ra 
    \exists w \in L(f(\phi)): \#_\sigma(w)>1, \forall \sigma \in \Sigma$\\
Ferner wissen wir, dass $\beta_\phi$ auch als Konkatenation der Teilausdr"ucke 
    $w_i$ geschrieben werden kann. $w_i$ enth"alt 
    Strings mit Klauseln-repr"asentierenden Zeichen f"ur ein 
    positive oder negative
    Belegung der jeweiligen Variablen $x_i$.\\
Wenn nun alle Zeichen in einem Wort $w$ vorkommen, so gibt es eine M"oglichkeit
    aus den einzelnen Teilausdr"ucken $w_i$ dieses zu bilden. Ferner existiert
    damit die M"oglichkeit alle $x_i$ so zu belegen, dass alle 
    Klauseln-repr"asentierenden Zust"ande gew"ahlt werden und ferner alle 
    Klauseln erf"ullt sind.\\
Wenn eine Belegung alle Klauseln erf"ullt, so erf"ullt sie auch $\phi$.\\
Damit gilt auch $f(\phi) \in ALLEZEICHEN \Ra \phi \in SAT$.\\
\end{document}
