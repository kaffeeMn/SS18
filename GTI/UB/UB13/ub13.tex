\documentclass{article}
\textwidth=6in
\hoffset=0in
\voffset=0in


\usepackage[a4paper, total={6in, 8in}]{geometry}
\usepackage{amsmath}
\usepackage{amssymb}
\usepackage{stmaryrd}
\usepackage{graphicx}

\usepackage{tikz}
\usetikzlibrary{automata, arrows}
\tikzset{initial text={}}

\usepackage{pifont}
\usepackage{amssymb}
\usepackage{gensymb}
\usepackage{ngerman}
\usepackage[ampersand]{easylist}
\usepackage{xcolor}

% needs to be updated
\author{Max Springenberg, 177792}
\title{\
    GTI "Ubungsblatt 12\\
    Tutor: Marko Schmellenkamp\\
    ID: MS1\\
    "Ubung: Mi 16-18
    }
\setcounter{section}{13}
\date{}

% custom commands
% \Theta \Omega \omega
\newcommand{\tab}{\null\ \qquad}
\newcommand{\gap}{\null\ \\ \\}
\newcommand{\da}{\downarrow}
\newcommand{\la}{\leftarrow}
\newcommand{\lA}{\leftarrow}
\newcommand{\ra}{\rightarrow}
\newcommand{\rA}{\rightarrow}
\newcommand{\LA}{\Leftarrow}
\newcommand{\RA}{\Rightarrow}
\newcommand{\Ra}{\Rightarrow}
\newcommand{\Lra}{\Leftrightarrow}
\newcommand{\ue}{\infty}
\newcommand{\eps}{\epsilon}
\newcommand{\task}[1]{\textbf{#1} \gap}
\newcommand{\cmark}{\ding{51}}
\newcommand{\xmark}{\ding{55}}
\newcommand{\degr}{\null \degree}
\newcommand{\error}[1]{\colorbox{red}{\task{FEHLER:}\\#1}}
\newcommand{\correction}[1]{\colorbox{green}{\task{KORREKTUR:}\\#1}}
\newcommand{\mdef}{\overset{\text{def}}{=}}
\newcommand{\rao}[1]{\overset{#1}{\rightarrow}}
\newcommand{\automaton}[1]{
    \begin{tikzpicture}
    #1
    \end{tikzpicture}
    }
\newcommand{\nd}[4]{
    \node[#1](#2)at(#3){#4};
    }
\newcommand{\dm}{\mathbin{\scriptstyle\dot{\smash{\textstyle-}}}}
\newcommand{\s}{\rhd}
\renewcommand{\u}{\underline}
\renewcommand{\phi}{\varphi}


% content
\begin{document}
% title page
\maketitle
\newpage

% A1
\subsection\
Wir beginnen mit den $\eps$-closures eines jeden Zustandes:\\
$
\begin{array}{ll}
    closure(a) & \{a,b,c\}\\
    closure(b) & \{b\}\\
    closure(c) & \{c\}\\
\end{array}\\
$\\
Im folgenden wird der Automat $A_P$, durch sukzessives verfolgen aller 
    Transitionen, ausgehend von a, konsturiert.\\
\\
Konvention:\\
Alle nicht gezeichneten Transitionen f"uhren in den Senkenzustand.\\
\\
\automaton{
    \nd{initial,state,accepting}        {abc}{0,0}{\{a,b,c\}}
    \nd{state, accepting}               {bc}{4,0}{\{b,c\}}
    \nd{state, accepting}               {c}{0,4}{\{c\}}

    \path
        (abc)
            edge [->, above, bend right] node {1} (bc)
            edge [->, loop below] node {0,2} (abc)
        (bc)
            edge [->, above, bend right] node {0,2} (abc)
            edge [->, right] node {1} (c)
    ;}

% A2
\subsection\
Sei:\\$
\Sigma = \{a,b,c\}\\
L = \{w \in \Sigma^* | \#_a(w) \text{ gerade, }, \#_b(w),\#_c(w) \text{ ungerade}\}\\
$\\
$[ \tau_a \tau_b \tau_c ], \tau_a, \tau_b, \tau\in \{u,g\}$, sei die 
    Klasse von Sprachen f"ur die die Anzahl von der Zeichen $\sigma \in \Sigma$
    ungerade, wenn $\tau_{\sigma} = u$ oder gerade, wenn $\tau_{\sigma} = g$ ist.\\
Daraus ergeben sich die "Aquivalenzklassen zu:\\
$
$[ uuu ]$, z = a\\
$[ uug ]$, z = ac\\
$[ ugu ]$, z = ab\\
$[ ugg ]$, z = abc\\
$[ guu ]$, z = \eps\\
$[ gug ]$, z = c\\
$[ ggu ]$, z = b\\
$[ ggg ]$, z = bc\\
$\\
\\
Die Konstruktion des Automaten erfolgt "uber die Zust"ande 
    $Q = \{q_{vwx} | v,w,x \in \Sigma\}$.\\
sowie den transitionen bei einlsesen eines $a$ von $q_{uwx}$ nach $a_{gwx}$
    und umgekehrt,
    eines $b$ von $q_{vux}$ nach $a_{vgx}$ und umgekehrt,
    eines $c$ von $q_{vwu}$ nach $a_{vwg}$ und umgekehrt.\\
Der akzeptierende Zustand ist $q_{guu}$, der Startzustand $q_{uuu}$.\\
Es wird $\Sigma^*$ abgedeckt, da das vorkommen eines Zeichens entweder ungerade
    oder gerade oft passieren kann. Ferner wurden alle kombinatioenen von geraden,
    und ungeraden Vorkommen abgedeckt.\\

% A3
\subsection\

\subsubsection\
Sprache ist endlich, damit regul"ar.\\

\subsubsection\
$\alpha = (a+b+c)^4(a+b+c)^*$, damit regul"ar.\\

\end{document}
