\documentclass{article}
\textwidth=6in
\hoffset=0in
\voffset=0in


\usepackage[a4paper, total={6in, 8in}]{geometry}
\usepackage{amsmath}
\usepackage{amssymb}
\usepackage{stmaryrd}
\usepackage{graphicx}

\usepackage{tikz}
\usetikzlibrary{automata, arrows}
\tikzset{initial text={}}

\usepackage{pifont}
\usepackage{amssymb}
\usepackage{gensymb}
\usepackage{ngerman}
\usepackage[ampersand]{easylist}

% needs to be updated
\author{Max Springenberg, 177792}
\title{\
    GTI "Ubungsblatt 8\\
    Tutor: Marko Schmellenkamp\\
    ID: MS1\\
    "Ubung: Mi 16-18
    }
\setcounter{section}{8}
\date{}

% custom commands
% \Theta \Omega \omega
\newcommand{\tab}{\null\ \qquad}
\newcommand{\gap}{\null\ \\ \\}
\newcommand{\lA}{\leftarrow}
\newcommand{\ra}{\rightarrow}
\newcommand{\rA}{\rightarrow}
\newcommand{\LA}{\Leftarrow}
\newcommand{\RA}{\Rightarrow}
\newcommand{\ue}{\infty}
\newcommand{\eps}{\epsilon}
\newcommand{\task}[1]{\textbf{#1} \gap}
\newcommand{\cmark}{\ding{51}}
\newcommand{\xmark}{\ding{55}}
\newcommand{\degr}{\null \degree}
\newcommand{\error}{\task{FEHLER:}}
\newcommand{\correction}{\task{KORREKTUR:}}
\newcommand{\mdef}{\overset{\text{def}}{=}}
\newcommand{\rao}[1]{\overset{#1}{\rightarrow}}
\newcommand{\automaton}[1]{
    \begin{tikzpicture}
    #1
    \end{tikzpicture}
    }
\newcommand{\nd}[4]{
    \node[#1](#2)at(#3){#4};
    }
\newcommand{\dm}{\mathbin{\scriptstyle\dot{\smash{\textstyle-}}}}
\newcommand{\s}{\rhd}
\renewcommand{\u}{\underline}


% content
\begin{document}
% title page
\maketitle
\newpage
% actual paper

% A1
\subsection\

% a)
\subsubsection\
\task{(i)}
$f(1) = \bot$\\ %\Bottom\\
$x_4$ ist bei $x_1 = 1$ nach ablauf des ersten WHILE-Programms echt gr"o"ser
    0, damit endet das Folgende WHILE-Programm nie.\\
\\
$f(2) = 4$\\

\gap
\task{(ii)}
$f_P(n) = \begin{cases}
    2^n, n \geq 2\\ 
    \bot, \text{sonst}\\
\end{cases}$\\

% b)
\subsubsection\
\begin{tabular}{ll}
    1:  & $x_3 := 0$;\\
    2:  & $x_4 := 1$;\\
    3:  & $x_5 := x_1$;\\
    4:  & $x_6 := x_2$;\\
    \\
    5:  & IF $x_1 = 0$ THEN GOTO 9;\\
    6:  & $x_1 := x_1 \dm 1$;\\
    7:  & $x_3 := x_3 + 1$;\\
    8:  & IF $x_4 = 1$ THEN GOTO 5;\\
    \\
    9:  & IF $x_3 = 0$ THEN GOTO 15;\\
    \\
    10: & IF $x_3 = 0$ THEN GOTO 20;\\
    11: & IF $x_2 = 0$ THEN GOTO 18;\\
    12: & $x_2 := x_2 \dm 1$;\\
    13: & $x_3 := x_3 \dm 1$;\\
    14: & IF $x_4 = 1$ THEN GOTO 10;\\
    \\
    15: & IF $x_2 = 0$ THEN GOTO 20; \\
    16: & $x_2 := x_2 \dm 1$;\\
    17: & IF $x_4 = 1$ THEN GOTO 15;\\
    \\
    18: & $x_1 := x_5$;\\
    19: & IF $x_4 = 1$ THEN GOTO 21;\\
    20: & $x_1 := x_6$;\\
    \\
    21: & HALT\\
\end{tabular}\\

\newpage
% A2
\subsection\

% a)
\subsubsection\
Betrachte:\\
$
w \mdef bab\\
\\
\text{Ersten 10 Konfigurationen:}\\
(q_b,(\eps,\s,bab)) \vdash (q_b, (\s,b,ab))\\
                    \vdash (q_r, (\s \u{b},a,b))\\
                    \vdash (q_r, (\s \u{b}a,b,\eps))\\
                    \vdash (q_r, (\s \u{b}ab,\sqcup,\eps))\\
                    \vdash (q_l, (\s \u{b}a,b,\u{b}))\\
                    \vdash (q_l, (\s \u{b},a,b\u{b}))\\
                    \vdash (q_l, (\s ,\u{b},ab\u{b}))\\
                    \vdash (q_l, (\eps ,\s,\u{b}ab\u{b}))\\
                    \vdash (q_b, (\s ,\u{b},ab\u{b}))\\
\\
\text{Erste Konfiguration in $q_c$:}\\
(q_c, (\s\u{b}a\u{b}\u{b},\u{b}, \sqcup))\\
$

% b)
\subsubsection\
Bedeutung der Zust"ande:\\
\task{$q_b$}
$q_b$ lie"st den gesamten String von links nach rechts, bis er endet, bzw. ein 
    $\sqcup$ auftritt, oder ein b gelesen wird.\\
Wenn ein b gelesen wird, so wird dieses durch ein markiertes b ($\u{b}$)
    §ersetzt und in den Zustand $q_l$ gewechselt.\\
Wenn ein $\sqcup$ gelesen wird, hat der String keine unmarkierten b mehr und es
    wird in den Zustand $q_c$ gewechselt.\\
\gap
\task{$q_c$}
$q_c$ liest den gesamten String von links nach rechts und ersetzt alle 
    markierten b mit unmarkierten, bis das Startsymbol erreicht wird.\\
In diesem Fall wird aufgeh"ort den Lesekopf zu bewegen und in den Zustand h
    gewechselt.\\
\gap
\task{$q_r$}
$q_r$ verschiebt den Lesekopf solange nach rechts, bis der String endet, bzw.
    ein $\sqcup$ gelesen wird.\\
Dann wird dieses durch ein markiertes b ersetzt, bzw. ein markiertes b an den
    String hinten angehangen.\\
\gap
\task{$q_l$}
$q_l$ verschiebt den Lesekopf solange nach links, bis das Startsymbol gelesen
    wird.\\
Dann wird in den Zustand $q_b$ gewechselt.\\
\gap
\task{$h$}
Im Zustand h h"alt die Touringmaschine.\\
\gap
Aus der Bedeutung der Zust"ande geht hervor, dass:\\
Die Touringmaschine jedes b markiert und anschlie"send ein markiertes b an den
    String hinten anh"angt, bis es keine unmarkierten b mehr gibt.\\
Abschlie"send werden alle markierten b durch normale b ersetzt und die TM h"alt.\\
Da nach endlichen Operationen bei endlichen Eingaben alle b markiert worden sind
    h"alt die TM auch f"ur alle Eingaben aus $\{a,b\}^*$ und weisst damit keine
    Definitionsl"ucken auf.\\
\\
F"ur die Funktion $f_M$ der TM bedeutet das, dass die Touring maschine aus einem
    Wort $w \in \{a,b\}^*$ ein Wort $v \in \{a,b\}$, das aus w konkateniert mit
    $n=\#_b(w)$ b besteht
    , also der Form $v = wb^{n=\#_b(w)}$ macht.\\
Dadurch ergibt sich dann auch die Funktion $f_M$ zu:\[
    f_M(w) = wb^{n=\#_b(w)}
    \]

\newpage
% A3
\subsection\

% a)
\subsubsection\

% b)
\subsubsection\

\end{document}
