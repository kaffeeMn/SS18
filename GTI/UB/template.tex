\documentclass{article}
\usepackage{amsmath}
\usepackage{amssymb}
\usepackage{stmaryrd}
\usepackage{graphicx}
\usepackage{tikz}
\usetikzlibrary{automata, arrows}
\usepackage[ampersand]{easylist}

% needs to be updated
\author{Max Springenberg, 177792}
\title{\
}
\setcounter{section}{1}
\date{}

% custom commands
% \Theta \Omega \omega
\newcommand{\tab}{\null\ \qquad}
\newcommand{\gap}{\null\ \\ \\}
\newcommand{\lA}{$\leftarrow$}
\newcommand{\rA}{$\rightarrow$}
\newcommand{\ue}{$\infty$}
\newcommand{\eps}{$\epsilon$}
\newcommand{\task}[1]{\textbf{#1} \\ \gap}

% content
\begin{document}
% title page
\maketitle
\newpage
% actual paper

\subsection\
\subsubsection{\
    Seien $\beta = (ab)^* $ und $ \alpha_1, ... , \alpha_8 $ die folgenden 
    erweiterten regulären Ausdrücke. Beurteilen Sie für alle
    $i \in\{3, ... , 8\}$, ob $L(\alpha_i) \subseteq L(\beta)$ gilt. 
    Vervollständigen Sie dazu die folgende Tabelle analog zu den 
    Beispiel-Ausdrücken $\alpha_1, \alpha_2$: 
    Falls $L(\alpha_i \not \subseteq L(\beta)$ gilt, geben Sie ein Wort 
    $w_i \in L(\apha_i) - L(\beta)$ an.
    }
\begin{tabular}{l|l|l}
    RE              &$L(\alpha_i) \subseteq L(\beta)$   &Gegenbeispiel\\
    \hline \\

\end{tabular}
\subsection\
\subsubsection{}
\subsection\
\subsubsection{}
\subsection\
\subsubsection{}
\end{document}
