\documentclass{article}
\textwidth=6in
\hoffset=0in
\voffset=0in


\usepackage[a4paper, total={6in, 8in}]{geometry}
\usepackage{amsmath}
\usepackage{amssymb}
\usepackage{stmaryrd}
\usepackage{graphicx}

\usepackage{tikz}
\usetikzlibrary{automata, arrows}
\tikzset{initial text={}}

\usepackage{pifont}
\usepackage{amssymb}
\usepackage{gensymb}
\usepackage{ngerman}
\usepackage[ampersand]{easylist}

% needs to be updated
\author{Max Springenberg, 177792}
\title{\
    GTI "Ubungsblatt 6\\
    Tutor: Marko Schmellenkamp\\
    ID: MS1\\
    "Ubung: Mi 16-18
    }
\setcounter{section}{6}
\date{}

% custom commands
% \Theta \Omega \omega
\newcommand{\tab}{\null\ \qquad}
\newcommand{\gap}{\null\ \\ \\}
\newcommand{\lA}{\leftarrow}
\newcommand{\rA}{\rightarrow}
\newcommand{\LA}{\Leftarrow}
\newcommand{\RA}{\Rightarrow}
\newcommand{\ue}{\infty}
\newcommand{\eps}{\epsilon}
\newcommand{\task}[1]{\textbf{#1} \\ \gap}
\newcommand{\cmark}{\ding{51}}
\newcommand{\xmark}{\ding{55}}
\newcommand{\degr}{\null \degree}
\newcommand{\error}{\task{FEHLER:}}
\newcommand{\correction}{\task{KORREKTUR:}}
\newcommand{\mdef}{\overset{\text{def}}{=}}
\newcommand{\rao}[1]{\overset{#1}{\rightarrow}}
\newcommand{\automaton}[1]{
    \begin{tikzpicture}
    #1
    \end{tikzpicture}
    }
\newcommand{\nd}[4]{
    \node[#1](#2)at(#3){#4};
    }


% content
\begin{document}
% title page
\maketitle
\newpage
% actual paper

% A1
\subsection\
% A1 a)
\subsubsection\
gegeben:\\
Grammatik $G$ mit:
$
\begin{array}{llllll}
    S   &\rA&   A\\
    A   &\rA&   aAB &|bA    &|cA    &|\eps\\
    B   &\rA&   a\\
\end{array}\\
$\\
gesucht:\\
Kellerautomat, der die Sprache \[
    L(G) = \{u(av_ia)^k
        | mit i \in \{1,\ldots n\}, v_i, u \in \{b,c\}^*, k \in \mathbb{N}_0
        \}
    \] entscheidet\\
\\
Eine m"ogliche L"osung ist der PDA $A$ mit:\\
\automaton{
    \nd{initial, state}{q}{0,0}{q}

    \path
        (q)
            edge [->, loop right] node {
                \begin{tabular}{l}
                $\eps$, S:A\\
                $\eps$, A:aAB\\
                $\eps$, A:bA\\
                $\eps$, A:cA\\
                $\eps$, A:$\eps$\\
                $\eps$, B:a\\
                a, a:$\eps$\\
                b, b:$\eps$\\
                c, c:$\eps$\\
            \end{tabular}
            } (q);
    }\\
\\
$A$ wurde nach dem Vorgehen der Vorlesung konstruiert.\\
\\
$A$ Kann "uber seine $\eps$-Transitionen alle Regeln zu den jeweiligen Variablen
    aufbauen und damit auch s"amtliche Ableitungen von $G$.\\
Die jeweilige gew"ahlte rechte Regelseite wird auf den Keller gelegt.\\
Nach einlesen eines Terminalsymbols $\sigma \in \Sigma$ wird dieses vom Keller
    gel"oscht, wenn nun eine Variable oben auf dem Keller liegt kann diese
    wieder abgeleitet, bzw. ihre rechte Regelseite auf den Keller gepackt
    werden.\\
Insbesondere werden hierbei solange Terminalsymbole aus $\{b,c\}$ auf den Keller
    gelegt und nach Einlesen gel"oscht, bis ein a auf den Keller gelegt und 
    nach Einlesen gel"oscht wird. Dann k"onnen
    wieder Terminalsymbole aus $\{b,c\}$ auf den Keller gelegt werden und 
    nach Einlsen gel"oscht werden, aber es wird insbesondere ein a am Ende auf 
    den Keller gelegt und nach Einlsen gel"oscht. Damit gilt, dass 
    $A$ W"orter der Form $
    L(G) = \{u(av_ia)^k
        |\text{, mit} i \in \{1,\ldots n\}, v_i, u \in \{b,c\}^*, k \in \mathbb{N}_0\}
    $ mit leerem Keller akzeptiert und ferner $L(G)$ entscheidet.\\
\\
    
% A1 b)
\subsubsection\
\task{(i)}
gegeben:\\
$w_1 = ab, w_2 = abaa, w_3 = abaaaa$\\
\gap
$w_1$:\\
$A$ akzeptiert $w_1$ nicht, da nach dem Einlsen des letzten Zeichen $b$ ein b 
    oben auf dem 
    Keller liegt und eine Transition zum Keller-leerenden Zustand 2 nicht
    mehr m"oglich ist.\\
\\
\\
$w_2$:\\
$A$ akzeptiert $w_2$ mit leerem Keller:$\\
\begin{array}{lll}
    (1,abaa,\#)     &\vdash      &(1,baa, a\#)\\
                    &\vdash      &(1,aa, ba\#)\\
                    &\vdash      &(1,a, aba\#)\\
                    &\vdash      &(1,\eps, aaba\#)\\
                    &\vdash      &(2,\eps, aaba\#)\\
                    &\vdash      &(2,\eps, aba\#)\\
                    &\vdash      &(2,\eps, ba\#)\\
                    &\vdash      &(2,\eps, a\#)\\
                    &\vdash      &(2,\eps, \#)\\
                    &\vdash      &(2,\eps, \eps)\\
\end{array}
$\\
\\
$w_3$:\\
$A$ akzeptiert $w_3$ mit leerem Keller:$\\
\begin{array}{lll}
    (1,abaaaa,\#)   &\vdash      &(1,baaaa, a\#)\\
                    &\vdash      &(1,aaaa, ba\#)\\
                    &\vdash      &(1,aaa, aba\#)\\
                    &\vdash      &(1,aa, aaba\#)\\
                    &\vdash      &(1,a, aaaba\#)\\
                    &\vdash      &(1,\eps, aaaaba\#)\\
                    &\vdash      &(2,\eps, aaaaba\#)\\
                    &\vdash      &(2,\eps, aaaba\#)\\
                    &\vdash      &(2,\eps, aaba\#)\\
                    &\vdash      &(2,\eps, aba\#)\\
                    &\vdash      &(2,\eps, ba\#)\\
                    &\vdash      &(2,\eps, a\#)\\
                    &\vdash      &(2,\eps, \#)\\
                    &\vdash      &(2,\eps, \eps)\\
\end{array}
$\\

% A1 c)
\task{(ii)}
Regeln f"ur die Variablen $X_{1,\tau,1}$ und $X_{1,\tau,2}$, mit $\tau \in \Gamma$
    waren bereits gegeben.\\
F"ur die Variablen $X_{2,\tau,1}$, mit $\tau \in \Gamma$ gilt, dass sie nicht
    erzeugend sind, da von 2 aus keine Transition zu 1 existiert.\\
Es w"urden sich ausschlie"slich Regeln, der Form:$
    X_{2,\tau,1} \rA X_{2,\tau',2} X_{2,\tau,1}
    $, mit $\tau, \tau' \in \Gamma$ ergeben, die keine endliche 
    Ableitung besitzen.\\
Deshalb k"onnen diese nicht erzeugenden Variablen und Regeln, die sie enthalten,
    gestrichen werden.\\
\\
Die Regeln der Form $X_{2,\tau,2}$, mit $\tau \in \Gamma$ ergeben sich zu:$\\
\begin{array}{lll}
    X_{2,\#,2}  &\rA    & \eps\\
    X_{2,a,2}   &\rA    & a\\
    X_{2,b,2}   &\rA    & a\\
\end{array}\\
$\\
Nachdem wir nun alle notwendigen Regeln aufgestellt haben w"ahlen wir das 
    Startsymbol gem"a"s der Vorlesung mit:\[
    \begin{array}{llll}
        S &\rA &X_{1,\#,1} &|X_{1,\#,2}\\
    \end{array}
    \]\\

% A2
\subsection\
gegeben:\\
$\Sigma = \{N_1, N_2, S_1, S_2\}$\\
\\
wir w"ahlen:$\\
    \tau_0 =  \#, \Gamma = \{\#, n, s\}\\
    $
\\
Eine m"ogliche L"osung unter dem gew"ahlten Kelleralphabet ist:\\
\automaton{
    \nd{initial, state}{q}{0,0}{q}
    \path
        (q) 
            edge [->, loop right] node {
                \begin{tabular}{l}
                    $\eps$, \# : $\eps$\\
                    $N_1$, * : n*\\
                    $N_1$, s : $\eps$\\
                    $S_1$, * : s*\\
                    $S_1$, n : $\eps$\\
                    $N_2$, * : nn*\\
                    $N_2$, s : n\\
                    $N_2$, ss : $\eps$\\
                    $S_2$, * : ss*\\
                    $S_2$, n : s\\
                    $S_2$, nn : $\eps$\\
                \end{tabular}
            }(q);
    }\\
\\
Dieser Kellerautomat ist sinnvoll gew"ahlt, da ein leerer Spaziergung und ferner
    das Beenden eines Spaziergangs durch $\eps, \# \eps$ m"oglich ist, Schritte
    jeweils ein Symbol f"ur die Schrittrichtung auf den Keller legen oder eines
    der Gegenrichtung l"oschen und die Spr"unge analog vorgehen, wobei zwei
    Symbole der Gegenrichtung gel"oscht, eines der Gegenrichtung durch eines 
    der Schrittrichtung ausgetauscht wird oder zwei der Schrittrichtung auf den
    Keller gelegt werden.\\
Dadurch l"asst sich das `"uber einem Punkt hin und her wandern' einfach
    simulieren.\\

% A3
\subsection\
% A3 a)
\subsubsection{$L_1 = \{a^lb^mc^pd^q | l,m,p,q \in \mathbb{N}_0, l<p \land q<m\}$}
Wir zeigen, dass $L$ nicht kontextfrei ist nach dem Korrolar des Pumpinglemma
    f"ur kontextfreie Sprachen.\\
\\
Insbesondere gilt f"ur alle W"orter aus $L$ die Eigenschaft:\[
    \#_a(w)<\#_c(w) \land \#_b(w)<\#_d(w)
\]
\\
wir w"ahlen:\[
    z \mdef a^n b^{n+1} c^{n+1} d^n
\]
wir betrachten Zerlegungen der Form $z = uvwxy$, mit:$\\
vx \neq \eps\\
|vwx| \leq n\\
$\\
Fortan werden 4 F"alle betrachtet:\\

(i) mindestens 1 $a$ in vx, aber kein $c$,
    da zwischen $a$ und $c$ $(n+1)$ $b$'s liegen und $|vwx| \leq n$.\\

(ii) mindestens 1 $b$ in vx, aber kein $d$,
    da zwischen $b$ und $d$ $(n+1)$ $c$'s liegen und $|vwx| \leq n$.\\

(iii) mindestens 1 $c$ in vx, aber kein $a$,
    da zwischen $a$ und $c$ $(n+1)$ $b$'s liegen und $|vwx| \leq n$.\\

(iv) mindestens 1 $d$ in vx, aber kein $b$,
    da zwischen $d$ und $b$ $(n+1)$ $c$'s liegen und $|vwx| \leq n$.\\
\gap
(i)\\
w"ahle $k=2$:\\
Da $c$'s vom aufpumpen unaffektiert sind, bleiben $n+1$ $c$'s.\\
Da mindestens ein $a$ aufgepumpt wird gilt, dass nun $\#_a(uv^kwx^ky) \geq n+1$
    existieren, damit gilt $\#_a(uv^kwx^ky) \geq \#_c(uv^kwx^ky)$\\
Dies verletzt die zuvor genannnte Eigenschaft von $L_1$.\\
\\
(ii)\\
w"ahle $k=2$:\\
Da $d$'s vom aufpumpen unaffektiert sind, bleiben $n+1$ $d$'s.\\
Da mindestens ein $b$ aufgepumpt wird gilt, dass nun $\#_b(uv^kwx^ky) \geq n+1$
    existieren, damit gilt $\#_b(uv^kwx^ky) \geq \#_d(uv^kwx^ky)$\\
Dies verletzt die zuvor genannnte Eigenschaft von $L_1$.\\
\\
(iii)\\
w"ahle $k=0$:\\
Da $a$'s vom aufpumpen unaffektiert sind, bleiben $n$ $a$'s.\\
Da mindestens ein $c$ abgepumpt wird gilt, dass nun $\#_c(uv^kwx^ky) \leq n$
    existieren, damit gilt $\#_a(uv^kwx^ky) \geq \#_c(uv^kwx^ky)$\\
Dies verletzt die zuvor genannnte Eigenschaft von $L_1$.\\
\\
(iv)\\
w"ahle $k=0$:\\
Da $b$'s vom aufpumpen unaffektiert sind, bleiben $n$ $b$'s.\\
Da mindestens ein $b$ abgepumpt wird gilt, dass nun $\#_b(uv^kwx^ky) \leq n$
    existieren, damit gilt $\#_b(uv^kwx^ky) \geq \#_d(uv^kwx^ky)$\\
Dies verletzt die zuvor genannnte Eigenschaft von $L_1$.\\
\\
Da $vx$ mindestens ein Zeichen aus $\{a,b,c,d\}$ enthalten muss, wurde f"ur 
    alle relevanten F"alle ein Pumpfaktor $k$ gefunden, f"ur den das
    auf- oder abgepumpte Wort nichtmehr in $L_1$ ist.\\
\\
nach dem Korrolar zum Pumpinglemma ist $L_1$ damit nicht kontextfrei.\\

% A3 b)
\subsubsection{$L_2 = \{ww^Rw | w \in \{a,b\}^*\}$}
Wir zeigen, dass $L$ nicht kontextfrei ist nach dem Korrolar des Pumpinglemma
    f"ur kontextfreie Sprachen.\\
\\
wir w"ahlen:\[
    z \mdef ba^nbba^nbba^nb
\]
Wir betrachten Zerlegungen der Form $z = uvwxy$, mit:$\\
    vx \neq \eps\\
    |vwx| \leq n\\
$\\
Fortan werden 2 F"alle betrachtet:\\
(i) genau ein $b$ und m"oglicher Weise $a$'s in $vx$, da zwischen $b$'s $n$ $a$'s 
    liegen\\
(ii) nur $a$'s und insbesondere mindestens ein $a$ in $vx$\\
\gap
(i)\\
w"ahle $k=0$:\\
Da genau ein $b$ abgepumpt wird und jedes Wort mit einem $b$ anf"angt und 
    aufh"ort, kann das abgepumpte Wort $uwy$ nicht die Form $ww^Rw$ haben, da
    ein b weniger und insbesondere 4 $b$'s "ubrig sind, die nicht gleichm"a"sig
    auf 3 Teilw"orter aufgeteilt werden k"onnen.\\
\\
(ii)\\
w"ahle $k=0$:\\
Da im verbleibenden Fall nur $a$'s in $vx$ vorkommen und alle Teilw"orter aus 
    der Definition von $L$ durch 
    $b$'s geklammert sind, sind die $a$'s aus $vx$ aus einem Teilwort.\\
Wir k"onnen annehmen, dass:\\
    (1) alle Teilw"orter $w$, mit $z=ww^Rw$ unter unserem gew"ahlten Wort z 
    die selbe Anzahl von $a$'s besitzt.\\
\\
Wenn nun mindestens ein $a$ abgepumpt wird, so besitzt ein Teilwort mindestens
    ein $a$ weniger.\\
Aus der Klammerung der Teilw"orter durch $b$'s folgt, dass die Zahl von $a$'s
    der jeweiligen Teilw"orter gleich der zwischen den $b$'s stehenden 
    Anzahl von $a$'s ist.\\
Wie gesagt enth"alt ein Teilwort ein $a$ weniger, damit ist die Eigenschaft (1)
    verletzt und das abgepumpte Wort nicht in der Sprache $L_2$.\\
\\
$vx$ mus ein $b$ oder nur $a$'s enthalten.\\
\\
nach dem Korrolar zum Pumpinglemma ist $L_2$ damit nicht kontextfrei.\\
    
\end{document}
