\documentclass{article}
\textwidth=6in
\hoffset=0in
\voffset=0in


\usepackage[a4paper, total={6in, 8in}]{geometry}
\usepackage{amsmath}
\usepackage{amssymb}
\usepackage{stmaryrd}
\usepackage{graphicx}

\usepackage{tikz}
\usetikzlibrary{automata, arrows}
\tikzset{initial text={}}

\usepackage{pifont}
\usepackage{amssymb}
\usepackage{gensymb}
\usepackage{ngerman}
\usepackage[ampersand]{easylist}

% needs to be updated
\author{Max Springenberg, 177792}
\title{\
    GTI "Ubungsblatt 6\\
    Tutor: Marko Schmellenkamp\\
    ID: MS1\\
    "Ubung: Mi 16-18
    }
\setcounter{section}{6}
\date{}

% custom commands
% \Theta \Omega \omega
\newcommand{\tab}{\null\ \qquad}
\newcommand{\gap}{\null\ \\ \\}
\newcommand{\lA}{\leftarrow}
\newcommand{\rA}{\rightarrow}
\newcommand{\LA}{\Leftarrow}
\newcommand{\RA}{\Rightarrow}
\newcommand{\ue}{\infty}
\newcommand{\eps}{\epsilon}
\newcommand{\task}[1]{\textbf{#1} \\ \gap}
\newcommand{\cmark}{\ding{51}}
\newcommand{\xmark}{\ding{55}}
\newcommand{\degr}{\null \degree}
\newcommand{\error}{\task{FEHLER:}}
\newcommand{\correction}{\task{KORREKTUR:}}
\newcommand{\mdef}{\overset{\text{def}}{=}}
\newcommand{\rao}[1]{\overset{#1}{\rightarrow}}
\newcommand{\automaton}[1]{
    \begin{tikzpicture}
    #1
    \end{tikzpicture}
    }
\newcommand{\nd}[4]{
    \node[#1](#2)at(#3){#4};
    }


% content
\begin{document}
% title page
\maketitle
\newpage
% actual paper

% A1
\subsection\
% A1 a)
\subsubsection\
gegeben:\\
Grammatik $G$ mit:
$
\begin{array}{llllll}
    S   &\rA&   A\\
    A   &\rA&   aAB &|bA    &|cA    &|\eps\\
    B   &\rA&   a\\
\end{array}\\
$\\
gesucht:\\
Kellerautomat, der die Sprache \[
    L(G) = \{u(av_ia)^k
        | mit i \in \{1,\ldots n\}, v_i, u \in \{b,c\}^*, k \in \mathbb{N}_0
        \}
    \] entscheidet\\
\\
Eine m"ogliche L"osung ist der PDA $A$ mit:\\
\automaton{
    \nd{initial, state}{q}{0,0}{q}

    \path
        (q)
            edge [->, loop right] node {
                \begin{tabular}{l}
                $\eps$, S:A\\
                $\eps$, A:aAB\\
                $\eps$, A:bA\\
                $\eps$, A:cA\\
                $\eps$, A:$\eps$\\
                $\eps$, B:a\\
                a, a:$\eps$\\
                b, b:$\eps$\\
                c, c:$\eps$\\
            \end{tabular}
            } (q);
    }\\
\\
$A$ wurde nach dem Vorgehen der Vorlesung konstruiert.\\
\\
$A$ Kann "uber seine $\eps$-Regeln die alle Regeln zu den jeweiligen Variablen
    aufbauen und damit auch s"amtliche Ableitungen von $G$.\\
Die jeweilige gew"ahlte rechte Regelseite wird auf den Keller gelegt.\\
Nach einlesen eines Terminalsymbols $\sigma \in \Sigma$ wird dieses vom Keller
    gel"oscht, wenn nun eine Variable oben auf dem Keller liegt kann diese
    wieder abgeleitet werden.\\
Insbesondere werden hierbei solange Terminalsymbole aus $\{b,c\}$ auf den Keller
    gelegt und nach Einlesen gel"oscht, bis ein a auf den Keller gelegt und 
    nach Einlesen gel"oscht wird. Dann k"onnen
    wieder Terminalsymbole aus $\{b,c\}$ auf den Keller gelegt werden und 
    nach Einlsen gel"oscht werden, aber es wird insbesondere ein a am Ende auf 
    den Keller gelegt und nach Einlsen gel"oscht. Damit gilt, dass 
    $A$ W"orter der Form $
    L(G) = \{u(av_ia)^k
        | mit i \in \{1,\ldots n\}, v_i, u \in \{b,c\}^*, k \in \mathbb{N}_0
    $ mit leerem Keller akzeptiert und ferner $L(G)$ entscheidet.\\
\\
    
% A1 b)
\subsubsection\
gegeben:\\
$w_1 = ab, w_2 = abaa, w_3 = abaaaa$\\
\gap
$w_1$:\\
$A$ akzeptiert $w_1$ nicht, da nach dem Einlsen des letzten Zeichen $b$ ein b 
    oben auf dem 
    Keller liegt und eine Transition zum Keller-leerenden Zustand 2 nicht
    mehr m"oglich ist.\\
\\
\\
$w_2$:\\
$A$ akzeptiert $w_2$ mit leerem Keller:$\\
\begin{array}{lll}
    (1,abaa,\#)     &\      &(1,baa, a\#)\\
                    &\      &(1,aa, ba\#)\\
                    &\      &(1,a, aba\#)\\
                    &\      &(1,\eps, aaba\#)\\
                    &\      &(2,\eps, aaba\#)\\
                    &\      &(2,\eps, aba\#)\\
                    &\      &(2,\eps, ba\#)\\
                    &\      &(2,\eps, a\#)\\
                    &\      &(2,\eps, \#)\\
                    &\      &(2,\eps, \eps)\\
\end{array}
$\\
\\
$w_3$:\\
$A$ akzeptiert $w_3$ mit leerem Keller:$\\
\begin{array}{lll}
    (1,abaaaa,\#)     &\      &(1,baaaa, a\#)\\
                    &\      &(1,aaaa, ba\#)\\
                    &\      &(1,aaa, aba\#)\\
                    &\      &(1,aa, aaba\#)\\
                    &\      &(1,a, aaaba\#)\\
                    &\      &(1,\eps, aaaaba\#)\\
                    &\      &(2,\eps, aaaaba\#)\\
                    &\      &(2,\eps, aaaba\#)\\
                    &\      &(2,\eps, aaba\#)\\
                    &\      &(2,\eps, aba\#)\\
                    &\      &(2,\eps, ba\#)\\
                    &\      &(2,\eps, a\#)\\
                    &\      &(2,\eps, \#)\\
                    &\      &(2,\eps, \eps)\\
\end{array}
$\\

% A1 c)
\subsubsection\
Regeln f"ur die Variablen $X_{1,\tau,1}$ und $X_{1,\tau,2}$, mit $\tau \in \Gamma$
    waren bereits gegeben.\\
F"ur die Variablen $X_{2,\tau,1}$, mit $\tau \in \Gamma$ gilt, dass sie nicht
    erzeugend sind, da von 2 aus keine Transition zu 1 existiert.\\
Es w"urden sich ausschlie"slich Regeln, der Form:$
    X_{2,\tau,1} \rA X_{2,\tau',2} X_{2,\tau,1}
    $, mit $\tau, \tau' \in \Gamma$ ergeben, die keine endliche 
    Ableitung besitzen.\\
Deshalb k"onnen diese nicht erzeugenden Variablen und Regeln, die sie enthalten
    gestrichen werden.\\
\\
Die Regeln der Form $X_{2,\tau,2}$, mit $\tau \in \Gamma$ ergeben sich zu:$\\
\begin{array}{lll}
    X_{2,\#,2}  &\rA    & \eps\\
    X_{2,a,2}   &\rA    & a\\
    X_{2,b,2}   &\rA    & a\\
\end{array}\\
$\\
Nachdem wir nun alle notwendigen Regeln aufgestellt haben w"ahlen wir das 
    Startsymbol gem"a"s der Vorlesung mit:\[
    \begin{array}{llll}
        S &\rA &X_{1,\#,1} &|X_{1,\#,2}\\
    \end{array}
    \]\\

% A2
\subsection\
gegeben:\\
$\Sigma = \{N_1, N_2, S_1, S_2\}$\\
\\
wir w"ahlen:$\\
    \tau_0 =  \#, \Gamma = \{\#, n, s\}\\
    $
\\
Eine m"ogliche L"osung unter dem gew"ahlten Kelleralphabet ist:\\
\automaton{
    \nd{initial, state}{q}{0,0}{q}
    \path
        (q) 
            edge [->, loop right] node {
                \begin{tabular}{l}
                    $\eps$, \# : $\eps$\\
                    $N_1$, * : n*\\
                    $N_1$, s : $\eps$\\
                    $S_1$, * : s*\\
                    $S_1$, n : $\eps$\\
                    $N_2$, * : nn*\\
                    $N_2$, s : n\\
                    $N_2$, ss : $\eps$\\
                    $S_2$, * : ss*\\
                    $S_2$, n : s\\
                    $S_2$, nn : $\eps$\\
                \end{tabular}
            }(q);
    }

% A3
\subsection\
% A3 a)
\subsubsection{$L = \{a^lb^mc^pd^q | l,m,p,q \in \mathbb{N}_0, l<p \land q<m\}$}

wir w"ahlen:\[
    z \mdef a^n b^{n+1} c^{n+1} d{n}
\]
wir betrachten Zerlegungen der Form $z = uvwxy$, mit:$\\
vx \neq \eps\\
|vwx| \leq n\\
$\\
Fortan werden 4 F"alle betrachtet:\\
(i) mind. 1 $a$ in vx, aber kein $c$,
    da zwischen $a$ und $c$ $n+1$ $b$'s liegen.\\

(ii) mind. 1 $b$ in vx, aber kein $d$,
    da zwischen $b$ und $d$ $n+1$ $c$'s liegen.\\

(iii) mind. 1 $c$ in vx, aber kein $a$,
    da zwischen $a$ und $c$ $n+1$ $b$'s liegen.\\

(iv) mind. 1 $d$ in vx, aber kein $b$,
    da zwischen $a$ und $c$ $n+1$ $b$'s liegen.\\
\gap
(i)\\
\\
(ii)\\
\\
(iii)\\
\\
(iv)\\
\\

% A3 b)
\subsubsection{$L = \{ww^Rw | w \in \{a,b\}^*\}$}

wir w"ahlen:\[
    z \mdef ba^nbba^nbba^nb
\]
Wir betrachten Zerlegungen der Form $z = uvwxy$, mit:$\\
    vx \neq \eps\\
    |vwx| \leq n\\
$\\
Fortan werden 2 F"alle betrachtet:\\
(i) mind. 1 $b$ in $vx$\\
(ii) nur $a$'s in $vx$\\
\gap
(i)\\
\\
(ii)\\
\end{document}
