\documentclass{article}
\author{Max Springenberg, 177792}
\title{
    GTI UB11\\
    Vorueberlegungen
}
\date{}
\usepackage{amsmath}
\usepackage{amssymb}
\usepackage{stmaryrd}
\usepackage{graphicx}
\setcounter{section}{11}
% \Theta \Omega \omega
\newcommand{\tab}{\null \qquad}
\newcommand{\lA}{$\leftarrow$}
\newcommand{\ue}{$\infty$}

\begin{document}
\maketitle
\newpage
\subsection{Satz von Rice}
\begin{tabular}{l|l}
    Frage&
    Antwort\\
    \hline

    wann sind Turing-berechenbare\\
    Funktionen entscheidbar?
    &mit:\\
    &$R = \{$f $|$ f berechenbar$\}$\\
    &$\emptyset \neq S \neq R$\\
    &$TM-FUNC(S) =$ ist $f_M \in S$?\\
    &ist TM-FUNC(S) unentscheidbar\\
    \\
    wann sind Probleme entscheidbar?
    &alle Probleme mit:\\
    &$PROBLEM \leq TM-FUNC(S)$\\
    &sind ueber die Reduzierbarkeit unentscheidbar.\\
    \\
\end{tabular}

\subsection{Abschlusseigenschaften}
\begin{tabular}{l|l}
    Frage&
    Antwort\\
    \hline

    Was ist semientscheidbar?
    &Eine Menge/Sprache L ist genau dann \\
    &semientscheidbar, wenn:\\
    &$L \subseteq \Sigma^* \land \exists M.L = L(M)$\\
    &fuer eine beliebige TM M\\
    Unter welchen Vorraussetzungen kann man\\
    sagen, dass eine Menge/Sprache abhaengig\\
    von einer anderen entscheidbar ist?\\
    &L entscheidbar $\Leftrightarrow L$ und $\bar L$ semientscheidbar\\
    &
    \\
\end{tabular}

\subsection{Entscheidbarkeit / Semientscheidbarkeit}
\begin{tabular}{l|l}
    Frage&
    Antwort\\
    \hline

\end{tabular}
\end{document}
