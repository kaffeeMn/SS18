\documentclass{article}
\textwidth=6in
\hoffset=0in
\voffset=0in


\usepackage[a4paper, total={6in, 8in}]{geometry}
\usepackage{amsmath}
\usepackage{amssymb}
\usepackage{stmaryrd}
\usepackage{graphicx}

\usepackage{tikz}
\usetikzlibrary{automata, arrows}
\tikzset{initial text={}}

\usepackage{pifont}
\usepackage{amssymb}
\usepackage{gensymb}
\usepackage{ngerman}
\usepackage[ampersand]{easylist}

% needs to be updated
\author{Max Springenberg, 177792}
\title{\
    Tutorium 7\\
    }
\setcounter{section}{7}
\date{}

% custom commands
% \Theta \Omega \omega
\newcommand{\tab}{\null\ \qquad}
\newcommand{\gap}{\null\ \\ \\}
\newcommand{\lA}{\leftarrow}
\newcommand{\rA}{\rightarrow}
\newcommand{\LA}{\Leftarrow}
\newcommand{\RA}{\Rightarrow}
\newcommand{\ue}{\infty}
\newcommand{\eps}{\epsilon}
\newcommand{\task}[1]{\textbf{#1} \\ \gap}
\newcommand{\cmark}{\ding{51}}
\newcommand{\xmark}{\ding{55}}
\newcommand{\degr}{\null \degree}
\newcommand{\error}{\task{FEHLER:}}
\newcommand{\correction}{\task{KORREKTUR:}}
\newcommand{\mdef}{\overset{\text{def}}{=}}
\newcommand{\rao}[1]{\overset{#1}{\rightarrow}}
\newcommand{\automaton}[1]{
    \begin{tikzpicture}
    #1
    \end{tikzpicture}
    }
\newcommand{\nd}[4]{
    \node[#1](#2)at(#3){#4};
    }


% content
\begin{document}
% title page
\maketitle
\newpage
% actual paper

% A1
\subsection\
% A1 a)
\subsubsection\
\begin{tikzpicture}
    \node{S}
    child{
        child{node{aa}}
        child{
            node{S}
            child{
                child{
                    node{A}
                    child{node{aa}}
                }
                child{
                    node{B}
                    child{node{bb}}
                    child{
                        node{B}
                        child{node{bbb}}
                    }
                    child{node{b}}
                }
            }
        }
        child{node{bb}}
    }
    ;
\end{tikzpicture}

% A1 b)
\subsubsection\
$
S \RA^G_l aaSbb\\
S \RA^G_l a^4ABb^4\\
S \RA^G_l a^4Bbb\\
S \RA^G_l a^4bbSb^4\\
S \RA^G_l a^4b^8\\
$
rechtsableitung analog...
% A1 c)
\subsubsection\
$a^6b^6$
hat zwei verschiedene ableitungen.\\
Man betrachte bereits die ersten Ableitungsschritte:\\
$S \RA aaSbb$ und $S \RA AB$\\
$\lightning$\\

\end{document}
