\documentclass{article}
\textwidth=6in
\hoffset=0in
\voffset=0in

\usepackage[a4paper, total={6in, 8in}]{geometry}
\usepackage{amsmath}
\usepackage{amssymb}
\usepackage{ngerman}
\usepackage{stmaryrd}
\usepackage{graphicx}
\usepackage{tikz}
\usetikzlibrary{automata, arrows}
\usepackage{pifont}
\usepackage{amssymb}
\usepackage{gensymb}
\usepackage[ampersand]{easylist}

% needs to be updated
\author{Max Springenberg, 177792}
\title{\
    IS Uebungsblatt 1
    }
\setcounter{section}{1}
\date{}

% custom commands
% \Theta \Omega \omega
\newcommand{\tab}{\null\ \qquad}
\newcommand{\gap}{\null\ \\ \\}
\newcommand{\lA}{\leftarrow}
\newcommand{\rA}{\rightarrow}
\newcommand{\ue}{\infty}
\newcommand{\eps}{\epsilon}
\newcommand{\task}[1]{\textbf{#1} \\ \gap}
\newcommand{\cmark}{\ding{51}}
\newcommand{\xmark}{\ding{55}}
\newcommand{\degr}{\degree}

% content
\begin{document}
% title page
\maketitle
\newpage
% actual paper
\subsection\
\subsubsection\
Siehe UB\\
\subsubsection\
Ja z.B. dass 1 ODER 3 Betten belegt werden und nicht 2 und auch, dass 
    nur M"anner oder Frauen in einem Zimmer liegen d"urfen.\\
\subsubsection\
Entities mit Attributen:\\$
    Klinik = \{Name, KrankID, Ort\}\\
    Station = \{Name, StatID, Stock\}\\
    Zimmer = \{ID=\text{stock.nummer}\}\\
    Patient = \{
        PatID, Name, Anschrift, Versicherung, Befund
        \}\\
    Krankenschwester = \{
        MitarbID, Name, Anschrift, Gehaltsdaten, Arbeitszeit, etc.
        \}\\
    $
\subsection\
\subsubsection{Anforderungsanalyse}
Entities sind Klausuren und Lehrst"uhle.\\
Attribute wurden durch das Nennen von PNr und dem Titel f"ur die 
    Klausur verteilt.\\
Attribute wurden durch das Nennen von LehrstuhlID, Professor und Geb"aude
    f"ur den Lehrstuhl verteilt.\\
Es wurde wahrscheinlich gennant, dass ein Lehrstuhl eine Klausur stellt und
    diese Beziehung durch ein Datum festgehalten wird.\\
Ferner muss genannt worden sein, dass ein Lehrstuhl beliebig viele Klausuren
    stellen kann, aber jede Klausur genau einem Lehrstuhl zugeordnet wird.\\
\subsubsection\
Die Kardinalit"aten zeigen an, wie viele Lehst"uhle mindestens und maximal einer
    Klausur angeh"oren und wie viele Klausuren ein Lehrstuhl mindestens und 
    maximal stellt.\\
\subsubsection\
Tables der Entities:\\
\\
\begin{tabular}{lp{0.8\linewidth}}
    Entity      &SQL-Befehl\\
    \hline
    Klausur     &CREATE TABLE Klausur (PNr INTEGER NOT NULL, Titel CHAR(n))\\
    Lehrstuhl   &CREATE TABLE Lehrstuhl (
                    LID INTEGER NOT NULL,
                    Prof CHAR(n) NOT NULL, 
                    Geb"aude CHAR(n) NOT NULL
                    )\\
\end{tabular}\\
\gap
Tables der Beziehungen:\\
\\
\begin{tabular}{lp{0.8\linewidth}}
    Beziehung   &SQL-Befehl\\
    \hline
    stellt      &CREATE TABLE stellt(
                    LID INTEGER NOT NULL,
                    PNr INTEGER NOT NULL,
                    Datum CHAR(n) NOT NULL,
                    FOREIGN KEY (LID) REFERENCE Lehrstuhl,
                    FOREIGN KEY (PNr) REFERENCE Klausur
                    )\\
\end{tabular}\\
\end{document}
