\documentclass{article}
\textwidth=6in
\hoffset=0in
\voffset=0in

\usepackage[a4paper, total={6in, 8in}]{geometry}
\usepackage{amsmath}
\usepackage{amssymb}
\usepackage{stmaryrd}
\usepackage{graphicx}
\usepackage{tikz}
\usetikzlibrary{automata, arrows}
\usepackage{pifont}
\usepackage{amssymb}
\usepackage{gensymb}
\usepackage[ampersand]{easylist}

% needs to be updated
\author{Max Springenberg, 177792}
\title{\
    IS Uebungsblatt 0
    }
\setcounter{section}{0}
\date{}

% custom commands
% \Theta \Omega \omega
\newcommand{\tab}{\null\ \qquad}
\newcommand{\gap}{\null\ \\ \\}
\newcommand{\lA}{\leftarrow}
\newcommand{\rA}{\rightarrow}
\newcommand{\ue}{\infty}
\newcommand{\eps}{\epsilon}
\newcommand{\task}[1]{\textbf{#1} \\ \gap}
\newcommand{\cmark}{\ding{51}}
\newcommand{\xmark}{\ding{55}}
\newcommand{\degr}{\degree}

% content
\begin{document}
% title page
\maketitle
\newpage
% actual paper
\subsection\
Aufgaben von Datenbanken sind\\
\begin{easylist}
    & strukturierte Daten und Anfragen (effiziente Filterung von Daten)
    & Verl"asslichkeit / Fehlertoleranz
    & viele Benutzer / verschiedenartige Benutzer (Rechte)
    & Effizienz
    & gro"se Datenmengen
\end{easylist}\ \\
5 Abstraktionen:\\
\begin{easylist}
    & DatenModell (relationale Modell)
        && Semantische Modell:
            &&& Header mit Bezeichnungen(Schema)
            &&& Eintr"age sind Zeilen/ Tupel
            &&& Spalten mit Attribute
            &&& Gesamtheit der Tabelle wird als Relation bezeichnet
        && Physikalische Modell
            &&& Columnstore ((Key, Value)-Paare)
    & Anfragesprache
        && meist SQL
    & Zugangskontrolle und Datenintegrit"at
    & multi-user-support
        && Atomarit"atsprinzip
        && Isolation von Transaktionen
    & Fehlertoleranz
        && Commits unabh"angig von Systemabsturz
        && Backup- und Recovery-Mechanismen
\end{easylist}
Benutzer interagieren mit dem Datenbankmanagementsystem, genauer mit dem 
    Semantischem Modell. Dieses interagiert mit dem Physikalischem Modell.\\
\subsection\ 
\begin{easylist}
    & nein
        && kleinere Verwaltungsaufgaben erfordert keinen Datenbankserver
    & ja
        && Effizienz
        && Mehrbenutzerverwaltung
        && Struktur und Filterung
    & ja
        && Es geht nicht zwingend um die Datenmengen aber die Struktur ist 
            wichtig
    & ja
        && Konsistenz
        && Effizienz
        && Mehrbenutzerverwaltung
        && Struktur und Filterung
        && etc
\end{easylist}
\gap \ 
\begin{tabular}{l|l}
    Vorteile                        &Nachteile\\
    \hline\\
    Struktur                        &Aufwand\\
\end{tabular}
\end{document}
